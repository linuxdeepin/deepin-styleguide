\chapter{Map}
对于空 \texttt{map} 请使用 \texttt{make(..)} 初始化, 并且 \texttt{map} 是通过编程方式填充的。
这使得 \texttt{map} 初始化在表现上不同于声明,并且它还可以方便地在 \texttt{make} 后添加大小提示。
\begin{itemize}[leftmargin=4em]
\item 错误用法

  \begin{minted}{go}
    // 声明和初始化看起来非常相似的
    var (
    	// m1 读写安全;
    	// m2 在写入时会 panic
    	m1 = map[T1]T2{}
    	m2 map[T1]T2
    )
  \end{minted}
\item 正确用法

  \begin{minted}{go}
    // 声明和初始化看起来差别非常大
    var (
    	// m1 读写安全;
    	// m2 在写入时会 panic
    	m1 = make(map[T1]T2)
    	m2 map[T1]T2
    )
  \end{minted}
\end{itemize}

尽可能在初始化时提供 \texttt{map} 容量大小。如果 \texttt{map} 包含固定的元素列表,
则使用 \texttt{map literals}(\texttt{map} 初始化列表)初始化映射。
\begin{itemize}[leftmargin=4em]
\item 错误用法

  \begin{minted}{go}
    m := make(map[T1]T2, 3)
    m[k1] = v1
    m[k2] = v2
    m[k3] = v3
  \end{minted}
\item 正确用法

  \begin{minted}{go}
    m := map[T1]T2{
      k1: v1,
      k2: v2,
      k3: v3,
    }
  \end{minted}
\end{itemize}

基本准则是:在初始化时使用 \texttt{map} 初始化列表来添加一组固定的元素。
否则使用 \texttt{make}(如果可以,尽量指定 \texttt{map} 容量)。

\section{指定容器容量}
尽可能指定容器容量,以便为容器预先分配内存。这将在添加元素时最小化后续分配(通过复制和调整容器大小)。

指定 \texttt{Map} 容量提示,尽可能在使用 \texttt{make()} 初始化的时候提供容量信息。
\begin{minted}[xleftmargin=3.5em]{go}
  make(map[T1]T2, hint)
\end{minted}

向 \texttt{make()} 提供容量提示会在初始化时尝试调整 \texttt{map} 的大小,这将减少在将元素添加到 \texttt{map} 时为 \texttt{map} 重新分配内存。
与 \texttt{slices} 不同, \texttt{map capacity} 提示并不保证完全的抢占式分配,而是用于估计所需的 \texttt{hashmap bucket} 的数量。 因此,在将元素添加到 \texttt{map} 时,甚至在指定 \texttt{map} 容量时,仍可能发生分配。
\begin{itemize}[leftmargin=4em]
\item 错误用法

  \begin{minted}{go}
    // m 是在没有大小提示的情况下创建的; 在运行时可能会有更多分配
    m := make(map[string]os.FileInfo)

    files, _ := ioutil.ReadDir("./files")
    for _, f := range files {
    	m[f.Name()] = f
    }
  \end{minted}
\item 正确用法

  \begin{minted}{go}
    // m 是有大小提示创建的;在运行时可能会有更少的分配
    files, _ := ioutil.ReadDir("./files")

    m := make(map[string]os.FileInfo, len(files))
    for _, f := range files {
    	m[f.Name()] = f
    }
  \end{minted}
\end{itemize}

在尽可能的情况下,在使用 \texttt{make()} 初始化切片时提供容量信息,特别是在追加切片时。
\begin{minted}[xleftmargin=3.5em]{go}
  make([]T, length, capacity)
\end{minted}

与 \texttt{maps} 不同, \texttt{slice capacity} 不是一个提示:
编译器将为提供给 \texttt{make()} 的 \texttt{slice} 的容量分配足够的内存,
这意味着后续的 \texttt{append()} 操作将导致零分配(直到 \texttt{slice} 的长度与容量匹配,
在此之后,任何 \texttt{append} 都可能调整大小以容纳其它元素)。
