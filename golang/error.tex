\chapter{Error}
\texttt{Go} 中有多种声明错误(Error) 的选项:
\begin{itemize}[leftmargin=4em]
\item \texttt{errors.New} 对于简单静态字符串的错误;
\item \texttt{fmt.Errorf} 用于格式化的错误字符串;
\item 实现 \texttt{Error()} 方法的自定义类型;
\item 用 \texttt{"pkg/errors".Wrap} 的 \texttt{Wrapped errors} 。
\end{itemize}

返回错误时,考虑以下因素以确定最佳选择:
\begin{itemize}[leftmargin=4em]
\item 这是一个不需要额外信息的简单错误吗?如果是这样, \texttt{errors.New} 足够了;
\item 客户需要检测并处理此错误吗?如果是这样,则应使用自定义类型并实现该 \texttt{Error()} 方法;
\item 当前是否正在传播下游函数返回的错误?如果是这样,使用错误包装;
\item 否则 \texttt{fmt.Errorf} 就可以了。
\end{itemize}

如果客户端需要检测错误,并且已使用创建了一个简单的错误 \texttt{errors.New} ,请使用一个错误变量。
\begin{itemize}[leftmargin=4em]
\item 错误用法

  \begin{minted}{go}
    // package foo

    func Open() error {
    	return errors.New("could not open")
    }


    // package bar

    func use() {
    	if err := foo.Open(); err != nil {
    		if err.Error() == "could not open" {
    			// handle
    		} else {
    			panic("unknown error")
    		}
    	}
    }
  \end{minted}
\item 正确用法

  \begin{minted}{go}
    // package foo

    var ErrCouldNotOpen = errors.New("could not open")

    func Open() error {
    	return ErrCouldNotOpen
    }

    // package bar

    if err := foo.Open(); err != nil {
    	if errors.Is(err, foo.ErrCouldNotOpen) {
    		// handle
    	} else {
    		panic("unknown error")
    	}
    }
  \end{minted}
\end{itemize}

如果有可能需要客户端检测的错误,并且想向其中添加更多信息(例如,它不是静态字符串),则应使用自定义类型。
\begin{itemize}[leftmargin=4em]
\item 错误用法

  \begin{minted}{go}
    func open(file string) error {
    	return fmt.Errorf("file %q not found", file)
    }

    func use() {
    	if err := open("testfile.txt"); err != nil {
    		if strings.Contains(err.Error(), "not found") {
    			// handle
    		} else {
    			panic("unknown error")
    		}
    	}
    }
  \end{minted}
\item 正确用法

  \begin{minted}{go}
    type errNotFound struct {
    	file string
    }

    func (e errNotFound) Error() string {
    	return fmt.Sprintf("file %q not found", e.file)
    }

    func open(file string) error {
    	return errNotFound{file: file}
    }

    func use() {
    	if err := open("testfile.txt"); err != nil {
    		if _, ok := err.(errNotFound); ok {
    			// handle
    		} else {
    			panic("unknown error")
    		}
    	}
    }
  \end{minted}
\end{itemize}

直接导出自定义错误类型时要小心,因为它们已成为程序包公共 API 的一部分。最好公开匹配器功能以检查错误。
\begin{minted}[xleftmargin=3.5em]{go}
  // package foo

  type errNotFound struct {
  	file string
  }

  func (e errNotFound) Error() string {
  	return fmt.Sprintf("file %q not found", e.file)
  }

  func IsNotFoundError(err error) bool {
  	_, ok := err.(errNotFound)
  	return ok
  }

  func Open(file string) error {
  	return errNotFound{file: file}
  }


  // package bar

  if err := foo.Open("foo"); err != nil {
  	if foo.IsNotFoundError(err) {
  		// handle
  	} else {
  		panic("unknown error")
  	}
  }
\end{minted}

\section{Error Wrapping}
一个(函数/方法)调用失败时,有三种主要的传播错误方式:
\begin{itemize}[leftmargin=4em]
\item 如果没有要添加的其它上下文,并且要维护原始错误类型,则返回原始错误;
\item 添加上下文,使用 \texttt{"pkg/errors".Wrap} 以便错误消息提供更多上下文, \texttt{"pkg/errors".Cause} 可用于提取原始错误;
\item 如果调用者不需要检测或处理的特定错误情况,使用 \texttt{fmt.Errorf} 。
\end{itemize}

建议在可能的地方添加上下文,用来获得诸如“调用服务 foo:连接被拒绝”之类的更有用的错误,而不是诸如“连接被拒绝”之类的模糊错误。

在将上下文添加到返回的错误时,请避免使用“failed to”之类的短语以保持上下文简洁,
这些短语会陈述明显的内容,并随着错误在堆栈中的渗透而逐渐堆积:
\begin{itemize}[leftmargin=4em]
\item 错误用法

  \begin{minted}{go}
    s, err := store.New()
    if err != nil {
    	return fmt.Errorf("failed to create new store: %v", err)
    }
  \end{minted}
\item 正确用法

  \begin{minted}{go}
    s, err := store.New()
    if err != nil {
    	return fmt.Errorf("new store: %v", err)
    }
  \end{minted}
\end{itemize}

但是,一旦将错误发送到另一个系统,就应该明确消息是错误消息(例如使用err标记,或在日志中以”Failed”为前缀)。

\section{处理类型断言失败}
\texttt{type assertion} 的单个返回值形式针对不正确的类型将产生 \texttt{panic} 。因此,始终使用“comma ok”的惯用法。
\begin{itemize}[leftmargin=4em]
\item 错误用法

  \begin{minted}{go}
    t := i.(string)
  \end{minted}
\item 正确用法

  \begin{minted}{go}
    t, ok := i.(string)
    if !ok {
    	// 优雅地处理错误
    }
  \end{minted}
\end{itemize}
