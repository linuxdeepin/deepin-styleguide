\chapter{分组}
Go 语言支持将相似的声明放在一个组内,适用于常量、变量和类型声明,不要将不相关的声明放在一组。

\begin{itemize}[leftmargin=4em]
\item 错误用法

  \begin{minted}{go}
    const _STATE_SUCESS = 1
    const _STATE_FAILED = 2
  \end{minted}
\item 正确用法

  \begin{minted}{go}
    const (
    	_STATE_SUCESS = iota + 1
    	_STATE_FAILED
    )
  \end{minted}
\end{itemize}

分组使用的位置没有限制,例如:你可以在函数内部使用它们。
\begin{itemize}[leftmargin=4em]
\item 错误用法

  \begin{minted}{go}
    func calcCost() {
    	var total int
    	idx := 0
    }
  \end{minted}
\item 正确用法

  \begin{minted}{go}
    func calcCost() {
    	var (
    		total int
    		idx int
    	)
    }
  \end{minted}
\end{itemize}

\section{Import}
分组对 \texttt{import} 也同样适用,但标准库和第三方库需要隔开,例如:
\begin{itemize}[leftmargin=4em]
\item 错误用法

  \begin{minted}{go}
    import (
    	"os"
    	"os/exec"
    	"example.com/client-go"
    	"example.com/http"
    )
  \end{minted}
\item 正确用法

  \begin{minted}{go}
    import (
    	"os"
    	"os/exec"

    	"example.com/client-go"
    	"example.com/http"
    )
  \end{minted}
\end{itemize}

推荐使用 \texttt{goimports} 导入包,其会进行自动分组。

\section{函数}
函数在编写时,也应按照以下原则进行分组:
\begin{itemize}[leftmargin=4em]
\item 函数应在 \texttt{struct、const、var} 等定义的后面;
\item 导出的函数应先出现在文件中;
\item 相同接受者的函数应在一起;
\item 普通工具函数应在文件末尾;
\item 函数应按调用顺序排序。
\end{itemize}

如:
\begin{itemize}[leftmargin=4em]
\item 错误用法

  \begin{minted}{go}
    func (s *something) Cost() {
    	return calcCost(s.weights)
    }

    type something struct{ ... }

    func calcCost(n []int) int {...}

    func (s *something) Stop() {...}

    func newSomething() *something {
    	return &something{}
    }
  \end{minted}
\item 正确用法

  \begin{minted}{go}
    type something struct{ ... }

    func newSomething() *something {
    	return &something{}
    }

    func (s *something) Cost() {
    	return calcCost(s.weights)
    }

    func (s *something) Stop() {...}

    func calcCost(n []int) int {...}
  \end{minted}
\end{itemize}
