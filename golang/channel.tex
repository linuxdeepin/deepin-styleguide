\chapter{Channel}
\textbf{channel 的 size 要么是 1,要么是无缓冲的。}

\texttt{channel} 通常 \texttt{size} 应为 \texttt{1} 或是无缓冲的。默认情况下,\texttt{channel} 是无缓冲的,其 \texttt{size} 为零。
任何其它尺寸都必须经过严格的审查,确定是否是通道边界,竞态条件,以及逻辑上下文导致需要 \texttt{size} 大于 \texttt{1} ,尽量从源头进行排查。
\begin{itemize}[leftmargin=4em]
\item 错误用法

  \begin{minted}{go}
    // 应该足以满足任何情况!
    c := make(chan int, 64)
  \end{minted}
\item 正确用法

  \begin{minted}{go}
    // 大小:1
    c := make(chan int, 1) // 或者
    // 无缓冲 channel,大小为 0
    c := make(chan int)
  \end{minted}
\end{itemize}

\section{禁止重复释放channel}
重复释放一般存在于异常流程判断中,如果恶意攻击者构造出异常条件使程序重复释放 \texttt{channel} ,则会触发运行时恐慌,从而造成 \texttt{DoS} 攻击。

\begin{itemize}[leftmargin=4em]
\item 错误用法

  下面代码中多次关掉channel会触发运行时错误。
  \begin{minted}{go}
    func foo(c chan int) {
    	defer close(c)
    	err := processBusiness()
    	if err != nil {
    		c <- 0
    		close(c) // 【错误】重复释放channel
    		return
    	}
    	c <- 1
    }
  \end{minted}
\item 正确用法

  使用defer延迟关闭channel,并且确保channel只释放一次。
  \begin{minted}{go}
    func foo(c chan int) {
    	defer close(c) // 【修改】使用defer延迟关闭channel
    	err := processBusiness()
    	if err != nil {
    		c <- 0
    		return
    	}
    	c <- 1
    }
  \end{minted}
\end{itemize}
