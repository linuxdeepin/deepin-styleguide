\chapter{变量}
\section{变量声明}
使用标准 \texttt{var} 关键字。请勿指定类型,除非它与表达式的类型不同。
\begin{itemize}[leftmargin=4em]
\item 错误用法

  \begin{minted}{go}
    var _s string = F()

    func F() string { return "A" }
  \end{minted}
\item 正确用法

  \begin{minted}{go}
    // 由于 F 已经明确了返回一个字符串类型,因此我们没有必要显式指定 _s 的类型
    var _s = F()

    func F() string { return "A" }
  \end{minted}
\end{itemize}

如果表达式的类型与所需的类型不完全匹配,请指定类型。如:
\begin{minted}[xleftmargin=3.5em]{go}
  type myError struct{}

  func (myError) Error() string { return "error" }

  func F() myError { return myError{} }

  var _e error = F()
\end{minted}

如果将变量明确设置为某个值,则应使用短变量声明形式 (\texttt{:=})。
\begin{itemize}[leftmargin=4em]
\item 错误用法

  \begin{minted}{go}
    var s = "foo"
  \end{minted}
\item 正确用法

  \begin{minted}{go}
    s := "foo"
  \end{minted}
\end{itemize}

但是,在某些情况下, \texttt{var} 使用关键字时默认值会更清晰。例如,声明空切片。
\begin{itemize}[leftmargin=4em]
\item 错误用法

  \begin{minted}{go}
    func f(list []int) {
    	filtered := []int{}
    	for _, v := range list {
    		if v > 10 {
    			filtered = append(filtered, v)
    		}
    	}
    }
  \end{minted}
\item 正确用法

  \begin{minted}{go}
    func f(list []int) {
    	var filtered []int
    	for _, v := range list {
    		if v > 10 {
    			filtered = append(filtered, v)
    		}
    	}
    }
  \end{minted}
\end{itemize}

\section{缩小变量作用域}
如果有可能,尽量缩小变量作用范围。除非它与减少嵌套的规则冲突。
\begin{itemize}[leftmargin=4em]
\item 错误用法

  \begin{minted}{go}
    err := ioutil.WriteFile(name, data, 0644)
    if err != nil {
    	return err
    }
  \end{minted}
\item 正确用法

  \begin{minted}{go}
    if err := ioutil.WriteFile(name, data, 0644); err != nil {
    	return err
    }
  \end{minted}
\end{itemize}

如果需要在 if 之外使用函数调用的结果,则不应尝试缩小范围。
\begin{itemize}[leftmargin=4em]
\item 错误用法

  \begin{minted}{go}
    if data, err := ioutil.ReadFile(name); err == nil {
    	err = cfg.Decode(data)
    	if err != nil {
    		return err
    	}

    	fmt.Println(cfg)
    	return nil
    } else {
    	return err
    }
  \end{minted}
\item 正确用法

  \begin{minted}{go}
    data, err := ioutil.ReadFile(name)
    if err != nil {
    	return err
    }

    if err := cfg.Decode(data); err != nil {
    	return err
    }

    fmt.Println(cfg)
    return nil
  \end{minted}
\end{itemize}

\section{使用原始字符串字面值}
\texttt{Go} 支持使用原始字符串字面值,也就是 '`' 来表示原生字符串,在需要转义的场景下,我们应该尽量使用这种方案来替换。

可以跨越多行并包含引号。使用这些字符串可以避免更难阅读的手工转义的字符串。

\begin{itemize}[leftmargin=4em]
\item 错误用法

  \begin{minted}{go}
    wantError := "unknown name:\"test\""
  \end{minted}
\item 正确用法

  \begin{minted}{go}
    wantError := `unknown error:"test"`
  \end{minted}
\end{itemize}

\section{避免可变全局变量}
使用选择依赖注入方式避免改变全局变量;既适用于函数指针又适用于其它值类型。
\begin{itemize}[leftmargin=4em]
\item 错误用法

  \begin{minted}{go}
    // sign.go
    var _timeNow = time.Now
    func sign(msg string) string {
    	now := _timeNow()
    	return signWithTime(msg, now)
    }


    // sign_test.go
    func TestSign(t *testing.T) {
    	oldTimeNow := _timeNow
    	_timeNow = func() time.Time {
    		return someFixedTime
    	}
    	defer func() { _timeNow = oldTimeNow }()
    	assert.Equal(t, want, sign(give))
    }
  \end{minted}
\item 正确用法

  \begin{minted}{go}
    // sign.go
    type signer struct {
    	now func() time.Time
    }
    func newSigner() *signer {
    	return &signer{
    		now: time.Now,
    	}
    }
    func (s *signer) Sign(msg string) string {
    	now := s.now()
    	return signWithTime(msg, now)
    }


    // sign_test.go
    func TestSigner(t *testing.T) {
    	s := newSigner()
    	s.now = func() time.Time {
    		return someFixedTime
    	}
    	assert.Equal(t, want, s.Sign(give))
    }
  \end{minted}
\end{itemize}
