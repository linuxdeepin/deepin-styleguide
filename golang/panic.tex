\chapter{Panic}
在生产环境中运行的代码必须避免出现 \texttt{panic} 。 \texttt{panic} 是 \texttt{cascading failures} 级联失败的主要根源 。
如果发生错误,该函数必须返回错误,并允许调用方决定如何处理它。
\begin{itemize}[leftmargin=4em]
\item 错误用法

  \begin{minted}{go}
    func run(args []string) {
    	if len(args) == 0 {
    		panic("an argument is required")
    	}
    	// ...
    }

    func main() {
    	run(os.Args[1:])
    }
  \end{minted}
\item 正确用法

  \begin{minted}{go}
    func run(args []string) error {
    	if len(args) == 0 {
    		return errors.New("an argument is required")
    	}
    	// ...
    	return nil
    }

    func main() {
    	if err := run(os.Args[1:]); err != nil {
    		fmt.Fprintln(os.Stderr, err)
    		os.Exit(1)
    	}
    }
  \end{minted}
\end{itemize}

\texttt{panic/recover} 不是错误处理策略。仅当发生不可恢复的事情(例如: \texttt{nil} 引用)时,程序才必须 \texttt{panic} 。
程序初始化是一个例外:程序启动时应使程序中止的不良情况可能会引起 \texttt{panic} 。
\begin{minted}[xleftmargin=3.5em,breaklines]{go}
var _statusTemplate = template.Must(template.New("name").Parse("_statusHTML"))
\end{minted}

即使在测试代码中,也优先使用 \texttt{t.Fatal} 或者 \texttt{t.FailNow} 而不是 \texttt{panic} 来确保失败被标记。
\begin{itemize}[leftmargin=4em]
\item 错误用法

  \begin{minted}{go}
    // func TestFoo(t *testing.T)

    f, err := ioutil.TempFile("", "test")
    if err != nil {
    	panic("failed to set up test")
    }
  \end{minted}
\item 正确用法

  \begin{minted}{go}
    // func TestFoo(t *testing.T)

    f, err := ioutil.TempFile("", "test")
    if err != nil {
    	t.Fatal("failed to set up test")
    }
  \end{minted}
\end{itemize}
