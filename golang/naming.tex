\chapter{命名}
变量名和函数使用驼峰命名法,首字母小写,但如果需要导出,首字母则应该大写。名称应能准确表达其含义,且容易记忆。

代码中应避免使用全局变量,若要使用全局变量,并且无需对外导出时,则在变量名前添加 \texttt{'\_'} 。

\begin{itemize}[leftmargin=4em]
\item 错误用法

  \begin{minted}{go}
    var count int

    func calc_cost() {
    	var Total int
    }
  \end{minted}
\item 正确用法

  \begin{minted}{go}
    var _count int

    func calcCost() {
    	var total int
    }
  \end{minted}
\end{itemize}

\section{避免使用内置名称}
Go 语言规范概述了几个内置的,不应在 Go 项目中使用的名称标识 \texttt{predeclared identifiers} 。

根据上下文的不同,将这些标识符作为名称重复使用, 将在当前作用域(或任何嵌套作用域)中隐藏原始标识符,或者混淆代码。
在最好的情况下,编译器会报错;在最坏的情况下,这样的代码可能会引入潜在的、难以恢复的错误。
\begin{itemize}[leftmargin=4em]
\item 错误用法

  \begin{minted}[breaklines]{go}
    var error string
    // `error` 作用域隐式覆盖
    // or

    func handleErrorMessage(error string) {
    	// `error` 作用域隐式覆盖
    }

    type Foo struct {
    	// 虽然这些字段在技术上不构成阴影,但`error`或`string`字符串的重映射现在是不明确的。
    	error  error
    	string string
    }

    func (f Foo) Error() error {
    	// `error` 和 `f.error` 在视觉上是相似的
    	return f.error
    }

    func (f Foo) String() string {
    	// `string` and `f.string` 在视觉上是相似的
    	return f.string
    }
  \end{minted}
\item 正确用法

  \begin{minted}{go}
    var errorMessage string
    // `error` 指向内置的非覆盖
    // or

    func handleErrorMessage(msg string) {
    	// `error` 指向内置的非覆盖
    }
    type Foo struct {
    	// `error` and `string` 现在是明确的。
    	err error
    	str string
    }

    func (f Foo) Error() error {
    	return f.err
    }

    func (f Foo) String() string {
    	return f.str
    }
  \end{minted}
\end{itemize}

注意,编译器在使用预先分配的标识符时不会生成错误,但是诸如 \texttt{go vet} 之类的工具会正确地指出这些和其它情况下的隐式问题。

\section{常量}
常量因使用蛇形命名法,全局常量需要全部大写。无需对外导出的全局常量,须在常量名前添加 \texttt{'\_'} 。

\begin{itemize}[leftmargin=4em]
\item 错误用法

  \begin{minted}{go}
    const maxBufSize = 1024
    const Version = "1.0"
  \end{minted}
\item 正确用法

  \begin{minted}{go}
    const _MAX_BUF_SIZE = 1024
    const VERSION = "1.0"
  \end{minted}
\end{itemize}

\section{包名}
包名需满足以下规则:
\begin{itemize}[leftmargin=4em]
\item 全部小写,没有大写或下划线及连接符;
\item 包名应避免重复;
\item 简短且简洁,容易记忆;
\item 不用复数;
\item 避免使用 \texttt{common/util/shared/lib} 这类模糊不清的包名。
\end{itemize}

如:
\begin{itemize}[leftmargin=4em]
\item 错误用法

  \begin{minted}{go}
    package URL
    package urls
  \end{minted}
\item 正确用法

  \begin{minted}{go}
    package url
  \end{minted}
\end{itemize}
