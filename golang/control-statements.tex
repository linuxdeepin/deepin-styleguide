\chapter{控制语句}
代码应通过尽可能先处理错误情况和特殊情况,并尽早返回或继续循环来减少嵌套。减少嵌套多个级别的代码的代码量。
\begin{itemize}[leftmargin=4em]
\item 错误用法

  \begin{minted}{go}
    for _, v := range data {
    	if v.F1 == 1 {
    		v = process(v)
    		if err := v.Call(); err == nil {
    			v.Send()
    		} else {
    			return err
    		}
    	} else {
    		log.Printf("Invalid v: %v", v)
    	}
    }
  \end{minted}
\item 正确用法

  \begin{minted}{go}
    for _, v := range data {
    	if v.F1 != 1 {
    		log.Printf("Invalid v: %v", v)
    		continue
    	}

    	v = process(v)
    	if err := v.Call(); err != nil {
    		return err
    	}
    	v.Send()
    }
  \end{minted}
\end{itemize}

去掉不必要的 \texttt{else} ,如果在 \texttt{if} 的两个分支中都设置了变量,则可以将其替换为单个 \texttt{if} 。
\begin{itemize}[leftmargin=4em]
\item 错误用法

  \begin{minted}{go}
    var a int
    if b {
    	a = 100
    } else {
    	a = 10
    }
  \end{minted}
\item 正确用法

  \begin{minted}{go}
    a := 10
    if b {
    	a = 100
    }
  \end{minted}
\end{itemize}
