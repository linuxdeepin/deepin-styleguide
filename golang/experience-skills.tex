\chapter{经验技巧}
\section{优先使用 strconv}
将数字、布尔值转换为字符串或从字符串转换时, \texttt{strconv} 速度比 \texttt{fmt} 快。
\begin{itemize}[leftmargin=4em]
\item 错误用法

  \begin{minted}{go}
    for i := 0; i < b.N; i++ {
    	s := fmt.Sprint(rand.Int())
    }
  \end{minted}
\item 正确用法

  \begin{minted}{go}
    for i := 0; i < b.N; i++ {
    	s := strconv.Itoa(rand.Int())
    }
  \end{minted}
\end{itemize}

\section{避免字符串到字节的转换}
不要反复从固定字符串创建字节 \texttt{slice} 。相反,请执行一次转换并捕获结果。
\begin{itemize}[leftmargin=4em]
\item 错误用法

  \begin{minted}{go}
    for i := 0; i < b.N; i++ {
    	w.Write([]byte("Hello world"))
    }
  \end{minted}
\item 正确用法

  \begin{minted}{go}
    data := []byte("Hello world")
    for i := 0; i < b.N; i++ {
    	w.Write(data)
    }
  \end{minted}
\end{itemize}

\section{零值 Mutex 是有效的}
零值 \texttt{sync.Mutex} 和 \texttt{sync.RWMutex} 是有效的。所以指向 \texttt{mutex} 的指针基本是不必要的。
\begin{itemize}[leftmargin=4em]
\item 错误用法

  \begin{minted}{go}
    mu := new(sync.Mutex)
    mu.Lock()
  \end{minted}
\item 正确用法

  \begin{minted}{go}
    var mu sync.Mutex
    mu.Lock()
  \end{minted}
\end{itemize}

如果你使用结构体指针, \texttt{mutex} 应该作为结构体的非指针字段。即使该结构体不被导出,也不要直接把 \texttt{mutex} 嵌入到结构体中。
\begin{itemize}[leftmargin=4em]
\item 错误用法

  \begin{minted}{go}
    // Mutex 字段, Lock 和 Unlock 方法是 SMap 导出的 API 中不刻意说明的一部分
    type SMap struct {
    	sync.Mutex

    	data map[string]string
    }

    func NewSMap() *SMap {
    	return &SMap{
    		data: make(map[string]string),
    	}
    }

    func (m *SMap) Get(k string) string {
    	m.Lock()
    	defer m.Unlock()

    	return m.data[k]
    }
  \end{minted}
\item 正确用法

  \begin{minted}{go}
    // mutex 及其方法是 SMap 的实现细节,对其调用者不可见
    type SMap struct {
    	mu sync.Mutex

    	data map[string]string
    }

    func NewSMap() *SMap {
    	return &SMap{
    		data: make(map[string]string),
    	}
    }

    func (m *SMap) Get(k string) string {
    	m.mu.Lock()
    	defer m.mu.Unlock()

    	return m.data[k]
    }
  \end{minted}
\end{itemize}

\section{在边界处拷贝 Slices 和 Maps}
\texttt{Slices} 和 \texttt{Maps} 包含了指向底层数据的指针,因此在需要复制它们时要特别注意。

接收 \texttt{Slices} 和 \texttt{Maps} 需注意,当 \texttt{map} 或 \texttt{slice} 作为函数参数传入时,如果存储了对它们的引用,则用户可以对其进行修改。
\begin{itemize}[leftmargin=4em]
\item 错误用法

  \begin{minted}{go}
    func (d *Driver) SetTrips(trips []Trip) {
    	d.trips = trips
    }

    trips := ...
    d1.SetTrips(trips)

    // 你是要修改 d1.trips 吗?
    trips[0] = ...
  \end{minted}
\item 正确用法

  \begin{minted}{go}
    func (d *Driver) SetTrips(trips []Trip) {
    	d.trips = make([]Trip, len(trips))
    	copy(d.trips, trips)
    }

    trips := ...
    d1.SetTrips(trips)

    // 这里我们修改 trips[0],但不会影响到 d1.trips
    trips[0] = ...
  \end{minted}
\end{itemize}

返回 \texttt{Slices} 或 \texttt{Maps} 同样需注意,用户对暴露内部状态的 \texttt{map} 或 \texttt{slice} 可修改。
\begin{itemize}[leftmargin=4em]
\item 错误用法

  \begin{minted}{go}
    type Stats struct {
    	mu sync.Mutex

    	counters map[string]int
    }



    // Snapshot 返回当前状态。
    func (s *Stats) Snapshot() map[string]int {
    	s.mu.Lock()
    	defer s.mu.Unlock()

    	return s.counters
    }

    // snapshot 不再受互斥锁保护
    // 因此对 snapshot 的任何访问都将受到数据竞争的影响
    // 影响 stats.counters
    snapshot := stats.Snapshot()
  \end{minted}
\item 正确用法

  \begin{minted}{go}
    type Stats struct {
    	mu sync.Mutex

    	counters map[string]int
    }

    func (s *Stats) Snapshot() map[string]int {
    	s.mu.Lock()
    	defer s.mu.Unlock()

    	result := make(map[string]int, len(s.counters))
    	for k, v := range s.counters {
    		result[k] = v
    	}
    	return result
    }

    // snapshot 现在是一个拷贝
    snapshot := stats.Snapshot()
  \end{minted}
\end{itemize}

\section{使用 time 处理时间}
时间处理很复杂。关于时间的错误假设通常包括以下几点。
\begin{enumerate}[leftmargin=4em]
\item 一天有 24 小时;
\item 一小时有 60 分钟;
\item 一周有七天;
\item 一年 365 天;
\item 还有更多。
\end{enumerate}

例如,1 表示在一个时间点上加上 24 小时并不总是产生一个新的日历日。

因此,在处理时间时始终使用 "time" 包,因为它有助于以更安全、更准确的方式处理这些不正确的假设。

\textbf{使用 time.Time 表达瞬时时间:}在处理时间的瞬间时使用 \texttt{time.Time} ,在比较、添加或减去时间时使用 \texttt{time.Time} 中的方法。
\begin{itemize}[leftmargin=4em]
\item 错误用法

  \begin{minted}{go}
    func isActive(now, start, stop int) bool {
    	return start <= now && now < stop
    }
  \end{minted}
\item 正确用法

  \begin{minted}[breaklines]{go}
    func isActive(now, start, stop time.Time) bool {
    	return (start.Before(now) || start.Equal(now)) && now.Before(stop)
    }
  \end{minted}
\end{itemize}

使用 \texttt{time.Duration} 表达时间段,在处理时间段时也使用 \texttt{time.Duration} 。

\begin{itemize}[leftmargin=4em]
\item 错误用法

  \begin{minted}{go}
    func poll(delay int) {
    	for {
    		// ...
    		time.Sleep(time.Duration(delay) * time.Millisecond)
    	}
    }
    poll(10) // 是几秒钟还是几毫秒?
  \end{minted}
\item 正确用法

  \begin{minted}{go}
    func poll(delay time.Duration) {
    	for {
    		// ...
    		time.Sleep(delay)
    	}
    }
    poll(10*time.Second)
  \end{minted}
\end{itemize}

回到第一个例子,在一个时间瞬间加上 24 小时,我们用于添加时间的方法取决于意图。
如果我们想要下一个日历日(当前天的下一天)的同一个时间点,我们应该使用 \texttt{Time.AddDate} 。
但是,如果我们想保证某一时刻比前一时刻晚 24 小时,我们应该使用 \texttt{Time.Add} 。

\begin{minted}[xleftmargin=2em]{go}
  newDay := t.AddDate(0 /* years */, 0 /* months */, 1 /* days */)
  maybeNewDay := t.Add(24 * time.Hour)
\end{minted}

对外部系统使用 \texttt{time.Time} 和 \texttt{time.Duration} ,尽可能在与外部系统的交互中使用 \texttt{time.Duration} 和 \texttt{time.Time} 例如 :
\begin{itemize}
\item \texttt{Command-line} 标志: \texttt{flag} 通过 \texttt{time.ParseDuration} 支持 \texttt{time.Duration} ;
\item \texttt{JSON: encoding/json} 通过其 \texttt{UnmarshalJSON method} 方法支持将 \texttt{time.Time} 编码为 \texttt{RFC 3339} 字符串;
\item \texttt{SQL: database/sql} 支持将 \texttt{DATETIME} 或 \texttt{TIMESTAMP} 列转换为 \texttt{time.Time} ,如果底层驱动程序支持则返回;
\item \texttt{YAML: gopkg.in/yaml.v2} 支持将 \texttt{time.Time} 作为 \texttt{RFC 3339} 字符串,并通过 \texttt{time.ParseDuration} 支持 \texttt{time.Duration} 。
\end{itemize}

当不能在这些交互中使用 \texttt{time.Duration} 时,请使用 \texttt{int} 或 \texttt{float64},并在字段名称中包含单位。
例如,由于 \texttt{encoding/json} 不支持 \texttt{time.Duration} ,因此该单位包含在字段的名称中。

\begin{itemize}[leftmargin=4em]
\item 错误用法

  \begin{minted}{go}
    // {"interval": 2}
    type Config struct {
    	Interval int `json:"interval"`
    }
  \end{minted}
\item 正确用法

  \begin{minted}{go}
    // {"intervalMillis": 2000}
    type Config struct {
    	IntervalMillis int `json:"intervalMillis"`
    }
  \end{minted}
\end{itemize}

当在这些交互中不能使用 \texttt{time.Time} 时,除非达成一致,否则使用 \texttt{string} 和 \texttt{RFC 3339} 中定义的格式时间戳。
默认情况下, \texttt{Time.UnmarshalText} 使用此格式,并可通过 \texttt{time.RFC3339} 在 \texttt{Time.Format} 和 \texttt{time.Parse} 中使用。

尽管这在实践中并不成问题,但请记住,"time" 包不支持解析闰秒时间戳,也不在计算中考虑闰秒。
如果比较两个时间瞬间,则差异将不包括这两个瞬间之间可能发生的闰秒。

\section{使用 crypto random 生成随机数}
不要使用 \texttt{math/rand} 生成随机数,甚至是一次性的。没有随机种子,生成器的结果是完全可预测的。
即便使用 \texttt{time.Nanoseconds} 初始化随机种子,也只能产生有限的几位熵。

相反,使用 \texttt{crypto/rand} 生成的随机数,使用的是内核的 \texttt{/dev/urandom} 设备,更加安全可靠。

\begin{minted}[xleftmargin=2em]{go}
import (
	"crypto/rand"
	// "encoding/base64"
	// "encoding/hex"
	"fmt"
)

func Key() string {
	buf := make([]byte, 16)
	_, err := rand.Read(buf)
	if err != nil {
		panic(err)  // out of randomness, should never happen
	}
	return fmt.Sprintf("%x", buf)
	// or hex.EncodeToString(buf)
	// or base64.StdEncoding.EncodeToString(buf)
}
\end{minted}

\section{整数安全}
整数在使用时应注意有无符号,避免出现以下错误:
\begin{itemize}[leftmargin=4em]
\item 无符号整数运算时出现反转

  \textbf{反转} 是指无法用无符号整数表示的运算结果,这个结果将会根据该类型可以表示的最大值加1执行求模操作。
\item 有符号整数运算时出现溢出

  \textbf{整数溢出}是一种未定义的行为,意味着编译器在处理有符号整数溢出时具有很多选择。
\item 整型转换时出现截断错误

  将一个较大整型转换为较小整型,并且该数的原值超出较小整型的表示范围,就会发生截断错误,原值的低位被保留而高位被丢弃。
  截断错误会引起数据丢失,甚至可能引发安全问题。
\item 整型转换时出现符号错误

  有时从带符号整型向无符号整型转换时,最高位会丧失作为符号位的功能,即产生符号丢失但数据不丢失的问题,从而数据失去原来的含义。
  \begin{itemize}
  \item 带符号整型数的值非负时,它向无符号整型转换后,值不变;
  \item 带符号整型数的值为负时,它向无符号整型转换后,结果通常是一个非常大的正数;
  \end{itemize}
\end{itemize}

整数通常会在以下操作符中被使用:\texttt{+、-、*、/、\%、++、--、=、+=、-=、*=、/=、\%=、<<=、<<、–} 。
最后的\texttt{'-'} 表示一元否定(\texttt{unary negation})。

将运算结果用于以下之一的用途,应防止反转、溢出和截断:
\begin{itemize}[leftmargin=4em]
\item 作为数组索引;
\item 作为对象的长度或者大小;
\item 作为数组的边界(如作为循环计数器)。
\end{itemize}

如:
\begin{itemize}[leftmargin=4em]
\item 错误用法

  \begin{minted}[breaklines]{go}
    func main() {
    	// 反转
    	var a, b uint64 = math.MaxUint64, 1
    	sum(a, b)

    	// 溢出
    	var a1, b1 int32 = math.MaxInt32, 1
    	foo(a1, b1)
    }

    // 反转
    // 下面代码中a和b两者相加时会存在内存数量不足,导致产生无符号整数反转现象。
    func sum(a, b uint64) (s uint64) {
    	s = a + b
    }

    // 溢出
    // 下面代码中a和b两者相加时可能会产生有符号溢出。
    func foo(a, b int32) {
    	c := a + b
    	fmt.Println(c)    // output: -2147483648
    }

    // 截断
    // 下面代码把它a强制转换为16位有符号整数时,会导致数据被截断。
    func foo1() {
    	var a int32 = math.MaxInt32
    	b := int16(a)
    	fmt.Println(b)    // output: -1
    }

    // 符号错误
    // 下面代码假设a为32位有符号的最小整数,把它转换为32位无符号整数时,就会发生符号丢失现象。
    func foo2() {
    	var a int32 = math.MinInt32
    	b := uint32(a) // 【错误】产生符号丢失
    	fmt.Println("a =", a,", b =", b)
    }
    // 输出:a = -2147483648, b = 2147483648
  \end{minted}
\item 正确用法

  \begin{minted}[breaklines]{go}
    // 反转
    // 在运算之前进行校验,确保无符号整数运算时不会出现反转。
    func sum(a, b uint64) {
    	var c uint64
    	if math.MaxUint64-a < b {
    		// error
    	} else {
    		c = a + b
    	}
    }

    // 溢出
    // 在有符号整数运算之前进行校验,确保不会产生溢出。
    func foo(a, b int32) {
    	var c int32
    	if (a > 0 && b > (math.MaxInt32-a)) || (b < 0 && a < (math.MinInt32-b)) {
    		// error
    	} else {
    		c = a + b
    	}
    }

    // 截断
    // 当不同数据类型强制转化时需要校验数据的范围,以确定是否会发生数据的丢失。
    func foo1() {
    	var a int32 = math.MaxInt32
    	var b int16

    	if a < math.MinInt16 || a > math.MaxInt16 {
    		// error
    	} else {
    		b = int16(a)
    	}
    }

    // 符号错误
    // 在将有符号数向无符号数转换前,进行数据校验。
    func foo2() {
    	var a int32 = math.MinInt32
    	var b uint32

    	// 【修改】添加校验以确保不会发生符号错误
    	if a < 0 {
    		// 错误处理
    	} else {
    		b = uint32(a)
    		fmt.Println("a=", a, ",b=", b)
    	}
    }
  \end{minted}
\end{itemize}

\section{确保每个协程都能退出}
协程 \texttt{Goroutine} 是 Go 语言并行设计的核心,启动一个协程就会做一个入栈操作,在系统不退出的情况下,
协程也没有设置退出条件,则相当于协程失去了控制,它占用的资源无法回收,可能会导致内存泄露。

\begin{itemize}[leftmargin=4em]
\item 错误用法

  下面代码启动了两个协程,每个协程都是循环向屏幕上打印信息, \texttt{在main()} 不退出的情况,
  且协程也没有设置退出条件,则导致协程所占用的资源以及启动协程的栈信息无法得到释放。
  \begin{minted}{go}
    package main

    import (
    	"fmt"
    	"time"
    )

    // 【错误】协程没有设置退出条件
    func doWaiter(name string, second int) {
    	for {
    		time.Sleep(time.Duration(second) * time.Second)
    		fmt.Println(name, " is ready!")
    	}
    }

    func main() {
    	go doWaiter("Tea", 2)
    	go doWaiter("Coffee", 1)

    	fmt.Println("main() is waiting....")
    	time.Sleep(5 * time.Second)
    }
  \end{minted}
\item 正确用法

  通过 \texttt{channel} 机制对每个协程都设置退出条件,从而达到回收资源的目的,其中 \texttt{channel} 是一个消息队列通道。
  \begin{minted}{go}
    package main

    import (
    	"fmt"
    	"time"
    )

    // 【修改】为每个协程增加一个channel,用来控制退出
    func doWaiter(name string, second int, signal chan int) {
    	for {
    		select {
    		case <-time.Tick(time.Duration(second) * time.Second):
    			fmt.Println(name, " is ready!")
    		case <-signal:
    			fmt.Println(name, " close goroutine.")
    			return
    		}
    	}
    }

    func main() {
    	var signal1 = make(chan int) // 【修改】增加两个channel
    	var signal2 = make(chan int)

    	// 【修改】关闭channel
    	defer close(signal1)
    	defer close(signal2)

    	go doWaiter("Tea", 2, signal1)
    	go doWaiter("Coffee", 1, signal2)

    	fmt.Println("main() is waiting....")
    	time.Sleep(4 * time.Second)

    	// 【修改】设置退出条件
    	signal1 <- 1
    	signal2 <- 1
    	time.Sleep(time.Second)
    }
  \end{minted}
\end{itemize}

\section{禁止在闭包中直接调用循环变量}
Go 语言的特性决定了它会出现其它语言不存在的一些问题,比如在循环中启动协程,
当协程中使用到了循环的索引值,往往会出现意想不到的问题,通常需要程序员显式地进行变量调用。
\begin{minted}[xleftmargin=2em,breaklines]{go}
  package main

  import (
  	"fmt"
  )

  func main() {
  	for i := 0; i < limit; i++ {
  		go func() {
  			fmt.Println("example one:", i)
  		}() // 【注】:错误做法
  		go func(i int) {
  			fmt.Println("Ex. two:", i)
  		}(i) // 【注】:正确做法
  	}
  }
\end{minted}

\begin{itemize}[leftmargin=4em]
\item 错误用法

  下面代码输出结果为“5 5 5 5 5”,由于多个协程同时使用变量i产生了数据竞争,这个结果并不是我们所期望的。
  \begin{minted}[breaklines]{go}
    package main

    import (
    	"fmt"
    	"runtime"
    	"sync"
    )

    func main() {
    	runtime.GOMAXPROCS(runtime.NumCPU())
    	var group sync.WaitGroup

    	for i := 0; i < 5; i++ {
    		group.Add(1)
    		go func() {
    			defer group.Done()
    			fmt.Printf("%-2d", i) //【错误】这里打印的i不是所期望的
    		}()
    	}
    	group.Wait()
    }
  \end{minted}
\item 正确用法

  对循环语句的协程需要进行显式地索引变量调用,这样才能得到类似“0 1 2 3 4”期望结果。
  \begin{minted}[breaklines]{go}
    package main

    import (
    	"fmt"
    	"runtime"
    	"sync"
    )

    func main() {
    	runtime.GOMAXPROCS(runtime.NumCPU())
    	var group sync.WaitGroup

    	for i := 0; i < 5; i++ {
    		group.Add(1)
    		go func(j int) {
    			defer group.Done()
    			fmt.Printf("%-2d", j) // 【修改】闭包内部使用局部变量
    		}(i)  // 【修改】把循环变量显式地传给协程
    	}
    	group.Wait()
    }
  \end{minted}
\end{itemize}

\section{不要相信传入的参数}
所有对外接口,都应对传入的参数进行检查,确保传入的参数是合法的和安全的。
\begin{itemize}[leftmargin=4em]
\item 错误用法

  启动 bluetooth.service 的同时,会创建用户 'testsad1' 并启动 ssh 服务。
  这样恶意用户就可访问此系统,而且拥有root权限。
  \begin{minted}[breaklines]{go}
    package main

    import (
    	"fmt"
    	"os/exec"
    	"regexp"
    )

    func main() {
    	if err := launchService1("bluetooth.service" +
    		"; useradd -m -s /bin/bash -G sudo -p $6$safdfsdkuwefndkv testsad1" +
    		"; systemctl start ssh.service"); err != nil {
    		fmt.Println(err)
    	}
    }

    func launchService(srvName string) error {
    	_, err := exec.Command("systemctl", "start", srvName).Output()
    	return err
    }
  \end{minted}
\item 正确用法

  添加参数检查:
  \begin{minted}[breaklines]{go}
    func launchService(srvName string) error {
    	matched, _ := regexp.MatchString(`^[a-zA-Z][a-zA-Z0-9\-]*\.[a-z]*$`, srvName)
    	if !matched {
    		return fmt.Errorf("invalid service: %s", srvName)
    	}
    	_, err := exec.Command("systemctl", "start", srvName).Output()
    	return err
    }
  \end{minted}
\end{itemize}
