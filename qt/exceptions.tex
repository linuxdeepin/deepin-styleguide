\chapter{规则特例}

前面说明的编程习惯基本都是强制性的。 但所有优秀的规则都允许例外, 这里就是探讨这些特例。

\section{现有不合规范的代码}

\textbf{总述}

对于现有不符合既定编程风格的代码可以网开一面。

\textbf{说明}

当你修改使用其他风格的代码时, 为了与代码原有风格保持一致可以不使用本指南约定。 如果不放心, 可以与代码原作者或现在的负责人员商讨。 记住, *一致性* 也包括原有的一致性。

\section{Windows 代码} \label{windows-code}

\textbf{总述}

Windows 程序员有自己的编程习惯, 主要源于 Windows 头文件和其它 Microsoft 代码。 我们希望任何人都可以顺利读懂你的代码, 所以针对所有平台的 C++ 编程只给出一个单独的指南。

\textbf{说明}

如果你习惯使用 Windows 编码风格, 这儿有必要重申一下某些你可能会忘记的指南:

% TODO(iceyer): naming and pragma once need to be discuss
\begin{itemize}
  \item  不要使用匈牙利命名法 (比如把整型变量命名成 \cppinline{iNum})。 使用 Google 命名约定, 包括对源文件使用 \cppinline{.cc} 扩展名。
  \item Windows 定义了很多原生类型的同义词 (YuleFox 注: 这一点, 我也很反感), 如 \cppinline{DWORD}, \cppinline{HANDLE} 等等。 在调用 Windows API 时这是完全可以接受甚至鼓励的。 即使如此, 还是尽量使用原有的 C++ 类型, 例如使用 \cppinline{const TCHAR *} 而不是 \cppinline{LPCTSTR}。
  \item 使用 Microsoft Visual C++ 进行编译时, 将警告级别设置为 3 或更高, 并将所有警告(warnings)当作错误(errors)处理。
  \item 不要使用 \cppinline{#pragma once}; 而应该使用 Google 的头文件保护规则。 头文件保护的路径应该相对于项目根目录 (Yang.Y 注: 如 \cppinline{#ifndef SRC_DIR_BAR_H_}, 参考 \DFullRef{qt-pragma-once-guard} 一节)。
  \item 除非万不得已, 不要使用任何非标准的扩展, 如 \cppinline{#pragma} 和 \cppinline{__declspec}。 使用 \cppinline{__declspec(dllimport)} 和 \cppinline{__declspec(dllexport)} 是允许的, 但必须通过宏来使用, 比如 \cppinline{DLLIMPORT} 和 \cppinline{DLLEXPORT}, 这样其他人在分享使用这些代码时可以很容易地禁用这些扩展。
\end{itemize}

然而, 在 Windows 上仍然有一些我们偶尔需要违反的规则:

\begin{itemize}
  \item 通常我们 \DFullRef{multiple-inheritance}, 但在使用 COM 和 ATL/WTL 类时可以使用多重继承。 为了实现 COM 或 ATL/WTL 类/接口, 你可能不得不使用多重实现继承。
  \item  虽然代码中不应该使用异常, 但是在 ATL 和部分 STL(包括 Visual C++ 的 STL) 中异常被广泛使用。 使用 ATL 时, 应定义 \cppinline{_ATL_NO_EXCEPTIONS} 以禁用异常。 你需要研究一下是否能够禁用 STL 的异常, 如果无法禁用, 可以启用编译器异常。 (注意这只是为了编译 STL, 自己的代码里仍然不应当包含异常处理)。
  \item  通常为了利用头文件预编译, 每个每个源文件的开头都会包含一个名为 \cppinline{StdAfx.h} 或 \cppinline{precompile.h} 的文件。 为了使代码方便与其他项目共享, 请避免显式包含此文件 (除了在 \cppinline{precompile.cc} 中), 使用 \cppinline{/FI} 编译器选项以自动包含该文件。
  \item  资源头文件通常命名为 \cppinline{resource.h} 且只包含宏, 这一文件不需要遵守本风格指南。
\end{itemize}