\chapter{命名约定}

在 \deepin 发行版本中,大约有 100 个左右的项目是 \deepin 来进行维护的。为了保障项目的统一性,这里对 \deepin 的总体命名进行一个阐述。

\section{通用名词} \label{general-naming-define}

通用名词是指由 \deepin 所持有的或主导的相关名词以及缩写。

通用名词在代码,文件名,文档中有不同的变体。每个名词都会分别说明。

\subsection{\deepin}

\textbf{总述}

\deepin 是指由 deepin.org 所有发行的发行版本。在指代发行版本时,应该永远使用小写的\deepin。

\begin{DWarn}
在 deepin 23 以后的版本中,deepin 将网站的主体迁移到 deepin.org 中,这将影响绝大部分的项目,特别是 DBus 接口部分。deepin 承诺在 2028 年之前保障旧接口还是可以使用的。
\end{DWarn}

\textbf{使用}

在文档,图片中,需要使用全小写的\deepin,即使是首字母,也应该使用小写。

\begin{cppcode}
  deepin is an opensoucre os.  // 正确

  Deepin is an opensoucre os.  // 错误,即使是段落首字母,也不应该大写
\end{cppcode}

在代码中,需要使用全小写的\deepin,除非代码风格规定了必须使用全大写,或首字母大小的情况。

\begin{cppcode}
  #define DEEPIN_MACRO XXXX     // 正确,以代码规范为准
  const int kDeepinNumber = 1;  // 正确,以代码规范为准

  // 版权信息中也需要使用小写的 deepin
  // * Copyright (c) 2021. deepin All rights reserved.
\end{cppcode}

在文件名中,需要使用全小写的\deepin。

\begin{cppcode}
  /usr/lib/deepin-daemon/dde-system-daemon  // 正确
  /usr/share/Deepin/msc/res                 // 错误,应该为 /usr/share/deepin/msc/res
\end{cppcode}

\textbf{例外}

当 \deepin 和其他名词组成专有名词时,可以使用大小写混合,例如:

\begin{inicode}
  # desktop 文件中,deepin-music 相关的应用
  [Desktop Entry]
  Name=Deepin Music
\end{inicode}

这里 \iniinline{Deepin Music}是一个专有名词,在任何情况下都不可以拆开使用。

\subsection{DDE}

\textbf{总述}

DDE 是 \iniinline{Deepin Desktop Environment} 的缩写。

\DBox{
  \iniinline{Deepin Desktop Environment}也是专有名词,不要拆开使用,也不要写成\iniinline{deepin Desktop Environment},\iniinline{deepin desktop environment}等形式。
}

\textbf{使用}

在文档,图片中,需要使用全大写的`DDE`。

\begin{cppcode}
  The DDE is comprised of the Desktop Environment, deepin Window Manager, Control Center, Launcher and Dock.    // 正确
  Use dde in other os.                  // 错误,文档中只有大写
  Login to Dde.                         // 错误,文档中不允许混合大小写
\end{cppcode}

在代码中,需要使用全大写的`DDE`,除非代码风格规定了必须使用全大写,或首字母大写的情况。

\begin{cppcode}
  #define DDE_MACRO XXXX     // 正确,以代码规范为准
  const int kDdeNumber = 1;  // 正确,以代码规范为准
\end{cppcode}

在文件名中,需要使用全小写的`dde`。

\begin{cppcode}
  /usr/lib/deepin-daemon/dde-system-daemon  // 正确
\end{cppcode}

\section{项目命名} \label{deepin-project-naming}

\textbf{总述}

\deepin 项目应该使用全小写的命名方式,单词使用\cppinline{-}进行连接。但是如果是应用型的项目,也可以使用倒置域名进行命名。

\textbf{使用}

\begin{cppcode}
  org.deepin.lianliankan // 倒置域名格式,应用必须使用该方式命名
  plymouth-theme-deepin  // 正确
  deepin-font-manager    // 正确

  Robot-Autotest         // 错误,不使用大写
\end{cppcode}

\section{文件命名} \label{deepin-file-naming}

\textbf{总述}

对于安装到系统中的文件,其命名方式和\DFullRef{deepin-project-naming}相同。同时需要满足 GNU/Linux 的通用命名风格。

\textbf{使用}

\begin{cppcode}
  /usr/bin/dde-dock  // 正确
  /usr/share/polkit-1/actions/com.deepin.pkexec.dde-file-manager.policy  // 正确

  /usr/share/DeepinAIAssistant/translations  // 错误,不使用大写
  /usr/lib/deepin-daemon/logViewerService    // 错误,log-view-service
  /usr/lib/deepin-daemon/backlight_helper    // 错误,backlight-helper
\end{cppcode}

\section{DBus 命名}

\textbf{总述}

DBus 命名是一个较为模糊的地带,我们根据官方的设计文档\href{https://dbus.freedesktop.org/doc/dbus-api-design.html}{D-Bus API Design Guidelines}来指导 DBus 的命名规则:

DBus 由服务名,路径,接口,方法(包括属性,信号等)四个部分组成。

对于服务名,路径,接口,其应该分解成域名,项目,组件三个部分。例如:

\begin{cppcode}
  org.deepin.Manual1.Search
  /org/deepin/Manual1/Search
  org.deepin.Manual1.Search
\end{cppcode}

其中 \cppinline{org.deepin} 是域名,Manual 是项目名称,1 是 API 版本号,Search 是组件名称。其中:

\begin{itemize}
  \item 域名:使用倒置域名方法,目前 deepin 使用的域名为 \cppinline{org.deepin},  \cppinline{org.desktopspec}。
  \item 项目名称:使用大小写混合方式。根据\href{https://dbus.freedesktop.org/doc/dbus-api-design.html}{D-Bus API Design Guidelines},这里需要带上版本号。
  \item 组件名称:如果一个项目提供多个服务,那么这里就需要有组件名称,组件名称使用大小写混合方式。
\end{itemize}

DBus 的方法(包括属性,信号等)永远使用大小写混合方式。

\textbf{使用}

注意,这里 org.freedesktop.portal 是域名,这也是 DBus 的接口风格中最让人迷惑的地方。

\begin{cppcode}
  org.freedesktop.portal.Desktop      // 正确,但是缺少 API 版本号
  /org/freedesktop/portal/Desktop     // 正确,但是缺少 API 版本号
  org.freedesktop.portal.Desktop      // 正确,但是缺少 API 版本号
\end{cppcode}

\begin{cppcode}
  org.deepin.DDE1.Accounts       // 正确
  /org/deepin/DDE1/Accounts      // 正确
  org.deepin.DDE1.Accounts       // 正确
  com.deepin.daemon.Accounts     // 错误,这是目前的命名方式,其中 daemon 意义不明确

  org.desktopspec.ConfigManager  // 正确,deepin 将通用的标准在 desktopsepc 组织中进行实现
\end{cppcode}


\textbf{备注}

按照这种命名方式,deepin V20 中 DDE 相关的绝大部分 DBus 接口需要重新设计。

\chapter{创建}

在 deepin 发行版本中,大约有 100 个左右的项目是 deepin 来进行维护的。为了避免项目冗余,这里对项目以及文件的创建进行一个阐述。

\section{项目创建}

\textbf{总述}

在新创建一个项目之前,应首先考虑 deepin 发行版已有的项目是否支持功能扩展,不推荐一味的进行项目新建。对于某个新特性必须创建项目时,其命名方式和\DFullRef{deepin-project-naming}相同。

\DBox{
	\iniinline{deepin-default-settings}、\iniinline{uos-config}同属系统通用配置文件管理项目,在后续的版本中,将废弃 uos-config 项目,使用 deepin-default-settings 进行统一管理。
}

\DBox{
	\iniinline{dde-wayland-config}、\iniinline{kwayland-data}同属 wayland 协议配置文件管理项目,推荐使用 dde-wayland-config 进行统一管理。
}

\textbf{备注}

按照项目创建规则,目前 deepin/DDE 相关的部分项目需要重新设计。

\section{文件创建}

\textbf{总述}

在新创建一个文件时,其命名方式和\DFullRef{deepin-file-naming}相同,文件路径遵循\href{https://refspecs.linuxfoundation.org/FHS_3.0/fhs-3.0.html}{Filesystem Hierarchy Standard}文件系统层次标准,推荐使用 Debian 规则中的 install 文件进行管理,减少使用代码直接进行文件创建。

\textbf{使用}

\begin{cppcode}
  /usr/lib/libudis86.so       // 正确
  /etc/os-version             // 正确
  /usr/bin/apt                // 正确
  /etc/qemu-ifup              // 错误,/etc 目录下存放配置文件,不能存放二进制可执行文件
  /usr/share/uos_resources    // 错误,该文件没有归属于任意项目
\end{cppcode}