\chapter{命名约定}

\section{通用命名规则} \label{c-general-naming-rules}

\textbf{总述}

函数命名,变量命名,文件命名要有描述性,少用缩写。

\textbf{说明}

尽可能使用描述性的命名,别心疼空间,毕竟相比之下让代码易于新读者理解更重要。不要用只有项目开发者能理解的缩写,也不要通过砍掉几个字母来缩写单词。

可接受的命名方式:

\begin{ccode}
int price_count_reader;     // 无缩写
int num_errors;             // "num" 是一个常见的写法
int num_dns_connections;    // 人人都知道 "DNS" 是什么
\end{ccode}

不好的命名方式:

\begin{ccode}
int n;                    // 毫无意义
int n_err;                // 含糊不清的缩写
int pc_reader;            // 只有贵团队知道是什么意思
int cstmr_id;             // 删减了若干字母
int l, I, O, l0, Il;      // 易与数字混淆名称
int YE5, N0;              // 易与单词混淆名称
int fxxk;                 // 不良名称
\end{ccode}

同时需要注意编译器限制,如果两个名称有相同的前缀,当相同前缀超过一定长度时是危险的,可能会导致编译器无法有效区分相关名称。C 标准指明,保证名称前 31 位不同可避免这种问题。

\begin{ccode}
extern int identifier_of_a_very_very_long_name_1;  // 过长字符,部分编译器无法区分
extern int identifier_of_a_very_very_long_name_2;
\end{ccode}

避免有特殊意义的名称,包括:

\begin{itemize}
  \item 标准库或编译环境中的宏名称
  \item 标准库中具有外部链接性的对象或函数名称
  \item 标准库中的类型名称
\end{itemize}

\begin{ccode}
int errno;  //变量 errno 与标准库中的 errno 名称相同,不便于区分是自定义的还是系统定义的
\end{ccode}

\section{文件命名}

\textbf{总述}

文件名要全部小写,可以包含下划线(\cinline{_}) 。如果存在无法使用(\cinline{_})的情况,可以考虑使用连字符(\cinline{-}),否则不允许例外。

\textbf{说明}

可接受的文件命名:

\begin{ccode}
my_useful_class.c
my_useful_class_test.c
\end{ccode}

不接受的文件命名:

\begin{ccode}
my-useful-class.c  // 不接受,除非无法使用_
myusefulclass.c    // 不接受,难以看懂
\end{ccode}

不要使用标准库中的文件名,例如\cinline{stdio.h}。通常应尽量让文件名更加明确。\cinline{http_server_logs.h} 就比 \cinline{logs.h} 要好。

\section{类型命名} \label{c-type-names}

\textbf{总述}

类型名称的每个单词首字母均大写,不包含下划线:\cinline{MyExcitingStruct}。

\textbf{说明}

结构体,类型定义 (\cinline{typedef}),以大写字母开始,每个单词首字母均大写,不包含下划线。例如:

\begin{ccode}
typedef struct SessionStatusInfo {
    const char *id;
    const char *remote_host;
} SessionStatusInfo;

typedef union UnitDependencyInfo {
    void *data;
    struct {
        UnitDependencyMask origin_mask:16;
        UnitDependencyMask destination_mask:16;
    } _packed_;
} UnitDependencyInfo;
\end{ccode}

\section{变量命名} \label{c-variable-names}

\textbf{总述}

变量命名一律使用蛇形命名法(snake\_case)。

\textbf{说明}

\begin{ccode}
char *table_name;
char *tableName;   // 不接受 - 用下划线。
char *tablename;   // 不接受,无分割符合。
\end{ccode}

\section{函数命名} \label{c-function-names}

\textbf{总述}

函数命名一律使用蛇形命名法(snake\_case)。

\textbf{说明}

\begin{ccode}
  int is_system_up();  // 一个名为 is_system_up 的功能性函数
\end{ccode}

\section{宏命名} \label{c-macro-names}

\textbf{总述}

尽可能少的使用宏,如果必须使用,则用全大写,并使用下划线分隔。\cinline{MY_MACRO_THAT_SCARES_SMALL_CHILDREN}.

\textbf{说明}

\begin{ccode}
#define ROUND(x) ...
#define PI_ROUNDED 3.0
\end{ccode}

\section{枚举命名} \label{c-enum-names}

\textbf{总述}
枚举本身命名参考宏命名。

\textbf{说明}

对于标志类型(Flags)如果可能的话,使用枚举。通过 \cinline{1 <<} 表达式指示位值,并垂直对齐它们。为其定义枚举和类型。

\begin{ccode}
typedef enum FoobarFlags {
    FOOBAR_QUUX  = 1 << 0,
    FOOBAR_WALDO = 1 << 1,
    FOOBAR_XOXO  = 1 << 2,
} FoobarFlags;
\end{ccode}

对于非标志类型,定义一个 \cinline{_MAX} 枚举,用于表示已定义枚举值中最大的值,再加一。由于这不是常规枚举值,请使用 \cinline{_} 作为前缀。同时,定义一个特殊的“无效”枚举值,并将其设置为 \cinline{-EINVAL}。这样枚举类型可以安全地用于传播转换错误。

\begin{ccode}
typedef enum FoobarMode {
    FOOBAR_AAA,
    FOOBAR_BBB,
    FOOBAR_CCC,
    _FOOBAR_MAX,
    _FOOBAR_INVALID = -EINVAL,
} FoobarMode;
\end{ccode}

\section{完整实例}

\begin{ccode}
#include <gtk/gtk.h>

static void on_activate (GtkApplication *app) {
    // Create a new window
    GtkWidget *t_window = gtk_application_window_new (g_app);
    // Create a new button
    GtkWidget *t_button = gtk_button_new_with_label ("Hello, World!");
    // When the button is clicked, close the window passed as an argument
    g_signal_connect_swapped (g_button, "clicked", G_CALLBACK (gtk_window_close), window);
    gtk_window_set_child (GTK_WINDOW (t_window), g_button);
    gtk_window_present (GTK_WINDOW (t_window));
}

int main (int argc, char *argv[]) {
    // Create a new application
    GtkApplication *t_app = gtk_application_new ("com.example.GtkApplication", G_APPLICATION_FLAGS_NONE);
    g_signal_connect (t_app, "activate", G_CALLBACK (on_activate), NULL);
    return g_application_run (G_APPLICATION (t_app), argc, argv);
}
\end{ccode}