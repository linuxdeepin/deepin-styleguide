\begin{DNote}
    本部分内容参考\href{https://raw.githubusercontent.com/systemd/systemd/main/docs/CODING_STYLE.md}{Systemd 代码风格} 和 \href{https://github.com/zh-google-styleguide/zh-google-styleguide}{Google 开源项目风格指南——中文版}》。
\end{DNote}

\chapter{语义}

对外部接口,使用 ISO C89(ANSI X3.159-1989)标准,避免外部使用者必须使用 ISO C11(ISO/IEC 9899:2011)。

对于内部库,可以 GNU 扩展的 ISO C11(也称为 "gnu11")或 ISO C11 标准。

编译器建议最低支持到 GCC 8.3.0。

\section{头文件}

\subsection{头文件必须包含守卫}

\textbf{总述}

以 .h 或 为扩展名的头文件应包含头文件守卫。例中 foo.h 是“Library”模块中的头文件,宏 \cinline{LIBRARY_FOO_H} 即可作为它的守卫,保证头文件被重复引入也不会出现问题,守卫名称不可有重复,建议守卫名称遵循“模块名\_文件名”的形式。

\cinline{#pragma once} 指令也可作为头文件守卫,但并不是 C/C++ 的标准方式,只是多数编译器均有支持。这种方式由编译器维护一个列表,引入头文件时,如果发现文件中有 \cinline{#pragma once}  指令就将文件路径加入列表,当这个文件再次被 include 时便不会加载,而宏守卫的方式仍然要对文件进行预编译,所以 \cinline{#pragma once}  方式在编译效率上会更高一些。

\begin{DWarn}
注意在\deepin \ Qt 风格中,不允许使用\cinline{#pragma once} 。同时为兼容性考虑,在公开的头文件中,建议也不要使用\cinline{#pragma once} 。
\end{DWarn}

\textbf{说明}

\begin{ccode}
// Header file foo.h
#ifndef LIBRARY_FOO_H
#define LIBRARY_FOO_H
....
#endif

// Header file foo.h
#pragma once
\end{ccode}

\subsection{头文件不能有静态声明}

\textbf{总述}

头文件中由 static 关键字声明的对象、数组或函数,会在每个包含该头文件的翻译单元或模块中生成副本造成数据冗余,如果将静态数据误用作全局数据也会造成逻辑错误。

\textbf{说明}

\begin{ccode}
// In a header file
static int i = 0;   // 不合规,头文件中错误使用

static int foo() {  // 不合规,头文件中错误使用
    return i;
}
\end{ccode}

\subsection{头文件不能有函数实现}

\textbf{总述}

头文件是项目文档的重要组成部分,有必要保持头文件简洁清晰,头文件的主要内容应是类型或接口的声明。除非函数很简短,否则也不建议在头文件中内联实现,大段的函数实现会影响头文件的可读性。

定义在头文件中的函数发生变化时,所有相关模块均需重新编译,会增加构建和维护成本,在使用动态链接库时这个问题尤为突出,如果库的导入者没有及时编译,可能会造成严重后果。在头文件中定义的函数是模块二进制接口的一部分,应合理规划以降低维护成本。

\subsection{头文件合理声明}

\textbf{总述}

如果声明的位置不合理会降低代码的可维护性,甚至会导致标准未定义的行为。应遵循如下原则:

\begin{itemize}
    \item 外部链接的对象或函数应在头文件中声明,并避免重复声明
    \item 内部链接的对象或函数应在源文件中声明,不应在头文件中声明
    \item 避免在头文件外手工书写外部声明
    \item 避免在局部作用域内声明函数或全局对象
\end{itemize}

\begin{ccode}

int fun()
{
    extern int g;       // 不合规
    extern int foo();   // 不合规
    static int bar();   // 不合规,未定义行为
}
\end{ccode}