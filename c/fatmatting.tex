\chapter{格式化}

在一个项目中,必须保证代码风格一致性。除非这些代码是直接复制引用第三方的,这些第三方代码建议放在单独目录。同时,务必使用\cinline{.editorconfig}来保证编辑器能够正确处理代码风格。

\section{行长度}

\textbf{总述}

每一行代码字符数不超过 110。不同人对使用 80/100/110/120 持有不同的态度,请在 \deepin 的\cinline{C}项目中保持这个值。

\begin{DNote}
头文件保护可以无视该原则。
\end{DNote}

\section{字符编码}

\textbf{总述}

使用时必须使用 UTF-8 编码。除非在输出给用户直接观看的文本中,否则不要使用非 ASCII 字符;即使在输出文本,也应该慎重考虑对 tty 的支持,避免输出乱码。

\section{缩进}

\textbf{总述}

只使用空格,每次缩进 4 个空格。

\section{函数声明与定义}

\textbf{总述}

返回类型和函数名在同一行,参数也尽量放在同一行,如果放不下就对形参分行。需要注意如下细节:

\begin{itemize}
  \item 左大括号总在最后一个参数同一行的末尾处,不另起新行。
  \item 右大括号总是单独位于函数最后一行,或者与左大括号同一行。
  \item 左圆括号总是和函数名在同一行,函数名和左圆括号间永远没有空格。
  \item 右圆括号和左大括号间总是有一个空格
  \item 所有形参应尽可能对齐。
  \item 换行后的参数缺省缩进为 8 个空格,即 2 倍标准缩减。
\end{itemize}

\textbf{说明}

\begin{ccode}
void some_function(int foo) {
    int a;
}

void some_function(
        int foo,
        bool bar,
        char baz) {
    int a, b, c;
}
\end{ccode}

\section{条件表达式}

\textbf{总述}

单行 if 块应该被包含在 \{\} 中。\cinline{else}块通常应该从与闭合\}相同的行开始。不在圆括号内使用空格。

\textbf{说明}

\begin{ccode}
if (foobar) {
    find_and_waldo();
} else {
    dont_find_waldo();
}
\end{ccode}

\section{循环和开关选择语句}

\textbf{总述}

\cinline{switch}语句可以使用大括号分段,以表明 cases 之间不是连在一起的。在单语句循环里,括号可用可不用。空循环体应使用 \{\} 或 \cinline{continue},而不是一个简单的分号。

\textbf{说明}

\begin{ccode}
switch (var) {
case 0: {     // 不缩进
    ...       // 4 空格缩进
    break;
}
case 1: {
    ...
    break;
}
default: {
    assert(false);
}
}

for (int i = 0; i < num; ++i) {
  printf("I take it back\n");
}

while (condition) {
  // 反复循环直到条件失效。
}

for (int i = 0; i < num; ++i) {}  // 可 - 空循环体。

while (condition) continue;  // 可 - contunue 表明没有逻辑。
\end{ccode}

\section{指针}

\textbf{总述}

句点或箭头前后不要有空格。指针/地址操作符 (\*, \&) 之后不能有空格。

\textbf{说明}

\begin{ccode}
x = *p;
p = &x;
x = r.y;
x = r->y;

// 好,空格前置。
char *c;

// 可接受,空格后置。
char* c;

int x, *y;  // 不允许 - 在多重声明中不能使用 & 或 *
char * c;   // 差 - * 两边都有空格
\end{ccode}

\section{布尔表达式}

\textbf{总述}

如果一个布尔表达式超过标准行宽,断行方式要统一一下。

\textbf{说明}

可以考虑额外插入圆括号,合理使用的话对增强可读性是很有帮助的。

\begin{ccode}
    if (this_one_thing > this_other_thing
        && a_third_thing == a_fourth_thing
        && yet_another
        && last_one) {
    }
\end{ccode}

\section{函数返回值}

\textbf{总述}

不要在 \cinline{return} 表达式里加上非必须的圆括号。

\textbf{说明}

只有在写 \cinline{x = expr} 要加上括号的时候才在 \cinline{return expr;} 里使用括号。

\begin{ccode}
return result;  // 返回值很简单,没有圆括号。
// 可以用圆括号把复杂表达式圈起来,改善可读性。
return (some_long_condition &&
        another_condition);

return (value);                // 错误,毕竟您从来不会写 var = (value);
return(result);                // 错误,return 可不是函数!
\end{ccode}

\section{预处理指令}

\textbf{总述}

预处理指令不要缩进,从行首开始。

\textbf{说明}

即使预处理指令位于缩进代码块中,指令也应从行首开始。

\begin{ccode}
    if (lopsided_score) {
#if DISASTER_PENDING      // 正确 - 从行首开始
        drop_everything();
#if NOTIFY
        notify_client();
#endif
#endif
        back_to_normal();
    }
\end{ccode}

\section{留白}

\textbf{总述}

水平留白的使用根据在代码中的位置决定。永远不要在行尾添加没意义的留白。添加冗余的留白会给其他人编辑时造成额外负担。因此,行尾不要留空格。如果确定一行代码已经修改完毕,将多余的空格去掉; 或者在专门清理空格时去掉(尤其是在没有其他人在处理这件事的时候)。

垂直留白越少越好。这不仅仅是规则而是原则问题了:不在万不得已,不要使用空行。尤其是:两个函数定义之间的空行不要超过 2 行,函数体首尾不要留空行,函数体中也不要随意添加空行。基本原则是:同一屏可以显示的代码越多,越容易理解程序的控制流。当然,过于密集的代码块和过于疏松的代码块同样难看,这取决于你的判断。但通常是垂直留白越少越好。

\textbf{说明}

即使预处理指令位于缩进代码块中,指令也应从行首开始。

\begin{ccode}
void f(bool b) {  // 左大括号前总是有空格
int i = 0;  // 分号前不加空格

// 对于单行函数的实现,在大括号内加上空格
// 然后是函数实现
Foo(int b) : Bar(), baz_(b) {}  // 大括号里面是空的话,不加空格
void Reset() { baz_ = 0; }      // 用空格把大括号与实现分开

if (b) {            // if 条件语句和循环语句关键字后均有空格。
} else {            // else 前后有空格。
}
while (test) {}     // 圆括号内部不紧邻空格。

switch (i) {
case 1:             // switch case 的冒号前无空格。
case 2: break;      // 如果冒号有代码,加个空格。
\end{ccode}
