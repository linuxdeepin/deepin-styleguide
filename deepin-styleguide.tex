\documentclass[UTF8,a4paper,oneside]{ctexbook}

\usepackage{indentfirst}
\setlength{\parindent}{2em}

\usepackage[margin=1in]{geometry}

\usepackage{xcolor}
\definecolor{deepin_blue}{HTML}{0088FD}
\definecolor{deepin_black}{HTML}{000000}
\definecolor{deepin_gray}{HTML}{F0F0F0}
\definecolor{deepin_orange}{RGB}{244, 151, 0}

\usepackage{ifxetex,ifluatex}

% 使用 getenv 获取输出目录,避免 minted 插件在使用分离目录编译时找不到文件
\ifxetex
  \usepackage{catchfile}
  \newcommand\getenv[2][]{%
    \immediate\write18{kpsewhich --var-value #2 > \jobname.tmp}%
    \CatchFileDef{\temp}{\jobname.tmp}{\endlinechar=-1}%
    \if\relax\detokenize{#1}\relax\temp\else\let#1\temp\fi}
\else
  \ifluatex
    \newcommand\getenv[2][]{%
      \edef\temp{\directlua{tex.sprint(
        kpse.var_value("\luatexluaescapestring{#2}") or "" ) }}%
      \if\relax\detokenize{#1}\relax\temp\else\let#1\temp\fi}
  \else
    \usepackage{catchfile}
    \newcommand{\getenv}[2][]{%
      \CatchFileEdef{\temp}{"|kpsewhich --var-value #2"}{\endlinechar=-1}%
      \if\relax\detokenize{#1}\relax\temp\else\let#1\temp\fi}
  \fi
\fi

\getenv[\TEXMFOUTPUT]{TEXMFOUTPUT}
% 配置代码高亮
\ifx\TEXMFOUTPUT\empty
  \usepackage{minted}
\else
  \usepackage[outputdir=\TEXMFOUTPUT,cache=false]{minted}
\fi

\newminted{cpp}{frame=lines,framerule=1pt,breaklines,breakanywhere}
\newmintinline{cpp}{breaklines,breakanywhere}

\newminted{c}{frame=lines,framerule=1pt,breaklines,breakanywhere}
\newmintinline{c}{breaklines,breakanywhere}

\newminted{ini}{frame=lines,framerule=1pt,breaklines,breakanywhere}
\newmintinline{ini}{breaklines,breakanywhere}

\usepackage{nameref}
\usepackage[colorlinks=true]{hyperref}

\newcommand*{\DFullRef}[1]{\hyperref[{#1}]{\ref*{#1} \nameref*{#1}}}

\usepackage{titlesec}
\titleformat{\chapter}[hang]
{\huge\bfseries}
{\thechapter\hspace{20pt}\textcolor{deepin_black}{|}\hspace{20pt}}{0pt}
{\huge\bfseries}

\usepackage{enumitem}

\makeatletter
\renewcommand{\section}{
  \@startsection{section}{1}{0mm}
  {0.5\baselineskip}{0.5\baselineskip}{\bf\leftline}
 }
\makeatother

\makeatletter
\newcommand*{\DBox}[1]{
\@makeother\#
\vspace{0.5\baselineskip}
\noindent\fbox{\parbox{\linewidth}{#1}}
}
\makeatother

\usepackage{framed}

\newenvironment{DNote}{%
  \def\FrameCommand{%
    {\color{deepin_blue}\vrule width 2pt}%
    \colorbox{deepin_gray}%
  }%
  \MakeFramed{\advance\hsize-\width\FrameRestore}%
}
{%
  \vspace{2pt}\endMakeFramed%
}

\newenvironment{DWarn}{%
  \def\FrameCommand{%
    {\color{deepin_orange}\vrule width 2pt}%
    \colorbox{deepin_gray}%
  }%
  \MakeFramed{\advance\hsize-\width\FrameRestore}%
}
{%
  \vspace{2pt}\endMakeFramed%
}

\usepackage{verbatim}

\newcommand*{\deepin}[0]{\textcolor{deepin_blue}{deepin}}

% 设置段落间距
\setlength{\parskip}{0.5em}
\makeatletter
\@addtoreset{chapter}{part}
\makeatother

\title{\deepin 开源项目风格指南}
\date{\today}
\author{\deepin}

\begin{document}

\maketitle

\textbf{声明}

本项目是在《\href{https://github.com/zh-google-styleguide/zh-google-styleguide}{Google 开源项目风格指南——中文版}》基础上修改而来。

本项目使用 \LaTeX 构建。

\begin{DNote}

\textbf{Google 开源项目风格指南——中文版声明}

项目原始地址是:\href{https://github.com/zh-google-styleguide/zh-google-styleguide}{Google 开源项目风格指南}

如果你关注的是 Google 官方英文版,请移步 \href{https://github.com/google/styleguide}{Google Style Guide}

每个较大的开源项目都有自己的风格指南:关于如何为该项目编写代码的一系列约定(有时候会比较武断)。当所有代码均保持一致的风格,在理解大型代码库时更为轻松。

“风格”的含义涵盖范围广,从“变量使用驼峰格式(camelCase)”到“决不使用全局变量”再到“决不使用异常”,等等诸如此类。

英文版项目维护的是在 Google 使用的编程风格指南。如果你正在修改的项目源自 Google,你可能会被引导至英文版项目页面,以了解项目所使用的风格。

中文版项目采用 reStructuredText 纯文本标记语法,并使用 Sphinx 生成 HTML / CHM / PDF 等文档格式。

英文版项目还包含 \href{https://github.com/google/styleguide/tree/gh-pages/cpplint}{cpplint} ——一个用来帮助适应风格准则的工具,以及 \href{https://raw.githubusercontent.com/google/styleguide/gh-pages/google-c-style.el}{google-c-style.el},Google 风格的 Emacs 配置文件。

\end{DNote}

\tableofcontents
\newpage

\part{Qt代码风格}

\input{qt/flyleaf.tex}
\chapter{头文件}

通常每一个 \cppinline{.cpp} 文件都有一个对应的 \cppinline{.h} 文件. 也有一些常见例外, 如单元测试代码和只包含 \cppinline{main()} 函数的\cppinline{.cpp} 文件。

正确使用头文件可令代码在可读性、文件大小和性能上大为改观。

下面的规则将引导你规避使用头文件时的各种陷阱。

\section{Self-contained 头文件} \label{self-contained-headers}

\DBox {
	头文件应该能够自给自足(self-contained,也就是可以作为第一个头文件被引入),以 \cppinline{.h} 结尾。至于用来插入文本的文件,说到底它们并不是头文件,所以应以 \cppinline{.inc} 结尾。不允许分离出 \cppinline{-inl.h} 头文件的做法。
}

所有头文件要能够自给自足。换言之,用户和重构工具不需要为特别场合而包含额外的头文件。详言之,一个头文件要有\DFullRef{pragma-once-guard},统统包含它所需要的其它头文件,也不要求定义任何特别 symbols。

不过有一个例外,即一个文件并不是 self-contained 的,而是作为文本插入到代码某处。或者,文件内容实际上是其它头文件的特定平台(platform-specific)扩展部分。这些文件就要用\cppinline{.inc} 文件扩展名。

如果 \cppinline{.h} 文件声明了一个模板或内联函数,同时也在该文件加以定义。凡是有用到这些的 \cppinline{.cpp} 文件,就得统统包含该头文件,否则程序可能会在构建中链接失败。不要把这些定义放到分离的 \cppinline{-inl.h} 文件里(译者注:过去该规范曾提倡把定义放到 -inl.h 里过)。

有个例外:如果某函数模板为所有相关模板参数显式实例化,或本身就是某类的一个私有成员,那么它就只能定义在实例化该模板的 \cppinline{.cpp} 文件里。

\section{#define 保护} \label{pragma-once-guard}

\DBox{
  所有头文件都应该使用 \cppinline{#define} 来防止头文件被多重包含,命名格式当是:\cppinline{<PROJECT>_<PATH>_<FILE>_H_}。
}

为保证唯一性,头文件的命名应该基于所在项目源代码树的全路径。例如,项目 foo 中的头文件 foo/src/bar/baz.h 可按如下方式保护:

\begin{cppcode}
  #ifndef FOO_BAR_BAZ_H_
  #define FOO_BAR_BAZ_H_
  ...
  #endif // FOO_BAR_BAZ_H_
\end{cppcode}

\section{前置声明} \label{forward-declarations}

\DBox {
	尽可能地避免使用前置声明。使用 \cppinline{#include} 包含需要的头文件即可。
}

\textbf{定义:}

所谓「前置声明」(forward declaration)是类、函数和模板的纯粹声明,没伴随着其定义.

\textbf{优点:}

\begin{itemize}
	\item 前置声明能够节省编译时间,多余的 \cppinline{#include} 会迫使编译器展开更多的文件,处理更多的输入。
	\item 前置声明能够节省不必要的重新编译的时间。 \cppinline{#include} 使代码因为头文件中无关的改动而被重新编译多次。
\end{itemize}

\textbf{缺点:}

\begin{itemize}
	\item 前置声明隐藏了依赖关系,头文件改动时,用户的代码会跳过必要的重新编译过程。
	\item 前置声明可能会被库的后续更改所破坏。前置声明函数或模板有时会妨碍头文件开发者变动其 API。例如扩大形参类型,加个自带默认参数的模板形参等等。
	\item 前置声明来自命名空间 \cppinline{std::} 的 symbol 时,其行为未定义。
	\item 很难判断什么时候该用前置声明,什么时候该用 \cppinline{#include} 。极端情况下,用前置声明代替 \cppinline{#include} 甚至都会暗暗地改变代码的含义:

\begin{cppcode}
// b.h:
struct B {};
struct D : B {};

// good_user.cpp:
#include "b.h"
void f(B*);
void f(void*);
void test(D* x) { f(x); }  // calls f(B*)
\end{cppcode}

	      如果 \cppinline{#include} 被 \cppinline{B} 和 \cppinline{D} 的前置声明替代, \cppinline{test()} 就会调用 \cppinline{f(void*)} 。	\item 前置声明了不少来自头文件的 symbol 时,就会比单单一行的 \cppinline{include} 冗长。
	\item 仅仅为了能前置声明而重构代码(比如用指针成员代替对象成员)会使代码变得更慢更复杂.
\end{itemize}

\textbf{结论:}

\begin{itemize}
	\item 尽量避免前置声明那些定义在其他项目中的实体。
	\item 函数:总是使用 \cppinline{#include} 。
	\item 类模板:优先使用 \cppinline{#include} 。
\end{itemize}

至于什么时候包含头文件,参见 \DFullRef{name-and-order-of-includes} 。


\section{内联函数} \label{inline-functions}

\DBox {
	只有当函数只有 10 行甚至更少时才将其定义为内联函数。
}

\textbf{定义:}

当函数被声明为内联函数之后, 编译器会将其内联展开, 而不是按通常的函数调用机制进行调用。

\textbf{优点:}

只要内联的函数体较小, 内联该函数可以令目标代码更加高效. 对于存取函数以及其它函数体比较短, 性能关键的函数, 鼓励使用内联.

\textbf{缺点:}

滥用内联将导致程序变得更慢. 内联可能使目标代码量或增或减, 这取决于内联函数的大小. 内联非常短小的存取函数通常会减少代码大小,但内联一个相当大的函数将戏剧性的增加代码大小. 现代处理器由于更好的利用了指令缓存, 小巧的代码往往执行更快。

\textbf{结论:}

一个较为合理的经验准则是, 不要内联超过 10 行的函数. 谨慎对待析构函数, 析构函数往往比其表面看起来要更长,因为有隐含的成员和基类析构函数被调用!

另一个实用的经验准则: 内联那些包含循环或 \cppinline{switch} 语句的函数常常是得不偿失 (除非在大多数情况下, 这些循环或 \cppinline{switch} 语句从不被执行).

有些函数即使声明为内联的也不一定会被编译器内联, 这点很重要; 比如虚函数和递归函数就不会被正常内联. 通常,递归函数不应该声明成内联函数.(YuleFox 注: 递归调用堆栈的展开并不像循环那么简单, 比如递归层数在编译时可能是未知的,大多数编译器都不支持内联递归函数). 虚函数内联的主要原因则是想把它的函数体放在类定义内, 为了图个方便, 抑或是当作文档描述其行为,比如精短的存取函数.

\section{ include 的路径及顺序} \label{name-and-order-of-includes}

\DBox{
	使用标准的头文件包含顺序可增强可读性, 避免隐藏依赖: 相关头文件, C 库, C++ 库, 其他库的 `.h`, 本项目内的 `.h`.
}

项目内头文件应按照项目源代码目录树结构排列, 避免使用 UNIX 特殊的快捷目录 \cppinline{.} (当前目录) 或 \cppinline{..} (上级目录)。

例如, \cppinline{google-awesome-project/src/base/logging.h} 应该按如下方式包含:

\begin{cppcode}
	#include "base/logging.h"
\end{cppcode}

又如, \cppinline{dir/foo.cpp} 或 \cppinline{dir/foo_test.cpp} 的主要作用是实现或测试 \cppinline{dir2/foo2.h}的功能, \cppinline{foo.cpp} 中包含头文件的次序如下:

\begin{enumerate}
	\item \cppinline{dir2/foo2.h} (优先位置, 详情如下)
	\item C 系统文件
	\item C++ 系统文件
	\item 其他库的 \cppinline{.h} 文件
	\item 本项目内 \cppinline{.h} 文件
\end{enumerate}

这种优先的顺序排序保证当 \cppinline{dir2/foo2.h} 遗漏某些必要的库时, \cppinline{dir/foo.cpp} 或 \cppinline{dir/foo_test.cpp} 的构建会立刻中止。因此这一条规则保证维护这些文件的人们首先看到构建中止的消息而不是维护其他包的人们。

\cppinline{dir/foo.cpp} 和 \cppinline{dir2/foo2.h} 通常位于同一目录下,如\cppinline{base/basictypes_unittest.cpp} 和 \cppinline[breaklines]{base/basictypes.h}, 但也可以放在不同目录下.

按字母顺序分别对每种类型的头文件进行二次排序是不错的主意。注意较老的代码可不符合这条规则,要在方便的时候改正它们。

您所依赖的符号 (symbols) 被哪些头文件所定义,您就应该包含(include)哪些头文件,`前置声明`(forward declarations) 情况除外。比如您要用到 \cppinline{bar.h} 中的某个符号, 哪怕您所包含的 \cppinline{foo.h} 已经包含了\cppinline{bar.h}, 也照样得包含 \cppinline{bar.h}, 除非 \cppinline{foo.h} 有明确说明它会自动向您提供 \cppinline{bar.h} 中的symbol. 不过,凡是 cc 文件所对应的「相关头文件」已经包含的,就不用再重复包含进其 cc 文件里面了,就像 \cppinline{foo.cpp}只包含 \cppinline{foo.h} 就够了,不用再管后者所包含的其它内容。

举例来说, \cppinline{google-awesome-project/src/foo/internal/fooserver.cpp} 的包含次序如下:

\begin{cppcode}
	#include "foo/public/fooserver.h" // 优先位置

	#include <sys/types.h>
	#include <unistd.h>

	#include <hash_map>
	#include <vector>

	#include "base/basictypes.h"
	#include "base/commandlineflags.h"
	#include "foo/public/bar.h"
\end{cppcode}

\textbf{例外:}

有时,平台特定(system-specific)代码需要条件编译(conditional includes),这些代码可以放到其它 includes 之后。当然,您的平台特定代码也要够简练且独立,比如:

\begin{cppcode}
	#include "foo/public/fooserver.h"

	#include "base/port.h"  // For LANG_CXX11.

	#ifdef LANG_CXX11
	#include <initializer_list>
	#endif  // LANG_CXX11
\end{cppcode}


\section{注解}

\subsection{译者 (YuleFox) 笔记}

\begin{itemize}
	\item  避免多重包含是学编程时最基本的要求;
	\item  前置声明是为了降低编译依赖,防止修改一个头文件引发多米诺效应;
	\item  内联函数的合理使用可提高代码执行效率;
	\item  \cppinline{-inl.h} 可提高代码可读性 (一般用不到吧:D);
	\item  标准化函数参数顺序可以提高可读性和易维护性 (对函数参数的堆栈空间有轻微影响, 我以前大多是相同类型放在一起);
	\item  包含文件的名称使用 \cppinline{.} 和 \cppinline{..} 虽然方便却易混乱, 使用比较完整的项目路径看上去很清晰, 很条理,包含文件的次序除了美观之外, 最重要的是可以减少隐藏依赖, 使每个头文件在 "最需要编译" (对应源文件处 :D) 的地方编译,有人提出库文件放在最后, 这样出错先是项目内的文件, 头文件都放在对应源文件的最前面, 这一点足以保证内部错误的及时发现了.
\end{itemize}

\subsection{译者(acgtyrant)笔记}

\begin{itemize}
	\item  原来还真有项目用 \cppinline{#include} 来插入文本,且其文件扩展名 \cppinline{.inc} 看上去也很科学。
	\item  Google 已经不再提倡 \cppinline{-inl.h} 用法。
	\item  注意,前置声明的类是不完全类型(incomplete type),我们只能定义指向该类型的指针或引用,或者声明(但不能定义)以不完全类型作为参数或者返回类型的函数。毕竟编译器不知道不完全类型的定义,我们不能创建其类的任何对象,也不能声明成类内部的数据成员。
	\item  类内部的函数一般会自动内联。所以某函数一旦不需要内联,其定义就不要再放在头文件里,而是放到对应的 \cppinline{.cpp} 文件里。这样可以保持头文件的类相当精炼,也很好地贯彻了声明与定义分离的原则。
	\item  在 \cppinline{#include} 中插入空行以分割相关头文件, C 库, C++ 库, 其他库的 \cppinline{.h} 和本项目内的 \cppinline{.h} 是个好习惯。
\end{itemize}


\chapter{作用域}

\section{命名空间} \label{qt-namespace}

\DBox{
鼓励在 \cppinline{.cpp} 文件内使用匿名命名空间或 \cppinline{static} 声明。使用具名的命名空间时,其名称可基于项目名或相对路径。
禁止使用 using 指示(using-directive)。禁止使用内联命名空间(inline namespace)。
}

\textbf{定义:}

命名空间将全局作用域细分为独立的,具名的作用域,可有效防止全局作用域的命名冲突。

\textbf{优点:}

虽然类已经提供了(可嵌套的)命名轴线 (YuleFox 注:将命名分割在不同类的作用域内), 命名空间在这基础上又封装了一层。

举例来说,两个不同项目的全局作用域都有一个类 \cppinline{Foo}, 这样在编译或运行时造成冲突。如果每个项目将代码置于不同命名空间中,
\cppinline{project1::Foo} 和 \cppinline{project2::Foo} 作为不同符号自然不会冲突。

内联命名空间会自动把内部的标识符放到外层作用域,比如:

% \begin{noindent}
\begin{cppcode}
namespace X {
inline namespace Y {
  void foo();
}  // namespace Y
}  // namespace X
\end{cppcode}
%\end{noindent}

\cppinline{X::Y::foo()} 与 \cppinline{X::foo()} 彼此可代替。内联命名空间主要用来保持跨版本的 ABI 兼容性。

\textbf{缺点:}

命名空间具有迷惑性,因为它们使得区分两个相同命名所指代的定义更加困难。

内联命名空间很容易令人迷惑,毕竟其内部的成员不再受其声明所在命名空间的限制。内联命名空间只在大型版本控制里有用。

有时候不得不多次引用某个定义在许多嵌套命名空间里的实体,使用完整的命名空间会导致代码的冗长。

在头文件中使用匿名空间导致违背 C++ 的唯一定义原则 (One Definition Rule (ODR))。

\textbf{结论:}

根据下文将要提到的策略合理使用命名空间。

\begin{itemize}
  \item 遵守 \DFullRef{qt-namespace-names} 中的规则。
  \item 像之前的几个例子中一样,在命名空间的最后注释出命名空间的名字。
  \item 用命名空间把文件包含,`gflags <https://gflags.github.io/gflags/>` 的声明/定义,以及类的前置声明以外的整个源文件封装起来,以区别于其它命名空间:

% \begin{noindent}
\begin{cppcode}
  // .h 文件
namespace mynamespace {

// 所有声明都置于命名空间中
// 注意不要使用缩进
class MyClass {
public:
...
void Foo();
};

} // namespace mynamespace
\end{cppcode}
%\end{noindent}

%\begin{noindent}
\begin{cppcode}
// .cpp 文件
namespace mynamespace {

// 函数定义都置于命名空间中
void MyClass::Foo() {
...
}

} // namespace mynamespace
\end{cppcode}
%\end{noindent}

更复杂的 \cppinline{.cpp} 文件包含更多,更复杂的细节,比如 gflags 或 using 声明。

%\begin{noindent}
\begin{cppcode}
#include "a.h"

DEFINE_FLAG(bool, someflag, false, "dummy flag");

namespace a {

...code for a...// 左对齐

} // namespace a
\end{cppcode}
% \end{noindent}

  \item 不要在命名空间 \cppinline{std} 内声明任何东西,包括标准库的类前置声明。在 \cppinline{std} 命名空间声明实体是未定义的行为,会导致如不可移植。声明标准库下的实体,需要包含对应的头文件。

  \item 不应该使用 \textit{using 指示} 引入整个命名空间的标识符号。

%\begin{noindent}
\begin{cppcode}
// 禁止 —— 污染命名空间
using namespace foo;
\end{cppcode}
% \end{noindent}

  \item  不要在头文件中使用 \textit{命名空间别名} 除非显式标记内部命名空间使用。因为任何在头文件中引入的命名空间都会成为公开 API 的一部分。

%\begin{noindent}
\begin{cppcode}
// 在 .cpp 中使用别名缩短常用的命名空间
namespace baz = ::foo::bar::baz;
\end{cppcode}

\begin{cppcode}
// 在 .h 中使用别名缩短常用的命名空间
namespace librarian {
namespace impl {  // 仅限内部使用
namespace sidetable = ::pipeline_diagnostics::sidetable;
}  // namespace impl

inline void my_inline_function() {
  // 限制在一个函数中的命名空间别名
  namespace baz = ::foo::bar::baz;
...
}
}  // namespace librarian
\end{cppcode}
% \end{noindent}

  \item  禁止用内联命名空间。
\end{itemize}

\section{匿名命名空间和静态变量} \label{unnamed-namespace-and-static-variables}

\DBox {
在 \cppinline{.cpp} 文件中定义一个不需要被外部引用的变量时,可以将它们放在匿名命名空间或声明为 \cppinline{static} 。但是不要在\cppinline{.h} 文件中这么做。
}

\textbf{定义:}

所有置于匿名命名空间的声明都具有内部链接性,函数和变量可以经由声明为 \cppinline{static} 拥有内部链接性,这意味着你在这个文件中声明的这些标识符都不能在另一个文件中被访问。即使两个文件声明了完全一样名字的标识符,它们所指向的实体实际上是完全不同的。

\textbf{结论:}

推荐、鼓励在 \cppinline{.cpp} 中对于不需要在其他地方引用的标识符使用内部链接性声明,但是不要在 \cppinline{.h} 中使用。

匿名命名空间的声明和具名的格式相同,在最后注释上 \cppinline{namespace} :

%\begin{noindent}
\begin{cppcode}
namespace {
...
}  // namespace
\end{cppcode}
% \end{noindent}

\section{非成员函数、静态成员函数和全局函数} \label{nonmember-static-member-and-global-functions}

\DBox{
使用静态成员函数或命名空间内的非成员函数,尽量不要用裸的全局函数。将一系列函数直接置于命名空间中,不要用类的静态方法模拟出命名空间的效果,类的静态方法应当和类的实例或静态数据紧密相关。
}

\textbf{优点:}

某些情况下,非成员函数和静态成员函数是非常有用的,将非成员函数放在命名空间内可避免污染全局作用域。

\textbf{缺点:}

将非成员函数和静态成员函数作为新类的成员或许更有意义,当它们需要访问外部资源或具有重要的依赖关系时更是如此。

\textbf{结论:}

有时,把函数的定义同类的实例脱钩是有益的,甚至是必要的。这样的函数可以被定义成静态成员,或是非成员函数。

非成员函数不应依赖于外部变量,应尽量置于某个命名空间内。相比单纯为了封装若干不共享任何静态数据的静态成员函数而创建类,不如使用\DFullRef{qt-namespace}。举例而言,对于头文件 \cppinline{myproject/foo_bar.h}  , 应当使用

\begin{cppcode}
  namespace myproject {
  namespace foo_bar {
  void Function1();
  void Function2();
  }  // namespace foo_bar
  }  // namespace myproject
\end{cppcode}

而非

\begin{cppcode}
  namespace myproject {
      class FooBar {
          public:
          static void Function1();
          static void Function2();
        };
    }  // namespace myproject
\end{cppcode}

定义在同一编译单元的函数,被其他编译单元直接调用可能会引入不必要的耦合和链接时依赖; 静态成员函数对此尤其敏感。可以考虑提取到新类中,或者将函数置于独立库的命名空间内。

如果你必须定义非成员函数,又只是在 \cppinline{.cpp} 文件中使用它,可使用匿名 \DFullRef{qt-namespace} 或 \cppinline{static} 链接关键字 (如 \cppinline{static int Foo() {...}}) 限定其作用域。

\section{局部变量} \label{local-variables}

\DBox{
将函数变量尽可能置于最小作用域内,并在变量声明时进行初始化。
}

C++ 允许在函数的任何位置声明变量。我们提倡在尽可能小的作用域中声明变量,离第一次使用越近越好。这使得代码浏览者更容易定位变量声明的位置,了解变量的类型和初始值。特别是,应使用初始化的方式替代声明再赋值,比如:

\begin{cppcode}
  int i;
  i = f(); // 坏——初始化和声明分离

  int j = g(); // 好——初始化时声明

  vector<int> v;
  v.push_back(1); // 用花括号初始化更好
  v.push_back(2);

  vector<int> v = {1, 2}; // 好——v 一开始就初始化
\end{cppcode}

属于 \cppinline{if}, \cppinline{while} 和 \cppinline{for} 语句的变量应当在这些语句中正常地声明,这样子这些变量的作用域就被限制在这些语句中了,举例而言:

\begin{cppcode}
  while (const char* p = strchr(str, '/')) str = p + 1;
\end{cppcode}

\begin{DWarn}
有一个例外,如果变量是一个对象,每次进入作用域都要调用其构造函数,每次退出作用域都要调用其析构函数。这会导致效率降低。
\end{DWarn}

\begin{cppcode}
  // 低效的实现
  for (int i = 0; i < 1000000; ++i) {
      Foo f;                  // 构造函数和析构函数分别调用 1000000 次!
      f.DoSomething(i);
    }
\end{cppcode}

在循环作用域外面声明这类变量要高效的多:

\begin{cppcode}
  Foo f;                      // 构造函数和析构函数只调用 1 次
  for (int i = 0; i < 1000000; ++i) {
      f.DoSomething(i);
    }
\end{cppcode}


\section{静态和全局变量} \label{static-and-global-variables}

\DBox{
禁止定义静态储存周期非 POD 变量,禁止使用含有副作用的函数初始化 POD 全局变量,因为多编译单元中的静态变量执行时的构造和析构顺序是未明确的,这将导致代码的不可移植。
}

禁止使用类的\href{http://zh.cppreference.com/w/cpp/language/storage_duration#.E5.AD.98.E5.82.A8.E6.9C.9F}{静态储存周期}变量:由于构造和析构函数调用顺序的不确定性,它们会导致难以发现的 bug。不过 \cppinline{constexpr}变量除外,毕竟它们又不涉及动态初始化或析构。

静态生存周期的对象,即包括了全局变量,静态变量,静态类成员变量和函数静态变量,都必须是原生数据类型 (POD : Plain OldData): 即 int, char 和 float, 以及 POD 类型的指针、数组和结构体。

静态变量的构造函数、析构函数和初始化的顺序在 C++ 中是只有部分明确的,甚至随着构建变化而变化,导致难以发现的 bug。所以除了禁用类类型的全局变量,我们也不允许用函数返回值来初始化 POD 变量,除非该函数(比如 \cppinline{getenv()} 或\cppinline{getpid()})不涉及任何全局变量。函数作用域里的静态变量除外,毕竟它的初始化顺序是有明确定义的,而且只会在指令执行到它的声明那里才会发生。

\begin{DNote}
  Xris 译注:

  同一个编译单元内是明确的,静态初始化优先于动态初始化,初始化顺序按照声明顺序进行,销毁则逆序。不同的编译单元之间初始化和销毁顺序属于未明确行为
  (unspecified behaviour)。

\end{DNote}

同理,全局和静态变量在程序中断时会被析构,无论所谓中断是从 \cppinline{main()} 返回还是对 \cppinline{exit()} 的调用。析构顺序正好与构造函数调用的顺序相反。但既然构造顺序未定义,那么析构顺序当然也就不定了。比如,在程序结束时某静态变量已经被析构了,但代码还在跑——比如其它线程——并试图访问它且失败;再比如,一个静态 string 变量也许会在一个引用了前者的其它变量析构之前被析构掉。

改善以上析构问题的办法之一是用 \cppinline{quick_exit()} 来代替 \cppinline{exit()}并中断程序。它们的不同之处是前者不会执行任何析构,也不会执行 \cppinline{atexit()} 所绑定的任何 handlers。如果您想在执行 \cppinline{quick_exit()} 来中断时执行某 handler(比如刷新 log),您可以把它绑定到\cppinline{_at_quick_exit()}. 如果您想在 \cppinline{exit()} 和 \cppinline{quick_exit()} 都用上该 handler,都绑定上去。

综上所述,我们只允许 POD 类型的静态变量,即完全禁用 \cppinline{vector} (使用 C 数组替代) 和 \cppinline{string} (使用\cppinline{const char []})。

如果您确实需要一个 \cppinline{class} 类型的静态或全局变量,可以考虑在 \cppinline{main()} 函数或 \cppinline{pthread_once()}
内初始化一个指针且永不回收。注意只能用 raw 指针,别用智能指针,毕竟后者的析构函数涉及到上文指出的不定顺序问题。

\begin{DNote}
  Yang.Y 译注:

  上文提及的静态变量泛指静态生存周期的对象,包括:全局变量,静态变量,静态类成员变量,以及函数静态变量。
\end{DNote}

\section{注解}

\subsection{ 译者 (YuleFox) 笔记}

\begin{itemize}
  \item \cppinline{cpp} 中的匿名命名空间可避免命名冲突,限定作用域,避免直接使用 \cppinline{using} 关键字污染命名空间。
  \item 嵌套类符合局部使用原则,只是不能在其他头文件中前置声明,尽量不要 \cppinline{public}。
  \item 尽量不用全局函数和全局变量,考虑作用域和命名空间限制,尽量单独形成编译单元。
  \item 多线程中的全局变量 (含静态成员变量) 不要使用 \cppinline{class} 类型 (含 STL 容器), 避免不明确行为导致的 bug。
  \item 作用域的使用,除了考虑名称污染,可读性之外,主要是为降低耦合,提高编译/执行效率。
\end{itemize}

\subsection{  译者(acgtyrant)笔记 }

\begin{itemize}
  \item 注意「using 指示(using-directive)」和「using 声明(using-declaration)」的区别。
  \item 匿名命名空间说白了就是文件作用域,就像 C static 声明的作用域一样,后者已经被 C++ 标准提倡弃用。
  \item 局部变量在声明的同时进行显式值初始化,比起隐式初始化再赋值的两步过程要高效,同时也贯彻了计算机体系结构重要的概念「局部性(locality)」。
  \item 注意别在循环犯大量构造和析构的低级错误。
\end{itemize}
\input{qt/classes.tex}
\chapter{函数}

\section{参数顺序}

\textbf{总述}

函数的参数顺序为: 输入参数在先, 后跟输出参数。

\textbf{说明}

C/C++ 中的函数参数或者是函数的输入, 或者是函数的输出, 或兼而有之。 输入参数通常是值参或 \cppinline{const} 引用, 输出参数或输入/输出参数则一般为非 \cppinline{const} 指针。 在排列参数顺序时, 将所有的输入参数置于输出参数之前。 特别要注意, 在加入新参数时不要因为它们是新参数就置于参数列表最后, 而是仍然要按照前述的规则, 即将新的输入参数也置于输出参数之前。

这并非一个硬性规定。 输入/输出参数 (通常是类或结构体) 让这个问题变得复杂。 并且, 有时候为了其他函数保持一致, 你可能不得不有所变通。

\section{编写简短函数}

\textbf{总述}

我们倾向于编写简短, 凝练的函数。

\textbf{说明}

我们承认长函数有时是合理的, 因此并不硬性限制函数的长度。 如果函数超过 40 行, 可以思索一下能不能在不影响程序结构的前提下对其进行分割。

即使一个长函数现在工作的非常好, 一旦有人对其修改, 有可能出现新的问题, 甚至导致难以发现的 bug。 使函数尽量简短, 以便于他人阅读和修改代码。

在处理代码时, 你可能会发现复杂的长函数。 不要害怕修改现有代码: 如果证实这些代码使用 / 调试起来很困难, 或者你只需要使用其中的一小段代码, 考虑将其分割为更加简短并易于管理的若干函数。

\section{引用参数}

\textbf{总述}

所有按引用传递的参数必须加上 \cppinline{const}。

\textbf{定义}

在 C 语言中, 如果函数需要修改变量的值, 参数必须为指针, 如 \cppinline{int foo(int *pval)}。 在 C++ 中, 函数还可以声明为引用参数: \cppinline{int foo(int &val)}。

\textbf{优点}

定义引用参数可以防止出现 \cppinline{(*pval)++} 这样丑陋的代码。 引用参数对于拷贝构造函数这样的应用也是必需的。 同时也更明确地不接受空指针。

\textbf{缺点}

容易引起误解, 因为引用在语法上是值变量却拥有指针的语义。

\textbf{结论}

函数参数列表中, 所有引用参数都必须是 \cppinline{const}:

\begin{cppcode}
  void Foo(const string &in, string *out);
\end{cppcode}

事实上这在 Google Code 是一个硬性约定: 输入参数是值参或 \cppinline{const} 引用, 输出参数为指针。 输入参数可以是 \cppinline{const} 指针, 但决不能是非 \cppinline{const} 的引用参数, 除非特殊要求, 比如 \cppinline{swap()}。

有时候, 在输入形参中用 \cppinline{const T*} 指针比 \cppinline{const T&} 更明智。 比如:

* 可能会传递空指针。

* 函数要把指针或对地址的引用赋值给输入形参。

总而言之, 大多时候输入形参往往是 \cppinline{const T&}。 若用 \cppinline{const T*} 则说明输入另有处理。 所以若要使用 \cppinline{const T*}, 则应给出相应的理由, 否则会使得读者感到迷惑。

\section{函数重载} \label{function-overloading}

\textbf{总述}

若要使用函数重载, 则必须能让读者一看调用点就胸有成竹, 而不用花心思猜测调用的重载函数到底是哪一种。 这一规则也适用于构造函数。

\textbf{定义}

你可以编写一个参数类型为 \cppinline{const string&} 的函数, 然后用另一个参数类型为 \cppinline{const char*} 的函数对其进行重载:


\begin{cppcode}
  class MyClass {
      public:
      void Analyze(const string &text);
      void Analyze(const char *text, size_t textlen);
    };
\end{cppcode}

\textbf{优点}

通过重载参数不同的同名函数, 可以令代码更加直观。 模板化代码需要重载, 这同时也能为使用者带来便利。

\textbf{缺点}

如果函数单靠不同的参数类型而重载 (acgtyrant 注:这意味着参数数量不变), 读者就得十分熟悉 C++ 五花八门的匹配规则, 以了解匹配过程具体到底如何。 另外, 如果派生类只重载了某个函数的部分变体, 继承语义就容易令人困惑。

\textbf{结论}
% TODO: change \textbf{列表初始化格式} ->  \DFullRef{braced-initializer-list} in formatting.tex not work now.
如果打算重载一个函数, 可以试试改在函数名里加上参数信息。 例如, 用 \cppinline{AppendString()} 和 \cppinline{AppendInt()} 等, 而不是一口气重载多个 \cppinline{Append()}。 如果重载函数的目的是为了支持不同数量的同一类型参数, 则优先考虑使用 \cppinline{std::vector} 以便使用者可以用\textbf{列表初始化格式}指定参数。

\section{缺省参数}

\textbf{总述}

只允许在非虚函数中使用缺省参数, 且必须保证缺省参数的值始终一致。 缺省参数与 \DFullRef{function-overloading} 遵循同样的规则。 一般情况下建议使用函数重载, 尤其是在缺省函数带来的可读性提升不能弥补下文中所提到的缺点的情况下。

\textbf{优点}

有些函数一般情况下使用默认参数, 但有时需要又使用非默认的参数。 缺省参数为这样的情形提供了便利, 使程序员不需要为了极少的例外情况编写大量的函数。 和函数重载相比, 缺省参数的语法更简洁明了, 减少了大量的样板代码, 也更好地区别了 "必要参数" 和 "可选参数"。

\textbf{缺点}

缺省参数实际上是函数重载语义的另一种实现方式, 因此所有 \DFullRef{function-overloading} 也都适用于缺省参数。

虚函数调用的缺省参数取决于目标对象的静态类型, 此时无法保证给定函数的所有重载声明的都是同样的缺省参数。

缺省参数是在每个调用点都要进行重新求值的, 这会造成生成的代码迅速膨胀。 作为读者, 一般来说也更希望缺省的参数在声明时就已经被固定了, 而不是在每次调用时都可能会有不同的取值。

缺省参数会干扰函数指针, 导致函数签名与调用点的签名不一致。 而函数重载不会导致这样的问题。

\textbf{结论}

对于虚函数, 不允许使用缺省参数, 因为在虚函数中缺省参数不一定能正常工作。 如果在每个调用点缺省参数的值都有可能不同, 在这种情况下缺省函数也不允许使用。 (例如, 不要写像 \cppinline{void f(int n = counter++);} 这样的代码。)

在其他情况下, 如果缺省参数对可读性的提升远远超过了以上提及的缺点的话, 可以使用缺省参数。 如果仍有疑惑, 就使用函数重载。

\section{函数返回类型后置语法}

\textbf{总述}

只有在常规写法 (返回类型前置) 不便于书写或不便于阅读时使用返回类型后置语法。

\textbf{定义}

C++ 现在允许两种不同的函数声明方式。 以往的写法是将返回类型置于函数名之前。 例如:

\begin{cppcode}
int foo(int x);
\end{cppcode}

C++11 引入了这一新的形式。 现在可以在函数名前使用 \cppinline{auto} 关键字, 在参数列表之后后置返回类型。 例如:

\begin{cppcode}
auto foo(int x) -> int;
\end{cppcode}

后置返回类型为函数作用域。 对于像 \cppinline{int} 这样简单的类型, 两种写法没有区别。 但对于复杂的情况, 例如类域中的类型声明或者以函数参数的形式书写的类型, 写法的不同会造成区别。

\textbf{优点}
% TODO: change \textbf{Lambda 表达式} -> \DFullRef{lambda-expressions} in formatting.tex not work now.
后置返回类型是显式地指定\textbf{Lambda 表达式}的返回值的唯一方式。 某些情况下, 编译器可以自动推导出 Lambda 表达式的返回类型, 但并不是在所有的情况下都能实现。 即使编译器能够自动推导, 显式地指定返回类型也能让读者更明了。

有时在已经出现了的函数参数列表之后指定返回类型, 能够让书写更简单, 也更易读, 尤其是在返回类型依赖于模板参数时。 例如:

\begin{cppcode}
  template <class T, class U> auto add(T t, U u) -> decltype(t + u);
\end{cppcode}

对比下面的例子:

\begin{cppcode}
  template <class T, class U> decltype(declval<T&>() + declval<U&>()) add(T t, U u);
\end{cppcode}

\textbf{缺点}

后置返回类型相对来说是非常新的语法, 而且在 C 和 Java 中都没有相似的写法, 因此可能对读者来说比较陌生。

在已有的代码中有大量的函数声明, 你不可能把它们都用新的语法重写一遍。 因此实际的做法只能是使用旧的语法或者新旧混用。 在这种情况下, 只使用一种版本是相对来说更规整的形式。

\textbf{结论}

在大部分情况下, 应当继续使用以往的函数声明写法, 即将返回类型置于函数名前。 只有在必需的时候 (如 Lambda 表达式) 或者使用后置语法能够简化书写并且提高易读性的时候才使用新的返回类型后置语法。 但是后一种情况一般来说是很少见的, 大部分时候都出现在相当复杂的模板代码中, 而多数情况下不鼓励写这样 \DFullRef{qt-template-metaprogramming}。

\input{qt/magic.tex}
% TODO(iceyer): use Qt smart pointer or STL?
\section{预处理宏} \label{qt-preprocessor-macros}

\begin{DNote}
  使用宏时要非常谨慎,尽量以内联函数,枚举和常量代替之。
\end{DNote}

宏意味着你和编译器看到的代码是不同的。这可能会导致异常行为,尤其因为宏具有全局作用域。

值得庆幸的是,C++ 中,宏不像在 C 中那么必不可少。以往用宏展开性能关键的代码,现在可以用内联函数替代。用宏表示常量可被 \cppinline{const} 变量代替。用宏 "缩写" 长变量名可被引用代替。用宏进行条件编译... 这个,千万别这么做,会令测试更加痛苦 ( \cppinline{#define} 防止头文件重包含当然是个特例).

宏可以做一些其他技术无法实现的事情,在一些代码库 (尤其是底层库中) 可以看到宏的某些特性 (如用 \cppinline{#} 字符串化,用 \cppinline{##} 连接等等). 但在使用前,仔细考虑一下能不能不使用宏达到同样的目的。

下面给出的用法模式可以避免使用宏带来的问题; 如果你要宏,尽可能遵守:

\begin{itemize}
  \item 不要在 \cppinline{.h} 文件中定义宏;
  \item 在马上要使用时才进行 \cppinline{#define}, 使用后要立即 \cppinline{#undef};
  \item 不要只是对已经存在的宏使用 \cppinline{#undef},选择一个不会冲突的名称;
  \item 不要试图使用展开后会导致 C++ 构造不稳定的宏,不然也至少要附上文档说明其行为;
  \item 不要用 \cppinline{##} 处理函数,类和变量的名字。
\end{itemize}

\section{模板编程} \label{qt-template-metaprogramming}

\begin{DNote}
不要使用复杂的模板编程。
\end{DNote}

\textbf{定义}

模板编程指的是利用 c++ 模板实例化机制是图灵完备性,可以被用来实现编译时刻的类型判断的一系列编程技巧。

\textbf{优点}

模板编程能够实现非常灵活的类型安全的接口和极好的性能,一些常见的工具比如 Google Test, std::tuple, std::function 和 Boost.Spirit. 这些工具如果没有模板是实现不了的。

\textbf{缺点}

\begin{itemize}
  \item 模板编程所使用的技巧对于使用 c++ 不是很熟练的人是比较晦涩,难懂的。在复杂的地方使用模板的代码让人更不容易读懂,并且 debug 和 维护起来都很麻烦。
  \item 模板编程经常会导致编译出错的信息非常不友好:在代码出错的时候,即使这个接口非常的简单,模板内部复杂的实现细节也会在出错信息显示。导致这个编译出错信息看起来非常难以理解。
  \item 大量的使用模板编程接口会让重构工具 (Visual Assist X, Refactor for C++ 等等) 更难发挥用途。首先模板的代码会在很多上下文里面扩展开来,所以很难确认重构对所有的这些展开的代码有用,其次有些重构工具只对已经做过模板类型替换的代码的 AST 有用。因此重构工具对这些模板实现的原始代码并不有效,很难找出哪些需要重构。
\end{itemize}

\textbf{结论}

\begin{itemize}
  \item 模板编程有时候能够实现更简洁更易用的接口,但是更多的时候却适得其反。因此模板编程最好只用在少量的基础组件,基础数据结构上,因为模板带来的额外的维护成本会被大量的使用给分担掉。
  \item 在使用模板编程或者其他复杂的模板技巧的时候,你一定要再三考虑一下。考虑一下你们团队成员的平均水平是否能够读懂并且能够维护你写的模板代码。或者一个非 c++ 程序员和一些只是在出错的时候偶尔看一下代码的人能够读懂这些错误信息或者能够跟踪函数的调用流程。如果你使用递归的模板实例化,或者类型列表,或者元函数,又或者表达式模板,或者依赖 SFINAE, 或者 sizeof 的 trick 手段来检查函数是否重载,那么这说明你模板用的太多了,这些模板太复杂了,我们不推荐使用。
  \item 如果你使用模板编程,你必须考虑尽可能的把复杂度最小化,并且尽量不要让模板对外暴露。你最好只在实现里面使用模板,然后给用户暴露的接口里面并不使用模板,这样能提高你的接口的可读性。并且你应该在这些使用模板的代码上写尽可能详细的注释。你的注释里面应该详细的包含这些代码是怎么用的,这些模板生成出来的代码大概是什么样子的。还需要额外注意在用户错误使用你的模板代码的时候需要输出更人性化的出错信息。因为这些出错信息也是你的接口的一部分,所以你的代码必须调整到这些错误信息在用户看起来应该是非常容易理解,并且用户很容易知道如何修改这些错误。
\end{itemize}
\chapter{命名约定}

最重要的一致性规则是命名管理。命名的风格能让我们在不需要去查找类型声明的条件下快速地了解某个名字代表的含义: 类型,变量,函数,常量,宏,等等,甚至 我们大脑中的模式匹配引擎非常依赖这些命名规则。\textbf{命名规则具有一定随意性,但相比按个人喜好命名,一致性更重要,所以无论你认为它们是否重要,规则总归是规则。}

\section{通用命名规则} \label{general-naming-rules}

\textbf{总述}

函数命名,变量命名,文件命名要有描述性; 少用缩写。

\textbf{说明}

尽可能使用描述性的命名,别心疼空间,毕竟相比之下让代码易于新读者理解更重要。不要用只有项目开发者能理解的缩写,也不要通过砍掉几个字母来缩写单词.

\begin{cppcode}
  int priceCountReader;     // 无缩写
  int numErrors;            // "num" 是一个常见的写法
  int numDnsConnections;    // 人人都知道 "DNS" 是什么
\end{cppcode}

\begin{cppcode}
  int n;                     // 毫无意义.
  int nerr;                  // 含糊不清的缩写.
  int nCompConns;            // 含糊不清的缩写.
  int wgcConnections;        // 只有贵团队知道是什么意思.
  int pcReader;              // "pc" 有太多可能的解释了.
  int cstmrID;               // 删减了若干字母.
\end{cppcode}

注意,一些特定的广为人知的缩写是允许的,例如用 \cppinline{i} 表示迭代变量和用 \cppinline{T} 表示模板参数。

\begin{DWarn}
  在D-Pointer风格中,\cppinline{d_ptr,q_ptr,dd_ptr,qq_ptr}都是保留的名称。
\end{DWarn}

模板参数的命名应当遵循对应的分类: 类型模板参数应当遵循 \DFullRef{type-names} 的规则,而非类型模板应当遵循  \DFullRef{variable-names} 的规则.

\section{文件命名}

\textbf{总述}

文件名要全部小写,可以包含下划线 (\cppinline{_}) 。如果存在无法\cppinline{_}的情况,可以考虑使用连字符\cppinline{-},否则不允许例外。

\begin{DWarn}
  Qt默认情况下不使用任何连接符合,这使得文件名非常难以看懂,我们不接受这种风格。deepin的Qt项目统一使用下划线(\cppinline{_})作为文件名连接符合。
\end{DWarn}

\textbf{说明}

可接受的文件命名示例:

\begin{cppcode}
  my_useful_class.cpp
  myusefulclass_test.cpp // \cppinline{_unittest} 和 \cppinline{_regtest} 已弃用.
\end{cppcode}

不接受的文件命名示例:

\begin{cppcode}
  my-useful-class.cpp  // 不接受,除非无法使用_
  myusefulclass.cpp    // 不接受,难以看懂
\end{cppcode}

C++ 文件要以 \cppinline{.cpp} 结尾,头文件以 \cppinline{.h} 结尾。专门插入文本的文件则以 \cppinline{.inc} 结尾,参见 \DFullRef{self-contained-headers}。

不要使用已经存在于 \cppinline{/usr/include} 下的文件名 (Yang.Y 注: 即编译器搜索系统头文件的路径),如 \cppinline{db.h}。

通常应尽量让文件名更加明确。\cppinline{http_server_logs.h} 就比 \cppinline{logs.h} 要好。定义类时文件名一般成对出现,如 \cppinline{foo_bar.h} 和 \cppinline{foo_bar.cpp},对应于类 \cppinline{FooBar}。

内联函数必须放在 \cppinline{.h} 文件中。如果内联函数比较短,就直接放在 \cppinline{.h} 中.

\section{类型命名} \label{type-names}

\textbf{总述}

类型名称的每个单词首字母均大写,不包含下划线: \cppinline{MyExcitingClass},\cppinline{MyExcitingEnum}。

\textbf{说明}

所有类型命名 —— 类,结构体,类型定义 (\cppinline{typedef}),枚举,类型模板参数 —— 均使用相同约定,即以大写字母开始,每个单词首字母均大写,不包含下划线。例如:

\begin{cppcode}
  // 类和结构体
  class UrlTable { ...
  class UrlTableTester { ...
  struct UrlTableProperties { ...

  // 类型定义
  typedef hash_map<UrlTableProperties *, string> PropertiesMap;

  // using 别名
  using PropertiesMap = hash_map<UrlTableProperties *, string>;

  // 枚举
  enum UrlTableErrors { ...
\end{cppcode}

\section{变量命名} \label{variable-names}

\textbf{总述}

\begin{DWarn}
  变量 (包括函数参数) 和数据成员名一律使用驼峰命名。

  在D-Pointer的Private类中,成员变量不加任何修饰。

  在一般的类中,使用\cppinline{m_}开头来标记成员变量。
\end{DWarn}

\textbf{说明}

\subsection{普通变量命名}

举例:

\begin{cppcode}
  QString tableName;   // 接受,驼峰命名

  QString table_name;  // 不接受 - 用下划线.
  QString tablename;   // 不接受 - 全小写.
\end{cppcode}

\subsection{类数据成员}

不管是静态的还是非静态的,类数据成员都可以和普通变量一样,但是需要使用\cppinline{m_}前缀来修饰。

\begin{DWarn}
  为了实现较好的封装,Qt中大量使用D-Pointer技术,在这种情况下,一般通过\cppinline{d->localValue}的方式访问Private类的变量,这时候就不需要使用\cppinline{m_}来修饰成员变量。
  对于非D-Pointer的Private类,使用\cppinline{m_}前缀来修饰成员变量。
\end{DWarn}


\begin{cppcode}
  class TableInfo {
      ...
      private:
      QString m_tableName;               // 好
      static Pool<TableInfo>* m_pool;    // 好
    };

  class TableInfoPrivate {
      ...
      public:
      QString tableName;               // 好,Private类不需要任何修饰
      static Pool<TableInfo>* pool;    // 好,Private类不需要任何修饰
    };
\end{cppcode}

\subsection{结构体变量}

不管是静态的还是非静态的,结构体数据成员都可以和普通变量一样,不用像类那样接下划线:

\begin{cppcode}
  struct TableInfoData {
      QString tableName;               // 好,命名风格和Private保持一致
      static Pool<TableInfo>* pool;    // 好,命名风格和Private保持一致
    }
\end{cppcode}

结构体与类的使用讨论,参考 \DFullRef{structs-vs-classes}。

\section{常量命名} \label{constant-names}

\textbf{总述}

声明为 \cppinline{constexpr} 或 \cppinline{const} 的变量,或在程序运行期间其值始终保持不变的,命名时以 "k" 开头,大小写混合。例如:

\begin{cppcode}
  const int kDaysInAWeek = 7;
\end{cppcode}

\textbf{说明}

所有具有静态存储类型的变量 (例如静态变量或全局变量,参见 \href{http://en.cppreference.com/w/cpp/language/storage_duration#Storage_duration}{存储类型}) 都应当以此方式命名。对于其他存储类型的变量,如自动变量等,这条规则是可选的。如果不采用这条规则,就按照一般的变量命名规则。

\section{函数命名} \label{function-names}

\textbf{总述}

常规函数使用大小写混合,取值和设值函数则要求与变量名匹配: \cppinline{myExcitingFunction()},\cppinline{myExcitingMethod()},\cppinline{my_exciting_member_variable()},\cppinline{set_my_exciting_member_variable()}。

\textbf{说明}

一般来说,函数名首字母小写,每个单词首字母大写 (即 "驼峰变量名" 或 "帕斯卡变量名"),没有下划线。对于首字母缩写的单词,更倾向于将它们视作一个单词进行首字母大写 (例如,写作 \cppinline{startRpc()} 而非 \cppinline{startRPC()})。

\begin{cppcode}
  AddTableEntry()
  DeleteUrl()
  OpenFileOrDie()
\end{cppcode}

\begin{DWarn}
  对于DBus接口函数,属性,信号,确保首字母大写,这也适用于一些其他风格的IPC/RPC接口或代码生成器生成的接口,包括dbus/protobuf/thrift等。
\end{DWarn}

同样的命名规则同时适用于类作用域与命名空间作用域的常量,因为它们是作为 API 的一部分暴露对外的,因此应当让它们看起来像是一个函数,因为在这时,它们实际上是一个对象而非函数的这一事实对外不过是一个无关紧要的实现细节。

取值和设值函数的命名与变量一致。一般来说它们的名称与实际的成员变量对应,但并不强制要求。例如 \cppinline{int getCount()} 与 \cppinline{void setCount(int count)}。

\section{命名空间命名}

\textbf{总述}
\begin{DWarn}
  命名空间以大写字母命名。最高级命名空间的名字取决于项目名称。要注意避免嵌套命名空间的名字之间和常见的顶级命名空间的名字之间发生冲突.
\end{DWarn}

顶级命名空间的名称应当是项目名或者是该命名空间中的代码所属的团队的名字。命名空间中的代码,应当存放于和命名空间的名字匹配的文件夹或其子文件夹中.

注意 \DFullRef{general-naming-rules} 的规则同样适用于命名空间。命名空间中的代码极少需要涉及命名空间的名称,因此没有必要在命名空间中使用缩写.

要避免嵌套的命名空间与常见的顶级命名空间发生名称冲突。由于名称查找规则的存在,命名空间之间的冲突完全有可能导致编译失败。尤其是,不要创建嵌套的 \cppinline{std} 命名空间。建议使用更独特的项目标识符 (\cppinline{WebSearch::Index},\cppinline{WebSearch::IndexUtil}) 而非常见的极易发生冲突的名称 (比如 \cppinline{WebSearch::Util}).

对于 \cppinline{Internal} 命名空间,要当心加入到同一 \cppinline{internal} 命名空间的代码之间发生冲突 (由于内部维护人员通常来自同一团队,因此常有可能导致冲突)。在这种情况下,请使用文件名以使得内部名称独一无二 (例如对于 \cppinline{frobber.h},使用 \cppinline{WebSearch::Index::FrobberInternal})。

\section{枚举命名}

\textbf{总述}

枚举的命名应当和 \DFullRef{type-names} 一致: \cppinline{EnumName} 。

\textbf{说明}

单独的枚举值使用首字母大写的大小写混合命名方式。枚举名 \cppinline{UrlTableErrors} (以及 \cppinline{AlternateUrlTableErrors}) 是类型,所以要用大小写混合的方式.

\begin{cppcode}
  enum UrlTableErrors {
      OK = 0,
      ErrorOutOfMemory,
      ErrorMalformedInput,
    };
\end{cppcode}

2009 年 1 月之前,Google 一直建议采用\DFullRef{macro-names}的方式命名枚举值。由于枚举值和宏之间的命名冲突,直接导致了很多问题。由此,这里改为优先选择\DFullRef{type-names}的方式。新代码应该尽可能优先使用\DFullRef{type-names}的方式。但是老代码没必要切换到\DFullRef{type-names}的方式,除非宏风格确实会产生编译期问题。

\section{宏命名} \label{macro-names}

\textbf{总述}

你并不打算 \DFullRef{preprocessor-macros},对吧? 如果你一定要用,像这样命名:

\cppinline{MY_MACRO_THAT_SCARES_SMALL_CHILDREN}.

\textbf{说明}

参考 \DFullRef{preprocessor-macros}; 通常 *不应该* 使用宏。如果不得不用,其命名像枚举命名一样全部大写,使用下划线:

\begin{cppcode}
  #define ROUND(x) ...
  #define PI_ROUNDED 3.0
\end{cppcode}

\section{命名规则的特例}

\textbf{总述}

如果你命名的实体与已有 C/C++ 实体相似,可参考现有命名策略。

\begin{DWarn}
  如果是为了扩展STL的接口,或继承其他底层库的函数,则可以不受命名规则限制,以避免功能错误。
\end{DWarn}

\cppinline{bigopen()}: 函数名,参照 \cppinline{open()} 的形式

\cppinline{uint}: \cppinline{typedef}

\cppinline{bigpos}: \cppinline{struct} 或 \cppinline{class},参照 \cppinline{pos} 的形式

\cppinline{sparse_hash_map}: STL 型实体; 参照 STL 命名约定

\cppinline{LONGLONG_MAX}: 常量,如同 \cppinline{INT_MAX}

\section{注解}

\subsection{ 译者 (YuleFox) 笔记}

感觉 Google 的命名约定很高明,比如写了简单的类 QueryResult,接着又可以直接定义一个变量 \cppinline{query_result},区分度很好; 再次,类内变量以下划线结尾,那么就可以直接传入同名的形参,比如 \cppinline{TextQuery::TextQuery(std::string word) : word_(word) {}} ,其中 \cppinline{word_} 自然是类内私有成员.

\subsection{ deepin风格注解 }

对Qt风格同理:

比如写了简单的类 \cppinline{QueryResult},接着又可以直接定义一个变量 \cppinline{queryResult},区分度很好; 再次,类内变量以下划线结尾,那么就可以直接传入同名的形参,比如 \cppinline{TextQuery::TextQuery(QString word) : m_word(word) {}} ,其中 \cppinline{m_word} 自然是类内私有成员.

\input{qt/comments.tex}
% TODO(iceyer): use clang-format
% \input{qt/formatting.tex}
\chapter{规则特例}

前面说明的编程习惯基本都是强制性的。 但所有优秀的规则都允许例外, 这里就是探讨这些特例。

\section{现有不合规范的代码}

\textbf{总述}

对于现有不符合既定编程风格的代码可以网开一面。

\textbf{说明}

当你修改使用其他风格的代码时, 为了与代码原有风格保持一致可以不使用本指南约定。 如果不放心, 可以与代码原作者或现在的负责人员商讨。 记住, *一致性* 也包括原有的一致性。

\section{Windows 代码} \label{windows-code}

\textbf{总述}

Windows 程序员有自己的编程习惯, 主要源于 Windows 头文件和其它 Microsoft 代码。 我们希望任何人都可以顺利读懂你的代码, 所以针对所有平台的 C++ 编程只给出一个单独的指南。

\textbf{说明}

如果你习惯使用 Windows 编码风格, 这儿有必要重申一下某些你可能会忘记的指南:

% TODO(iceyer): naming and pragma once need to be discuss
\begin{itemize}
  \item  不要使用匈牙利命名法 (比如把整型变量命名成 \cppinline{iNum})。 使用 Google 命名约定, 包括对源文件使用 \cppinline{.cc} 扩展名。
  \item Windows 定义了很多原生类型的同义词 (YuleFox 注: 这一点, 我也很反感), 如 \cppinline{DWORD}, \cppinline{HANDLE} 等等。 在调用 Windows API 时这是完全可以接受甚至鼓励的。 即使如此, 还是尽量使用原有的 C++ 类型, 例如使用 \cppinline{const TCHAR *} 而不是 \cppinline{LPCTSTR}。
  \item 使用 Microsoft Visual C++ 进行编译时, 将警告级别设置为 3 或更高, 并将所有警告(warnings)当作错误(errors)处理。
  \item 不要使用 \cppinline{#pragma once}; 而应该使用 Google 的头文件保护规则。 头文件保护的路径应该相对于项目根目录 (Yang.Y 注: 如 \cppinline{#ifndef SRC_DIR_BAR_H_}, 参考 \DFullRef{qt-pragma-once-guard} 一节)。
  \item 除非万不得已, 不要使用任何非标准的扩展, 如 \cppinline{#pragma} 和 \cppinline{__declspec}。 使用 \cppinline{__declspec(dllimport)} 和 \cppinline{__declspec(dllexport)} 是允许的, 但必须通过宏来使用, 比如 \cppinline{DLLIMPORT} 和 \cppinline{DLLEXPORT}, 这样其他人在分享使用这些代码时可以很容易地禁用这些扩展。
\end{itemize}

然而, 在 Windows 上仍然有一些我们偶尔需要违反的规则:

\begin{itemize}
  \item 通常我们 \DFullRef{multiple-inheritance}, 但在使用 COM 和 ATL/WTL 类时可以使用多重继承。 为了实现 COM 或 ATL/WTL 类/接口, 你可能不得不使用多重实现继承。
  \item  虽然代码中不应该使用异常, 但是在 ATL 和部分 STL(包括 Visual C++ 的 STL) 中异常被广泛使用。 使用 ATL 时, 应定义 \cppinline{_ATL_NO_EXCEPTIONS} 以禁用异常。 你需要研究一下是否能够禁用 STL 的异常, 如果无法禁用, 可以启用编译器异常。 (注意这只是为了编译 STL, 自己的代码里仍然不应当包含异常处理)。
  \item  通常为了利用头文件预编译, 每个每个源文件的开头都会包含一个名为 \cppinline{StdAfx.h} 或 \cppinline{precompile.h} 的文件。 为了使代码方便与其他项目共享, 请避免显式包含此文件 (除了在 \cppinline{precompile.cc} 中), 使用 \cppinline{/FI} 编译器选项以自动包含该文件。
  \item  资源头文件通常命名为 \cppinline{resource.h} 且只包含宏, 这一文件不需要遵守本风格指南。
\end{itemize}
\input{qt/end.tex}

\part{C代码项目风格}

\begin{DNote}
    本部分内容参考\href{https://raw.githubusercontent.com/systemd/systemd/main/docs/CODING_STYLE.md}{Systemd 代码风格} 和 \href{https://github.com/zh-google-styleguide/zh-google-styleguide}{Google 开源项目风格指南——中文版}》。
\end{DNote}

\chapter{语义}

对外部接口,使用 ISO C89(ANSI X3.159-1989)标准,避免外部使用者必须使用 ISO C11(ISO/IEC 9899:2011)。

对于内部库,可以 GNU 扩展的 ISO C11(也称为 "gnu11")或 ISO C11 标准。

编译器建议最低支持到 GCC 8.3.0。

\section{头文件}

\subsection{头文件必须包含守卫}

\textbf{总述}

以 .h 或 为扩展名的头文件应包含头文件守卫。例中 foo.h 是“Library”模块中的头文件,宏 \cinline{LIBRARY_FOO_H} 即可作为它的守卫,保证头文件被重复引入也不会出现问题,守卫名称不可有重复,建议守卫名称遵循“模块名\_文件名”的形式。

\cinline{#pragma once} 指令也可作为头文件守卫,但并不是 C/C++ 的标准方式,只是多数编译器均有支持。这种方式由编译器维护一个列表,引入头文件时,如果发现文件中有 \cinline{#pragma once}  指令就将文件路径加入列表,当这个文件再次被 include 时便不会加载,而宏守卫的方式仍然要对文件进行预编译,所以 \cinline{#pragma once}  方式在编译效率上会更高一些。

\begin{DWarn}
注意在\deepin \ Qt 风格中,不允许使用\cinline{#pragma once} 。同时为兼容性考虑,在公开的头文件中,建议也不要使用\cinline{#pragma once} 。
\end{DWarn}

\textbf{说明}

\begin{ccode}
// Header file foo.h
#ifndef LIBRARY_FOO_H
#define LIBRARY_FOO_H
....
#endif

// Header file foo.h
#pragma once
\end{ccode}

\subsection{头文件不能有静态声明}

\textbf{总述}

头文件中由 static 关键字声明的对象、数组或函数,会在每个包含该头文件的翻译单元或模块中生成副本造成数据冗余,如果将静态数据误用作全局数据也会造成逻辑错误。

\textbf{说明}

\begin{ccode}
// In a header file
static int i = 0;   // 不合规,头文件中错误使用

static int foo() {  // 不合规,头文件中错误使用
    return i;
}
\end{ccode}

\subsection{头文件不能有函数实现}

\textbf{总述}

头文件是项目文档的重要组成部分,有必要保持头文件简洁清晰,头文件的主要内容应是类型或接口的声明。除非函数很简短,否则也不建议在头文件中内联实现,大段的函数实现会影响头文件的可读性。

定义在头文件中的函数发生变化时,所有相关模块均需重新编译,会增加构建和维护成本,在使用动态链接库时这个问题尤为突出,如果库的导入者没有及时编译,可能会造成严重后果。在头文件中定义的函数是模块二进制接口的一部分,应合理规划以降低维护成本。

\subsection{头文件合理声明}

\textbf{总述}

如果声明的位置不合理会降低代码的可维护性,甚至会导致标准未定义的行为。应遵循如下原则:

\begin{itemize}
    \item 外部链接的对象或函数应在头文件中声明,并避免重复声明
    \item 内部链接的对象或函数应在源文件中声明,不应在头文件中声明
    \item 避免在头文件外手工书写外部声明
    \item 避免在局部作用域内声明函数或全局对象
\end{itemize}

\begin{ccode}

int fun()
{
    extern int g;       // 不合规
    extern int foo();   // 不合规
    static int bar();   // 不合规,未定义行为
}
\end{ccode}
\chapter{命名约定}

\section{通用命名规则} \label{c-general-naming-rules}

\textbf{总述}

函数命名,变量命名,文件命名要有描述性,少用缩写。

\textbf{说明}

尽可能使用描述性的命名,别心疼空间,毕竟相比之下让代码易于新读者理解更重要。不要用只有项目开发者能理解的缩写,也不要通过砍掉几个字母来缩写单词。

可接受的命名方式:

\begin{ccode}
int price_count_reader;     // 无缩写
int num_errors;             // "num" 是一个常见的写法
int num_dns_connections;    // 人人都知道 "DNS" 是什么
\end{ccode}

不好的命名方式:

\begin{ccode}
int n;                    // 毫无意义
int n_err;                // 含糊不清的缩写
int pc_reader;            // 只有贵团队知道是什么意思
int cstmr_id;             // 删减了若干字母
int l, I, O, l0, Il;      // 易与数字混淆名称
int YE5, N0;              // 易与单词混淆名称
int fxxk;                 // 不良名称
\end{ccode}

同时需要注意编译器限制,如果两个名称有相同的前缀,当相同前缀超过一定长度时是危险的,可能会导致编译器无法有效区分相关名称。C 标准指明,保证名称前 31 位不同可避免这种问题。

\begin{ccode}
extern int identifier_of_a_very_very_long_name_1;  // 过长字符,部分编译器无法区分
extern int identifier_of_a_very_very_long_name_2;
\end{ccode}

避免有特殊意义的名称,包括:

\begin{itemize}
  \item 标准库或编译环境中的宏名称
  \item 标准库中具有外部链接性的对象或函数名称
  \item 标准库中的类型名称
\end{itemize}

\begin{ccode}
int errno;  //变量 errno 与标准库中的 errno 名称相同,不便于区分是自定义的还是系统定义的
\end{ccode}

\section{文件命名}

\textbf{总述}

文件名要全部小写,可以包含下划线(\cinline{_}) 。如果存在无法使用(\cinline{_})的情况,可以考虑使用连字符(\cinline{-}),否则不允许例外。

\textbf{说明}

可接受的文件命名:

\begin{ccode}
my_useful_class.c
my_useful_class_test.c
\end{ccode}

不接受的文件命名:

\begin{ccode}
my-useful-class.c  // 不接受,除非无法使用_
myusefulclass.c    // 不接受,难以看懂
\end{ccode}

不要使用标准库中的文件名,例如\cinline{stdio.h}。通常应尽量让文件名更加明确。\cinline{http_server_logs.h} 就比 \cinline{logs.h} 要好。

\section{类型命名} \label{c-type-names}

\textbf{总述}

类型名称的每个单词首字母均大写,不包含下划线:\cinline{MyExcitingStruct}。

\textbf{说明}

结构体,类型定义 (\cinline{typedef}),以大写字母开始,每个单词首字母均大写,不包含下划线。例如:

\begin{ccode}
typedef struct SessionStatusInfo {
    const char *id;
    const char *remote_host;
} SessionStatusInfo;

typedef union UnitDependencyInfo {
    void *data;
    struct {
        UnitDependencyMask origin_mask:16;
        UnitDependencyMask destination_mask:16;
    } _packed_;
} UnitDependencyInfo;
\end{ccode}

\section{变量命名} \label{c-variable-names}

\textbf{总述}

变量命名一律使用蛇形命名法(snake\_case)。

\textbf{说明}

\begin{ccode}
char *table_name;
char *tableName;   // 不接受 - 用下划线。
char *tablename;   // 不接受,无分割符合。
\end{ccode}

\section{函数命名} \label{c-function-names}

\textbf{总述}

函数命名一律使用蛇形命名法(snake\_case)。

\textbf{说明}

\begin{ccode}
  int is_system_up();  // 一个名为 is_system_up 的功能性函数
\end{ccode}

\section{宏命名} \label{c-macro-names}

\textbf{总述}

尽可能少的使用宏,如果必须使用,则用全大写,并使用下划线分隔。\cinline{MY_MACRO_THAT_SCARES_SMALL_CHILDREN}.

\textbf{说明}

\begin{ccode}
#define ROUND(x) ...
#define PI_ROUNDED 3.0
\end{ccode}

\section{枚举命名} \label{c-enum-names}

\textbf{总述}
枚举本身命名参考宏命名。

\textbf{说明}

对于标志类型(Flags)如果可能的话,使用枚举。通过 \cinline{1 <<} 表达式指示位值,并垂直对齐它们。为其定义枚举和类型。

\begin{ccode}
typedef enum FoobarFlags {
    FOOBAR_QUUX  = 1 << 0,
    FOOBAR_WALDO = 1 << 1,
    FOOBAR_XOXO  = 1 << 2,
} FoobarFlags;
\end{ccode}

对于非标志类型,定义一个 \cinline{_MAX} 枚举,用于表示已定义枚举值中最大的值,再加一。由于这不是常规枚举值,请使用 \cinline{_} 作为前缀。同时,定义一个特殊的“无效”枚举值,并将其设置为 \cinline{-EINVAL}。这样枚举类型可以安全地用于传播转换错误。

\begin{ccode}
typedef enum FoobarMode {
    FOOBAR_AAA,
    FOOBAR_BBB,
    FOOBAR_CCC,
    _FOOBAR_MAX,
    _FOOBAR_INVALID = -EINVAL,
} FoobarMode;
\end{ccode}

\section{完整实例}

\begin{ccode}
#include <gtk/gtk.h>

static void on_activate (GtkApplication *app) {
    // Create a new window
    GtkWidget *t_window = gtk_application_window_new (g_app);
    // Create a new button
    GtkWidget *t_button = gtk_button_new_with_label ("Hello, World!");
    // When the button is clicked, close the window passed as an argument
    g_signal_connect_swapped (g_button, "clicked", G_CALLBACK (gtk_window_close), window);
    gtk_window_set_child (GTK_WINDOW (t_window), g_button);
    gtk_window_present (GTK_WINDOW (t_window));
}

int main (int argc, char *argv[]) {
    // Create a new application
    GtkApplication *t_app = gtk_application_new ("com.example.GtkApplication", G_APPLICATION_FLAGS_NONE);
    g_signal_connect (t_app, "activate", G_CALLBACK (on_activate), NULL);
    return g_application_run (G_APPLICATION (t_app), argc, argv);
}
\end{ccode}
\chapter{格式化}

在一个项目中,必须保证代码风格一致性。除非这些代码是直接复制引用第三方的,这些第三方代码建议放在单独目录。同时,务必使用\cinline{.editorconfig}来保证编辑器能够正确处理代码风格。

\section{行长度}

\textbf{总述}

每一行代码字符数不超过 110。不同人对使用 80/100/110/120 持有不同的态度,但是这个争论毫无意义,请在 \deepin 的\cinline{C}项目中保持这个值。

\begin{DNote}
头文件保护可以无视该原则。
\end{DNote}

\section{字符编码}

\textbf{总述}

使用时必须使用 UTF-8 编码。除非在输出给用户直接观看的文本中,否则不要使用非 ASCII 字符;即使在输出文本,也应该慎重考虑对 tty 的支持,避免输出乱码。

\section{缩进}

\textbf{总述}

只使用空格,每次缩进 4 个空格。

\section{函数声明与定义}

\textbf{总述}

返回类型和函数名在同一行,参数也尽量放在同一行,如果放不下就对形参分行。需要注意如下细节:

\begin{itemize}
  \item 左大括号总在最后一个参数同一行的末尾处,不另起新行。
  \item 右大括号总是单独位于函数最后一行,或者与左大括号同一行。
  \item 左圆括号总是和函数名在同一行,函数名和左圆括号间永远没有空格。
  \item 右圆括号和左大括号间总是有一个空格
  \item 所有形参应尽可能对齐。
  \item 换行后的参数缺省缩进为 8 个空格,即 2 倍标准缩减。
\end{itemize}

\textbf{说明}

\begin{ccode}
void some_function(int foo) {
    int a;
}

void some_function(
        int foo,
        bool bar,
        char baz) {
    int a, b, c;
}
\end{ccode}

\section{条件表达式}

\textbf{总述}

单行 if 块应该被包含在 \{\} 中。\cinline{else}块通常应该从与闭合\}相同的行开始。不在圆括号内使用空格。

\textbf{说明}

\begin{ccode}
if (foobar) {
    find_and_waldo();
} else {
    dont_find_waldo();
}
\end{ccode}

\section{循环和开关选择语句}

\textbf{总述}

\cinline{switch}语句可以使用大括号分段,以表明 cases 之间不是连在一起的。在单语句循环里,括号可用可不用。空循环体应使用 \{\} 或 \cinline{continue},而不是一个简单的分号。

\textbf{说明}

\begin{ccode}
switch (var) {
case 0: {     // 不缩进
    ...       // 4 空格缩进
    break;
}
case 1: {
    ...
    break;
}
default: {
    assert(false);
}
}

for (int i = 0; i < num; ++i) {
  printf("I take it back\n");
}

while (condition) {
  // 反复循环直到条件失效。
}

for (int i = 0; i < num; ++i) {}  // 可 - 空循环体。

while (condition) continue;  // 可 - contunue 表明没有逻辑。
\end{ccode}

\section{指针}

\textbf{总述}

句点或箭头前后不要有空格。指针/地址操作符 (\*, \&) 之后不能有空格。

\textbf{说明}

\begin{ccode}
x = *p;
p = &x;
x = r.y;
x = r->y;

// 好,空格前置。
char *c;

// 可接受,空格后置。
char* c;

int x, *y;  // 不允许 - 在多重声明中不能使用 & 或 *
char * c;   // 差 - * 两边都有空格
\end{ccode}

\section{布尔表达式}

\textbf{总述}

如果一个布尔表达式超过标准行宽,断行方式要统一一下。

\textbf{说明}

可以考虑额外插入圆括号,合理使用的话对增强可读性是很有帮助的。

\begin{ccode}
    if (this_one_thing > this_other_thing
        && a_third_thing == a_fourth_thing
        && yet_another
        && last_one) {
    }
\end{ccode}

\section{函数返回值}

\textbf{总述}

不要在 \cinline{return} 表达式里加上非必须的圆括号。

\textbf{说明}

只有在写 \cinline{x = expr} 要加上括号的时候才在 \cinline{return expr;} 里使用括号。

\begin{ccode}
return result;  // 返回值很简单,没有圆括号。
// 可以用圆括号把复杂表达式圈起来,改善可读性。
return (some_long_condition &&
        another_condition);

return (value);                // 错误,毕竟您从来不会写 var = (value);
return(result);                // 错误,return 可不是函数!
\end{ccode}

\section{预处理指令}

\textbf{总述}

预处理指令不要缩进,从行首开始。

\textbf{说明}

即使预处理指令位于缩进代码块中,指令也应从行首开始。

\begin{ccode}
    if (lopsided_score) {
#if DISASTER_PENDING      // 正确 - 从行首开始
        drop_everything();
#if NOTIFY
        notify_client();
#endif
#endif
        back_to_normal();
    }
\end{ccode}

\section{留白}

\textbf{总述}

水平留白的使用根据在代码中的位置决定。永远不要在行尾添加没意义的留白。添加冗余的留白会给其他人编辑时造成额外负担。因此,行尾不要留空格。如果确定一行代码已经修改完毕,将多余的空格去掉; 或者在专门清理空格时去掉(尤其是在没有其他人在处理这件事的时候)。

垂直留白越少越好。这不仅仅是规则而是原则问题了:不在万不得已,不要使用空行。尤其是:两个函数定义之间的空行不要超过 2 行,函数体首尾不要留空行,函数体中也不要随意添加空行。基本原则是:同一屏可以显示的代码越多,越容易理解程序的控制流。当然,过于密集的代码块和过于疏松的代码块同样难看,这取决于你的判断。但通常是垂直留白越少越好。

\textbf{说明}

即使预处理指令位于缩进代码块中,指令也应从行首开始。

\begin{ccode}
void f(bool b) {  // 左大括号前总是有空格
int i = 0;  // 分号前不加空格

// 对于单行函数的实现,在大括号内加上空格
// 然后是函数实现
Foo(int b) : Bar(), baz_(b) {}  // 大括号里面是空的话,不加空格
void Reset() { baz_ = 0; }      // 用空格把大括号与实现分开

if (b) {            // if 条件语句和循环语句关键字后均有空格。
} else {            // else 前后有空格。
}
while (test) {}     // 圆括号内部不紧邻空格。

switch (i) {
case 1:             // switch case 的冒号前无空格。
case 2: break;      // 如果冒号有代码,加个空格。
\end{ccode}


\part{Qt代码风格}

\input{qt/flyleaf.tex}
\chapter{头文件}

通常每一个 \cppinline{.cpp} 文件都有一个对应的 \cppinline{.h} 文件. 也有一些常见例外, 如单元测试代码和只包含 \cppinline{main()} 函数的\cppinline{.cpp} 文件。

正确使用头文件可令代码在可读性、文件大小和性能上大为改观。

下面的规则将引导你规避使用头文件时的各种陷阱。

\section{Self-contained 头文件} \label{self-contained-headers}

\DBox {
	头文件应该能够自给自足(self-contained,也就是可以作为第一个头文件被引入),以 \cppinline{.h} 结尾。至于用来插入文本的文件,说到底它们并不是头文件,所以应以 \cppinline{.inc} 结尾。不允许分离出 \cppinline{-inl.h} 头文件的做法。
}

所有头文件要能够自给自足。换言之,用户和重构工具不需要为特别场合而包含额外的头文件。详言之,一个头文件要有\DFullRef{pragma-once-guard},统统包含它所需要的其它头文件,也不要求定义任何特别 symbols。

不过有一个例外,即一个文件并不是 self-contained 的,而是作为文本插入到代码某处。或者,文件内容实际上是其它头文件的特定平台(platform-specific)扩展部分。这些文件就要用\cppinline{.inc} 文件扩展名。

如果 \cppinline{.h} 文件声明了一个模板或内联函数,同时也在该文件加以定义。凡是有用到这些的 \cppinline{.cpp} 文件,就得统统包含该头文件,否则程序可能会在构建中链接失败。不要把这些定义放到分离的 \cppinline{-inl.h} 文件里(译者注:过去该规范曾提倡把定义放到 -inl.h 里过)。

有个例外:如果某函数模板为所有相关模板参数显式实例化,或本身就是某类的一个私有成员,那么它就只能定义在实例化该模板的 \cppinline{.cpp} 文件里。

\section{#define 保护} \label{pragma-once-guard}

\DBox{
  所有头文件都应该使用 \cppinline{#define} 来防止头文件被多重包含,命名格式当是:\cppinline{<PROJECT>_<PATH>_<FILE>_H_}。
}

为保证唯一性,头文件的命名应该基于所在项目源代码树的全路径。例如,项目 foo 中的头文件 foo/src/bar/baz.h 可按如下方式保护:

\begin{cppcode}
  #ifndef FOO_BAR_BAZ_H_
  #define FOO_BAR_BAZ_H_
  ...
  #endif // FOO_BAR_BAZ_H_
\end{cppcode}

\section{前置声明} \label{forward-declarations}

\DBox {
	尽可能地避免使用前置声明。使用 \cppinline{#include} 包含需要的头文件即可。
}

\textbf{定义:}

所谓「前置声明」(forward declaration)是类、函数和模板的纯粹声明,没伴随着其定义.

\textbf{优点:}

\begin{itemize}
	\item 前置声明能够节省编译时间,多余的 \cppinline{#include} 会迫使编译器展开更多的文件,处理更多的输入。
	\item 前置声明能够节省不必要的重新编译的时间。 \cppinline{#include} 使代码因为头文件中无关的改动而被重新编译多次。
\end{itemize}

\textbf{缺点:}

\begin{itemize}
	\item 前置声明隐藏了依赖关系,头文件改动时,用户的代码会跳过必要的重新编译过程。
	\item 前置声明可能会被库的后续更改所破坏。前置声明函数或模板有时会妨碍头文件开发者变动其 API。例如扩大形参类型,加个自带默认参数的模板形参等等。
	\item 前置声明来自命名空间 \cppinline{std::} 的 symbol 时,其行为未定义。
	\item 很难判断什么时候该用前置声明,什么时候该用 \cppinline{#include} 。极端情况下,用前置声明代替 \cppinline{#include} 甚至都会暗暗地改变代码的含义:

\begin{cppcode}
// b.h:
struct B {};
struct D : B {};

// good_user.cpp:
#include "b.h"
void f(B*);
void f(void*);
void test(D* x) { f(x); }  // calls f(B*)
\end{cppcode}

	      如果 \cppinline{#include} 被 \cppinline{B} 和 \cppinline{D} 的前置声明替代, \cppinline{test()} 就会调用 \cppinline{f(void*)} 。	\item 前置声明了不少来自头文件的 symbol 时,就会比单单一行的 \cppinline{include} 冗长。
	\item 仅仅为了能前置声明而重构代码(比如用指针成员代替对象成员)会使代码变得更慢更复杂.
\end{itemize}

\textbf{结论:}

\begin{itemize}
	\item 尽量避免前置声明那些定义在其他项目中的实体。
	\item 函数:总是使用 \cppinline{#include} 。
	\item 类模板:优先使用 \cppinline{#include} 。
\end{itemize}

至于什么时候包含头文件,参见 \DFullRef{name-and-order-of-includes} 。


\section{内联函数} \label{inline-functions}

\DBox {
	只有当函数只有 10 行甚至更少时才将其定义为内联函数。
}

\textbf{定义:}

当函数被声明为内联函数之后, 编译器会将其内联展开, 而不是按通常的函数调用机制进行调用。

\textbf{优点:}

只要内联的函数体较小, 内联该函数可以令目标代码更加高效. 对于存取函数以及其它函数体比较短, 性能关键的函数, 鼓励使用内联.

\textbf{缺点:}

滥用内联将导致程序变得更慢. 内联可能使目标代码量或增或减, 这取决于内联函数的大小. 内联非常短小的存取函数通常会减少代码大小,但内联一个相当大的函数将戏剧性的增加代码大小. 现代处理器由于更好的利用了指令缓存, 小巧的代码往往执行更快。

\textbf{结论:}

一个较为合理的经验准则是, 不要内联超过 10 行的函数. 谨慎对待析构函数, 析构函数往往比其表面看起来要更长,因为有隐含的成员和基类析构函数被调用!

另一个实用的经验准则: 内联那些包含循环或 \cppinline{switch} 语句的函数常常是得不偿失 (除非在大多数情况下, 这些循环或 \cppinline{switch} 语句从不被执行).

有些函数即使声明为内联的也不一定会被编译器内联, 这点很重要; 比如虚函数和递归函数就不会被正常内联. 通常,递归函数不应该声明成内联函数.(YuleFox 注: 递归调用堆栈的展开并不像循环那么简单, 比如递归层数在编译时可能是未知的,大多数编译器都不支持内联递归函数). 虚函数内联的主要原因则是想把它的函数体放在类定义内, 为了图个方便, 抑或是当作文档描述其行为,比如精短的存取函数.

\section{ include 的路径及顺序} \label{name-and-order-of-includes}

\DBox{
	使用标准的头文件包含顺序可增强可读性, 避免隐藏依赖: 相关头文件, C 库, C++ 库, 其他库的 `.h`, 本项目内的 `.h`.
}

项目内头文件应按照项目源代码目录树结构排列, 避免使用 UNIX 特殊的快捷目录 \cppinline{.} (当前目录) 或 \cppinline{..} (上级目录)。

例如, \cppinline{google-awesome-project/src/base/logging.h} 应该按如下方式包含:

\begin{cppcode}
	#include "base/logging.h"
\end{cppcode}

又如, \cppinline{dir/foo.cpp} 或 \cppinline{dir/foo_test.cpp} 的主要作用是实现或测试 \cppinline{dir2/foo2.h}的功能, \cppinline{foo.cpp} 中包含头文件的次序如下:

\begin{enumerate}
	\item \cppinline{dir2/foo2.h} (优先位置, 详情如下)
	\item C 系统文件
	\item C++ 系统文件
	\item 其他库的 \cppinline{.h} 文件
	\item 本项目内 \cppinline{.h} 文件
\end{enumerate}

这种优先的顺序排序保证当 \cppinline{dir2/foo2.h} 遗漏某些必要的库时, \cppinline{dir/foo.cpp} 或 \cppinline{dir/foo_test.cpp} 的构建会立刻中止。因此这一条规则保证维护这些文件的人们首先看到构建中止的消息而不是维护其他包的人们。

\cppinline{dir/foo.cpp} 和 \cppinline{dir2/foo2.h} 通常位于同一目录下,如\cppinline{base/basictypes_unittest.cpp} 和 \cppinline[breaklines]{base/basictypes.h}, 但也可以放在不同目录下.

按字母顺序分别对每种类型的头文件进行二次排序是不错的主意。注意较老的代码可不符合这条规则,要在方便的时候改正它们。

您所依赖的符号 (symbols) 被哪些头文件所定义,您就应该包含(include)哪些头文件,`前置声明`(forward declarations) 情况除外。比如您要用到 \cppinline{bar.h} 中的某个符号, 哪怕您所包含的 \cppinline{foo.h} 已经包含了\cppinline{bar.h}, 也照样得包含 \cppinline{bar.h}, 除非 \cppinline{foo.h} 有明确说明它会自动向您提供 \cppinline{bar.h} 中的symbol. 不过,凡是 cc 文件所对应的「相关头文件」已经包含的,就不用再重复包含进其 cc 文件里面了,就像 \cppinline{foo.cpp}只包含 \cppinline{foo.h} 就够了,不用再管后者所包含的其它内容。

举例来说, \cppinline{google-awesome-project/src/foo/internal/fooserver.cpp} 的包含次序如下:

\begin{cppcode}
	#include "foo/public/fooserver.h" // 优先位置

	#include <sys/types.h>
	#include <unistd.h>

	#include <hash_map>
	#include <vector>

	#include "base/basictypes.h"
	#include "base/commandlineflags.h"
	#include "foo/public/bar.h"
\end{cppcode}

\textbf{例外:}

有时,平台特定(system-specific)代码需要条件编译(conditional includes),这些代码可以放到其它 includes 之后。当然,您的平台特定代码也要够简练且独立,比如:

\begin{cppcode}
	#include "foo/public/fooserver.h"

	#include "base/port.h"  // For LANG_CXX11.

	#ifdef LANG_CXX11
	#include <initializer_list>
	#endif  // LANG_CXX11
\end{cppcode}


\section{注解}

\subsection{译者 (YuleFox) 笔记}

\begin{itemize}
	\item  避免多重包含是学编程时最基本的要求;
	\item  前置声明是为了降低编译依赖,防止修改一个头文件引发多米诺效应;
	\item  内联函数的合理使用可提高代码执行效率;
	\item  \cppinline{-inl.h} 可提高代码可读性 (一般用不到吧:D);
	\item  标准化函数参数顺序可以提高可读性和易维护性 (对函数参数的堆栈空间有轻微影响, 我以前大多是相同类型放在一起);
	\item  包含文件的名称使用 \cppinline{.} 和 \cppinline{..} 虽然方便却易混乱, 使用比较完整的项目路径看上去很清晰, 很条理,包含文件的次序除了美观之外, 最重要的是可以减少隐藏依赖, 使每个头文件在 "最需要编译" (对应源文件处 :D) 的地方编译,有人提出库文件放在最后, 这样出错先是项目内的文件, 头文件都放在对应源文件的最前面, 这一点足以保证内部错误的及时发现了.
\end{itemize}

\subsection{译者(acgtyrant)笔记}

\begin{itemize}
	\item  原来还真有项目用 \cppinline{#include} 来插入文本,且其文件扩展名 \cppinline{.inc} 看上去也很科学。
	\item  Google 已经不再提倡 \cppinline{-inl.h} 用法。
	\item  注意,前置声明的类是不完全类型(incomplete type),我们只能定义指向该类型的指针或引用,或者声明(但不能定义)以不完全类型作为参数或者返回类型的函数。毕竟编译器不知道不完全类型的定义,我们不能创建其类的任何对象,也不能声明成类内部的数据成员。
	\item  类内部的函数一般会自动内联。所以某函数一旦不需要内联,其定义就不要再放在头文件里,而是放到对应的 \cppinline{.cpp} 文件里。这样可以保持头文件的类相当精炼,也很好地贯彻了声明与定义分离的原则。
	\item  在 \cppinline{#include} 中插入空行以分割相关头文件, C 库, C++ 库, 其他库的 \cppinline{.h} 和本项目内的 \cppinline{.h} 是个好习惯。
\end{itemize}


\chapter{作用域}

\section{命名空间} \label{qt-namespace}

\DBox{
鼓励在 \cppinline{.cpp} 文件内使用匿名命名空间或 \cppinline{static} 声明。使用具名的命名空间时,其名称可基于项目名或相对路径。
禁止使用 using 指示(using-directive)。禁止使用内联命名空间(inline namespace)。
}

\textbf{定义:}

命名空间将全局作用域细分为独立的,具名的作用域,可有效防止全局作用域的命名冲突。

\textbf{优点:}

虽然类已经提供了(可嵌套的)命名轴线 (YuleFox 注:将命名分割在不同类的作用域内), 命名空间在这基础上又封装了一层。

举例来说,两个不同项目的全局作用域都有一个类 \cppinline{Foo}, 这样在编译或运行时造成冲突。如果每个项目将代码置于不同命名空间中,
\cppinline{project1::Foo} 和 \cppinline{project2::Foo} 作为不同符号自然不会冲突。

内联命名空间会自动把内部的标识符放到外层作用域,比如:

% \begin{noindent}
\begin{cppcode}
namespace X {
inline namespace Y {
  void foo();
}  // namespace Y
}  // namespace X
\end{cppcode}
%\end{noindent}

\cppinline{X::Y::foo()} 与 \cppinline{X::foo()} 彼此可代替。内联命名空间主要用来保持跨版本的 ABI 兼容性。

\textbf{缺点:}

命名空间具有迷惑性,因为它们使得区分两个相同命名所指代的定义更加困难。

内联命名空间很容易令人迷惑,毕竟其内部的成员不再受其声明所在命名空间的限制。内联命名空间只在大型版本控制里有用。

有时候不得不多次引用某个定义在许多嵌套命名空间里的实体,使用完整的命名空间会导致代码的冗长。

在头文件中使用匿名空间导致违背 C++ 的唯一定义原则 (One Definition Rule (ODR))。

\textbf{结论:}

根据下文将要提到的策略合理使用命名空间。

\begin{itemize}
  \item 遵守 \DFullRef{qt-namespace-names} 中的规则。
  \item 像之前的几个例子中一样,在命名空间的最后注释出命名空间的名字。
  \item 用命名空间把文件包含,`gflags <https://gflags.github.io/gflags/>` 的声明/定义,以及类的前置声明以外的整个源文件封装起来,以区别于其它命名空间:

% \begin{noindent}
\begin{cppcode}
  // .h 文件
namespace mynamespace {

// 所有声明都置于命名空间中
// 注意不要使用缩进
class MyClass {
public:
...
void Foo();
};

} // namespace mynamespace
\end{cppcode}
%\end{noindent}

%\begin{noindent}
\begin{cppcode}
// .cpp 文件
namespace mynamespace {

// 函数定义都置于命名空间中
void MyClass::Foo() {
...
}

} // namespace mynamespace
\end{cppcode}
%\end{noindent}

更复杂的 \cppinline{.cpp} 文件包含更多,更复杂的细节,比如 gflags 或 using 声明。

%\begin{noindent}
\begin{cppcode}
#include "a.h"

DEFINE_FLAG(bool, someflag, false, "dummy flag");

namespace a {

...code for a...// 左对齐

} // namespace a
\end{cppcode}
% \end{noindent}

  \item 不要在命名空间 \cppinline{std} 内声明任何东西,包括标准库的类前置声明。在 \cppinline{std} 命名空间声明实体是未定义的行为,会导致如不可移植。声明标准库下的实体,需要包含对应的头文件。

  \item 不应该使用 \textit{using 指示} 引入整个命名空间的标识符号。

%\begin{noindent}
\begin{cppcode}
// 禁止 —— 污染命名空间
using namespace foo;
\end{cppcode}
% \end{noindent}

  \item  不要在头文件中使用 \textit{命名空间别名} 除非显式标记内部命名空间使用。因为任何在头文件中引入的命名空间都会成为公开 API 的一部分。

%\begin{noindent}
\begin{cppcode}
// 在 .cpp 中使用别名缩短常用的命名空间
namespace baz = ::foo::bar::baz;
\end{cppcode}

\begin{cppcode}
// 在 .h 中使用别名缩短常用的命名空间
namespace librarian {
namespace impl {  // 仅限内部使用
namespace sidetable = ::pipeline_diagnostics::sidetable;
}  // namespace impl

inline void my_inline_function() {
  // 限制在一个函数中的命名空间别名
  namespace baz = ::foo::bar::baz;
...
}
}  // namespace librarian
\end{cppcode}
% \end{noindent}

  \item  禁止用内联命名空间。
\end{itemize}

\section{匿名命名空间和静态变量} \label{unnamed-namespace-and-static-variables}

\DBox {
在 \cppinline{.cpp} 文件中定义一个不需要被外部引用的变量时,可以将它们放在匿名命名空间或声明为 \cppinline{static} 。但是不要在\cppinline{.h} 文件中这么做。
}

\textbf{定义:}

所有置于匿名命名空间的声明都具有内部链接性,函数和变量可以经由声明为 \cppinline{static} 拥有内部链接性,这意味着你在这个文件中声明的这些标识符都不能在另一个文件中被访问。即使两个文件声明了完全一样名字的标识符,它们所指向的实体实际上是完全不同的。

\textbf{结论:}

推荐、鼓励在 \cppinline{.cpp} 中对于不需要在其他地方引用的标识符使用内部链接性声明,但是不要在 \cppinline{.h} 中使用。

匿名命名空间的声明和具名的格式相同,在最后注释上 \cppinline{namespace} :

%\begin{noindent}
\begin{cppcode}
namespace {
...
}  // namespace
\end{cppcode}
% \end{noindent}

\section{非成员函数、静态成员函数和全局函数} \label{nonmember-static-member-and-global-functions}

\DBox{
使用静态成员函数或命名空间内的非成员函数,尽量不要用裸的全局函数。将一系列函数直接置于命名空间中,不要用类的静态方法模拟出命名空间的效果,类的静态方法应当和类的实例或静态数据紧密相关。
}

\textbf{优点:}

某些情况下,非成员函数和静态成员函数是非常有用的,将非成员函数放在命名空间内可避免污染全局作用域。

\textbf{缺点:}

将非成员函数和静态成员函数作为新类的成员或许更有意义,当它们需要访问外部资源或具有重要的依赖关系时更是如此。

\textbf{结论:}

有时,把函数的定义同类的实例脱钩是有益的,甚至是必要的。这样的函数可以被定义成静态成员,或是非成员函数。

非成员函数不应依赖于外部变量,应尽量置于某个命名空间内。相比单纯为了封装若干不共享任何静态数据的静态成员函数而创建类,不如使用\DFullRef{qt-namespace}。举例而言,对于头文件 \cppinline{myproject/foo_bar.h}  , 应当使用

\begin{cppcode}
  namespace myproject {
  namespace foo_bar {
  void Function1();
  void Function2();
  }  // namespace foo_bar
  }  // namespace myproject
\end{cppcode}

而非

\begin{cppcode}
  namespace myproject {
      class FooBar {
          public:
          static void Function1();
          static void Function2();
        };
    }  // namespace myproject
\end{cppcode}

定义在同一编译单元的函数,被其他编译单元直接调用可能会引入不必要的耦合和链接时依赖; 静态成员函数对此尤其敏感。可以考虑提取到新类中,或者将函数置于独立库的命名空间内。

如果你必须定义非成员函数,又只是在 \cppinline{.cpp} 文件中使用它,可使用匿名 \DFullRef{qt-namespace} 或 \cppinline{static} 链接关键字 (如 \cppinline{static int Foo() {...}}) 限定其作用域。

\section{局部变量} \label{local-variables}

\DBox{
将函数变量尽可能置于最小作用域内,并在变量声明时进行初始化。
}

C++ 允许在函数的任何位置声明变量。我们提倡在尽可能小的作用域中声明变量,离第一次使用越近越好。这使得代码浏览者更容易定位变量声明的位置,了解变量的类型和初始值。特别是,应使用初始化的方式替代声明再赋值,比如:

\begin{cppcode}
  int i;
  i = f(); // 坏——初始化和声明分离

  int j = g(); // 好——初始化时声明

  vector<int> v;
  v.push_back(1); // 用花括号初始化更好
  v.push_back(2);

  vector<int> v = {1, 2}; // 好——v 一开始就初始化
\end{cppcode}

属于 \cppinline{if}, \cppinline{while} 和 \cppinline{for} 语句的变量应当在这些语句中正常地声明,这样子这些变量的作用域就被限制在这些语句中了,举例而言:

\begin{cppcode}
  while (const char* p = strchr(str, '/')) str = p + 1;
\end{cppcode}

\begin{DWarn}
有一个例外,如果变量是一个对象,每次进入作用域都要调用其构造函数,每次退出作用域都要调用其析构函数。这会导致效率降低。
\end{DWarn}

\begin{cppcode}
  // 低效的实现
  for (int i = 0; i < 1000000; ++i) {
      Foo f;                  // 构造函数和析构函数分别调用 1000000 次!
      f.DoSomething(i);
    }
\end{cppcode}

在循环作用域外面声明这类变量要高效的多:

\begin{cppcode}
  Foo f;                      // 构造函数和析构函数只调用 1 次
  for (int i = 0; i < 1000000; ++i) {
      f.DoSomething(i);
    }
\end{cppcode}


\section{静态和全局变量} \label{static-and-global-variables}

\DBox{
禁止定义静态储存周期非 POD 变量,禁止使用含有副作用的函数初始化 POD 全局变量,因为多编译单元中的静态变量执行时的构造和析构顺序是未明确的,这将导致代码的不可移植。
}

禁止使用类的\href{http://zh.cppreference.com/w/cpp/language/storage_duration#.E5.AD.98.E5.82.A8.E6.9C.9F}{静态储存周期}变量:由于构造和析构函数调用顺序的不确定性,它们会导致难以发现的 bug。不过 \cppinline{constexpr}变量除外,毕竟它们又不涉及动态初始化或析构。

静态生存周期的对象,即包括了全局变量,静态变量,静态类成员变量和函数静态变量,都必须是原生数据类型 (POD : Plain OldData): 即 int, char 和 float, 以及 POD 类型的指针、数组和结构体。

静态变量的构造函数、析构函数和初始化的顺序在 C++ 中是只有部分明确的,甚至随着构建变化而变化,导致难以发现的 bug。所以除了禁用类类型的全局变量,我们也不允许用函数返回值来初始化 POD 变量,除非该函数(比如 \cppinline{getenv()} 或\cppinline{getpid()})不涉及任何全局变量。函数作用域里的静态变量除外,毕竟它的初始化顺序是有明确定义的,而且只会在指令执行到它的声明那里才会发生。

\begin{DNote}
  Xris 译注:

  同一个编译单元内是明确的,静态初始化优先于动态初始化,初始化顺序按照声明顺序进行,销毁则逆序。不同的编译单元之间初始化和销毁顺序属于未明确行为
  (unspecified behaviour)。

\end{DNote}

同理,全局和静态变量在程序中断时会被析构,无论所谓中断是从 \cppinline{main()} 返回还是对 \cppinline{exit()} 的调用。析构顺序正好与构造函数调用的顺序相反。但既然构造顺序未定义,那么析构顺序当然也就不定了。比如,在程序结束时某静态变量已经被析构了,但代码还在跑——比如其它线程——并试图访问它且失败;再比如,一个静态 string 变量也许会在一个引用了前者的其它变量析构之前被析构掉。

改善以上析构问题的办法之一是用 \cppinline{quick_exit()} 来代替 \cppinline{exit()}并中断程序。它们的不同之处是前者不会执行任何析构,也不会执行 \cppinline{atexit()} 所绑定的任何 handlers。如果您想在执行 \cppinline{quick_exit()} 来中断时执行某 handler(比如刷新 log),您可以把它绑定到\cppinline{_at_quick_exit()}. 如果您想在 \cppinline{exit()} 和 \cppinline{quick_exit()} 都用上该 handler,都绑定上去。

综上所述,我们只允许 POD 类型的静态变量,即完全禁用 \cppinline{vector} (使用 C 数组替代) 和 \cppinline{string} (使用\cppinline{const char []})。

如果您确实需要一个 \cppinline{class} 类型的静态或全局变量,可以考虑在 \cppinline{main()} 函数或 \cppinline{pthread_once()}
内初始化一个指针且永不回收。注意只能用 raw 指针,别用智能指针,毕竟后者的析构函数涉及到上文指出的不定顺序问题。

\begin{DNote}
  Yang.Y 译注:

  上文提及的静态变量泛指静态生存周期的对象,包括:全局变量,静态变量,静态类成员变量,以及函数静态变量。
\end{DNote}

\section{注解}

\subsection{ 译者 (YuleFox) 笔记}

\begin{itemize}
  \item \cppinline{cpp} 中的匿名命名空间可避免命名冲突,限定作用域,避免直接使用 \cppinline{using} 关键字污染命名空间。
  \item 嵌套类符合局部使用原则,只是不能在其他头文件中前置声明,尽量不要 \cppinline{public}。
  \item 尽量不用全局函数和全局变量,考虑作用域和命名空间限制,尽量单独形成编译单元。
  \item 多线程中的全局变量 (含静态成员变量) 不要使用 \cppinline{class} 类型 (含 STL 容器), 避免不明确行为导致的 bug。
  \item 作用域的使用,除了考虑名称污染,可读性之外,主要是为降低耦合,提高编译/执行效率。
\end{itemize}

\subsection{  译者(acgtyrant)笔记 }

\begin{itemize}
  \item 注意「using 指示(using-directive)」和「using 声明(using-declaration)」的区别。
  \item 匿名命名空间说白了就是文件作用域,就像 C static 声明的作用域一样,后者已经被 C++ 标准提倡弃用。
  \item 局部变量在声明的同时进行显式值初始化,比起隐式初始化再赋值的两步过程要高效,同时也贯彻了计算机体系结构重要的概念「局部性(locality)」。
  \item 注意别在循环犯大量构造和析构的低级错误。
\end{itemize}
\input{qt/classes.tex}
\chapter{函数}

\section{参数顺序}

\textbf{总述}

函数的参数顺序为: 输入参数在先, 后跟输出参数。

\textbf{说明}

C/C++ 中的函数参数或者是函数的输入, 或者是函数的输出, 或兼而有之。 输入参数通常是值参或 \cppinline{const} 引用, 输出参数或输入/输出参数则一般为非 \cppinline{const} 指针。 在排列参数顺序时, 将所有的输入参数置于输出参数之前。 特别要注意, 在加入新参数时不要因为它们是新参数就置于参数列表最后, 而是仍然要按照前述的规则, 即将新的输入参数也置于输出参数之前。

这并非一个硬性规定。 输入/输出参数 (通常是类或结构体) 让这个问题变得复杂。 并且, 有时候为了其他函数保持一致, 你可能不得不有所变通。

\section{编写简短函数}

\textbf{总述}

我们倾向于编写简短, 凝练的函数。

\textbf{说明}

我们承认长函数有时是合理的, 因此并不硬性限制函数的长度。 如果函数超过 40 行, 可以思索一下能不能在不影响程序结构的前提下对其进行分割。

即使一个长函数现在工作的非常好, 一旦有人对其修改, 有可能出现新的问题, 甚至导致难以发现的 bug。 使函数尽量简短, 以便于他人阅读和修改代码。

在处理代码时, 你可能会发现复杂的长函数。 不要害怕修改现有代码: 如果证实这些代码使用 / 调试起来很困难, 或者你只需要使用其中的一小段代码, 考虑将其分割为更加简短并易于管理的若干函数。

\section{引用参数}

\textbf{总述}

所有按引用传递的参数必须加上 \cppinline{const}。

\textbf{定义}

在 C 语言中, 如果函数需要修改变量的值, 参数必须为指针, 如 \cppinline{int foo(int *pval)}。 在 C++ 中, 函数还可以声明为引用参数: \cppinline{int foo(int &val)}。

\textbf{优点}

定义引用参数可以防止出现 \cppinline{(*pval)++} 这样丑陋的代码。 引用参数对于拷贝构造函数这样的应用也是必需的。 同时也更明确地不接受空指针。

\textbf{缺点}

容易引起误解, 因为引用在语法上是值变量却拥有指针的语义。

\textbf{结论}

函数参数列表中, 所有引用参数都必须是 \cppinline{const}:

\begin{cppcode}
  void Foo(const string &in, string *out);
\end{cppcode}

事实上这在 Google Code 是一个硬性约定: 输入参数是值参或 \cppinline{const} 引用, 输出参数为指针。 输入参数可以是 \cppinline{const} 指针, 但决不能是非 \cppinline{const} 的引用参数, 除非特殊要求, 比如 \cppinline{swap()}。

有时候, 在输入形参中用 \cppinline{const T*} 指针比 \cppinline{const T&} 更明智。 比如:

* 可能会传递空指针。

* 函数要把指针或对地址的引用赋值给输入形参。

总而言之, 大多时候输入形参往往是 \cppinline{const T&}。 若用 \cppinline{const T*} 则说明输入另有处理。 所以若要使用 \cppinline{const T*}, 则应给出相应的理由, 否则会使得读者感到迷惑。

\section{函数重载} \label{function-overloading}

\textbf{总述}

若要使用函数重载, 则必须能让读者一看调用点就胸有成竹, 而不用花心思猜测调用的重载函数到底是哪一种。 这一规则也适用于构造函数。

\textbf{定义}

你可以编写一个参数类型为 \cppinline{const string&} 的函数, 然后用另一个参数类型为 \cppinline{const char*} 的函数对其进行重载:


\begin{cppcode}
  class MyClass {
      public:
      void Analyze(const string &text);
      void Analyze(const char *text, size_t textlen);
    };
\end{cppcode}

\textbf{优点}

通过重载参数不同的同名函数, 可以令代码更加直观。 模板化代码需要重载, 这同时也能为使用者带来便利。

\textbf{缺点}

如果函数单靠不同的参数类型而重载 (acgtyrant 注:这意味着参数数量不变), 读者就得十分熟悉 C++ 五花八门的匹配规则, 以了解匹配过程具体到底如何。 另外, 如果派生类只重载了某个函数的部分变体, 继承语义就容易令人困惑。

\textbf{结论}
% TODO: change \textbf{列表初始化格式} ->  \DFullRef{braced-initializer-list} in formatting.tex not work now.
如果打算重载一个函数, 可以试试改在函数名里加上参数信息。 例如, 用 \cppinline{AppendString()} 和 \cppinline{AppendInt()} 等, 而不是一口气重载多个 \cppinline{Append()}。 如果重载函数的目的是为了支持不同数量的同一类型参数, 则优先考虑使用 \cppinline{std::vector} 以便使用者可以用\textbf{列表初始化格式}指定参数。

\section{缺省参数}

\textbf{总述}

只允许在非虚函数中使用缺省参数, 且必须保证缺省参数的值始终一致。 缺省参数与 \DFullRef{function-overloading} 遵循同样的规则。 一般情况下建议使用函数重载, 尤其是在缺省函数带来的可读性提升不能弥补下文中所提到的缺点的情况下。

\textbf{优点}

有些函数一般情况下使用默认参数, 但有时需要又使用非默认的参数。 缺省参数为这样的情形提供了便利, 使程序员不需要为了极少的例外情况编写大量的函数。 和函数重载相比, 缺省参数的语法更简洁明了, 减少了大量的样板代码, 也更好地区别了 "必要参数" 和 "可选参数"。

\textbf{缺点}

缺省参数实际上是函数重载语义的另一种实现方式, 因此所有 \DFullRef{function-overloading} 也都适用于缺省参数。

虚函数调用的缺省参数取决于目标对象的静态类型, 此时无法保证给定函数的所有重载声明的都是同样的缺省参数。

缺省参数是在每个调用点都要进行重新求值的, 这会造成生成的代码迅速膨胀。 作为读者, 一般来说也更希望缺省的参数在声明时就已经被固定了, 而不是在每次调用时都可能会有不同的取值。

缺省参数会干扰函数指针, 导致函数签名与调用点的签名不一致。 而函数重载不会导致这样的问题。

\textbf{结论}

对于虚函数, 不允许使用缺省参数, 因为在虚函数中缺省参数不一定能正常工作。 如果在每个调用点缺省参数的值都有可能不同, 在这种情况下缺省函数也不允许使用。 (例如, 不要写像 \cppinline{void f(int n = counter++);} 这样的代码。)

在其他情况下, 如果缺省参数对可读性的提升远远超过了以上提及的缺点的话, 可以使用缺省参数。 如果仍有疑惑, 就使用函数重载。

\section{函数返回类型后置语法}

\textbf{总述}

只有在常规写法 (返回类型前置) 不便于书写或不便于阅读时使用返回类型后置语法。

\textbf{定义}

C++ 现在允许两种不同的函数声明方式。 以往的写法是将返回类型置于函数名之前。 例如:

\begin{cppcode}
int foo(int x);
\end{cppcode}

C++11 引入了这一新的形式。 现在可以在函数名前使用 \cppinline{auto} 关键字, 在参数列表之后后置返回类型。 例如:

\begin{cppcode}
auto foo(int x) -> int;
\end{cppcode}

后置返回类型为函数作用域。 对于像 \cppinline{int} 这样简单的类型, 两种写法没有区别。 但对于复杂的情况, 例如类域中的类型声明或者以函数参数的形式书写的类型, 写法的不同会造成区别。

\textbf{优点}
% TODO: change \textbf{Lambda 表达式} -> \DFullRef{lambda-expressions} in formatting.tex not work now.
后置返回类型是显式地指定\textbf{Lambda 表达式}的返回值的唯一方式。 某些情况下, 编译器可以自动推导出 Lambda 表达式的返回类型, 但并不是在所有的情况下都能实现。 即使编译器能够自动推导, 显式地指定返回类型也能让读者更明了。

有时在已经出现了的函数参数列表之后指定返回类型, 能够让书写更简单, 也更易读, 尤其是在返回类型依赖于模板参数时。 例如:

\begin{cppcode}
  template <class T, class U> auto add(T t, U u) -> decltype(t + u);
\end{cppcode}

对比下面的例子:

\begin{cppcode}
  template <class T, class U> decltype(declval<T&>() + declval<U&>()) add(T t, U u);
\end{cppcode}

\textbf{缺点}

后置返回类型相对来说是非常新的语法, 而且在 C 和 Java 中都没有相似的写法, 因此可能对读者来说比较陌生。

在已有的代码中有大量的函数声明, 你不可能把它们都用新的语法重写一遍。 因此实际的做法只能是使用旧的语法或者新旧混用。 在这种情况下, 只使用一种版本是相对来说更规整的形式。

\textbf{结论}

在大部分情况下, 应当继续使用以往的函数声明写法, 即将返回类型置于函数名前。 只有在必需的时候 (如 Lambda 表达式) 或者使用后置语法能够简化书写并且提高易读性的时候才使用新的返回类型后置语法。 但是后一种情况一般来说是很少见的, 大部分时候都出现在相当复杂的模板代码中, 而多数情况下不鼓励写这样 \DFullRef{qt-template-metaprogramming}。

\input{qt/magic.tex}
% TODO(iceyer): use Qt smart pointer or STL?
\section{预处理宏} \label{qt-preprocessor-macros}

\begin{DNote}
  使用宏时要非常谨慎,尽量以内联函数,枚举和常量代替之。
\end{DNote}

宏意味着你和编译器看到的代码是不同的。这可能会导致异常行为,尤其因为宏具有全局作用域。

值得庆幸的是,C++ 中,宏不像在 C 中那么必不可少。以往用宏展开性能关键的代码,现在可以用内联函数替代。用宏表示常量可被 \cppinline{const} 变量代替。用宏 "缩写" 长变量名可被引用代替。用宏进行条件编译... 这个,千万别这么做,会令测试更加痛苦 ( \cppinline{#define} 防止头文件重包含当然是个特例).

宏可以做一些其他技术无法实现的事情,在一些代码库 (尤其是底层库中) 可以看到宏的某些特性 (如用 \cppinline{#} 字符串化,用 \cppinline{##} 连接等等). 但在使用前,仔细考虑一下能不能不使用宏达到同样的目的。

下面给出的用法模式可以避免使用宏带来的问题; 如果你要宏,尽可能遵守:

\begin{itemize}
  \item 不要在 \cppinline{.h} 文件中定义宏;
  \item 在马上要使用时才进行 \cppinline{#define}, 使用后要立即 \cppinline{#undef};
  \item 不要只是对已经存在的宏使用 \cppinline{#undef},选择一个不会冲突的名称;
  \item 不要试图使用展开后会导致 C++ 构造不稳定的宏,不然也至少要附上文档说明其行为;
  \item 不要用 \cppinline{##} 处理函数,类和变量的名字。
\end{itemize}

\section{模板编程} \label{qt-template-metaprogramming}

\begin{DNote}
不要使用复杂的模板编程。
\end{DNote}

\textbf{定义}

模板编程指的是利用 c++ 模板实例化机制是图灵完备性,可以被用来实现编译时刻的类型判断的一系列编程技巧。

\textbf{优点}

模板编程能够实现非常灵活的类型安全的接口和极好的性能,一些常见的工具比如 Google Test, std::tuple, std::function 和 Boost.Spirit. 这些工具如果没有模板是实现不了的。

\textbf{缺点}

\begin{itemize}
  \item 模板编程所使用的技巧对于使用 c++ 不是很熟练的人是比较晦涩,难懂的。在复杂的地方使用模板的代码让人更不容易读懂,并且 debug 和 维护起来都很麻烦。
  \item 模板编程经常会导致编译出错的信息非常不友好:在代码出错的时候,即使这个接口非常的简单,模板内部复杂的实现细节也会在出错信息显示。导致这个编译出错信息看起来非常难以理解。
  \item 大量的使用模板编程接口会让重构工具 (Visual Assist X, Refactor for C++ 等等) 更难发挥用途。首先模板的代码会在很多上下文里面扩展开来,所以很难确认重构对所有的这些展开的代码有用,其次有些重构工具只对已经做过模板类型替换的代码的 AST 有用。因此重构工具对这些模板实现的原始代码并不有效,很难找出哪些需要重构。
\end{itemize}

\textbf{结论}

\begin{itemize}
  \item 模板编程有时候能够实现更简洁更易用的接口,但是更多的时候却适得其反。因此模板编程最好只用在少量的基础组件,基础数据结构上,因为模板带来的额外的维护成本会被大量的使用给分担掉。
  \item 在使用模板编程或者其他复杂的模板技巧的时候,你一定要再三考虑一下。考虑一下你们团队成员的平均水平是否能够读懂并且能够维护你写的模板代码。或者一个非 c++ 程序员和一些只是在出错的时候偶尔看一下代码的人能够读懂这些错误信息或者能够跟踪函数的调用流程。如果你使用递归的模板实例化,或者类型列表,或者元函数,又或者表达式模板,或者依赖 SFINAE, 或者 sizeof 的 trick 手段来检查函数是否重载,那么这说明你模板用的太多了,这些模板太复杂了,我们不推荐使用。
  \item 如果你使用模板编程,你必须考虑尽可能的把复杂度最小化,并且尽量不要让模板对外暴露。你最好只在实现里面使用模板,然后给用户暴露的接口里面并不使用模板,这样能提高你的接口的可读性。并且你应该在这些使用模板的代码上写尽可能详细的注释。你的注释里面应该详细的包含这些代码是怎么用的,这些模板生成出来的代码大概是什么样子的。还需要额外注意在用户错误使用你的模板代码的时候需要输出更人性化的出错信息。因为这些出错信息也是你的接口的一部分,所以你的代码必须调整到这些错误信息在用户看起来应该是非常容易理解,并且用户很容易知道如何修改这些错误。
\end{itemize}
\chapter{命名约定}

最重要的一致性规则是命名管理。命名的风格能让我们在不需要去查找类型声明的条件下快速地了解某个名字代表的含义: 类型,变量,函数,常量,宏,等等,甚至 我们大脑中的模式匹配引擎非常依赖这些命名规则。\textbf{命名规则具有一定随意性,但相比按个人喜好命名,一致性更重要,所以无论你认为它们是否重要,规则总归是规则。}

\section{通用命名规则} \label{general-naming-rules}

\textbf{总述}

函数命名,变量命名,文件命名要有描述性; 少用缩写。

\textbf{说明}

尽可能使用描述性的命名,别心疼空间,毕竟相比之下让代码易于新读者理解更重要。不要用只有项目开发者能理解的缩写,也不要通过砍掉几个字母来缩写单词.

\begin{cppcode}
  int priceCountReader;     // 无缩写
  int numErrors;            // "num" 是一个常见的写法
  int numDnsConnections;    // 人人都知道 "DNS" 是什么
\end{cppcode}

\begin{cppcode}
  int n;                     // 毫无意义.
  int nerr;                  // 含糊不清的缩写.
  int nCompConns;            // 含糊不清的缩写.
  int wgcConnections;        // 只有贵团队知道是什么意思.
  int pcReader;              // "pc" 有太多可能的解释了.
  int cstmrID;               // 删减了若干字母.
\end{cppcode}

注意,一些特定的广为人知的缩写是允许的,例如用 \cppinline{i} 表示迭代变量和用 \cppinline{T} 表示模板参数。

\begin{DWarn}
  在D-Pointer风格中,\cppinline{d_ptr,q_ptr,dd_ptr,qq_ptr}都是保留的名称。
\end{DWarn}

模板参数的命名应当遵循对应的分类: 类型模板参数应当遵循 \DFullRef{type-names} 的规则,而非类型模板应当遵循  \DFullRef{variable-names} 的规则.

\section{文件命名}

\textbf{总述}

文件名要全部小写,可以包含下划线 (\cppinline{_}) 。如果存在无法\cppinline{_}的情况,可以考虑使用连字符\cppinline{-},否则不允许例外。

\begin{DWarn}
  Qt默认情况下不使用任何连接符合,这使得文件名非常难以看懂,我们不接受这种风格。deepin的Qt项目统一使用下划线(\cppinline{_})作为文件名连接符合。
\end{DWarn}

\textbf{说明}

可接受的文件命名示例:

\begin{cppcode}
  my_useful_class.cpp
  myusefulclass_test.cpp // \cppinline{_unittest} 和 \cppinline{_regtest} 已弃用.
\end{cppcode}

不接受的文件命名示例:

\begin{cppcode}
  my-useful-class.cpp  // 不接受,除非无法使用_
  myusefulclass.cpp    // 不接受,难以看懂
\end{cppcode}

C++ 文件要以 \cppinline{.cpp} 结尾,头文件以 \cppinline{.h} 结尾。专门插入文本的文件则以 \cppinline{.inc} 结尾,参见 \DFullRef{self-contained-headers}。

不要使用已经存在于 \cppinline{/usr/include} 下的文件名 (Yang.Y 注: 即编译器搜索系统头文件的路径),如 \cppinline{db.h}。

通常应尽量让文件名更加明确。\cppinline{http_server_logs.h} 就比 \cppinline{logs.h} 要好。定义类时文件名一般成对出现,如 \cppinline{foo_bar.h} 和 \cppinline{foo_bar.cpp},对应于类 \cppinline{FooBar}。

内联函数必须放在 \cppinline{.h} 文件中。如果内联函数比较短,就直接放在 \cppinline{.h} 中.

\section{类型命名} \label{type-names}

\textbf{总述}

类型名称的每个单词首字母均大写,不包含下划线: \cppinline{MyExcitingClass},\cppinline{MyExcitingEnum}。

\textbf{说明}

所有类型命名 —— 类,结构体,类型定义 (\cppinline{typedef}),枚举,类型模板参数 —— 均使用相同约定,即以大写字母开始,每个单词首字母均大写,不包含下划线。例如:

\begin{cppcode}
  // 类和结构体
  class UrlTable { ...
  class UrlTableTester { ...
  struct UrlTableProperties { ...

  // 类型定义
  typedef hash_map<UrlTableProperties *, string> PropertiesMap;

  // using 别名
  using PropertiesMap = hash_map<UrlTableProperties *, string>;

  // 枚举
  enum UrlTableErrors { ...
\end{cppcode}

\section{变量命名} \label{variable-names}

\textbf{总述}

\begin{DWarn}
  变量 (包括函数参数) 和数据成员名一律使用驼峰命名。

  在D-Pointer的Private类中,成员变量不加任何修饰。

  在一般的类中,使用\cppinline{m_}开头来标记成员变量。
\end{DWarn}

\textbf{说明}

\subsection{普通变量命名}

举例:

\begin{cppcode}
  QString tableName;   // 接受,驼峰命名

  QString table_name;  // 不接受 - 用下划线.
  QString tablename;   // 不接受 - 全小写.
\end{cppcode}

\subsection{类数据成员}

不管是静态的还是非静态的,类数据成员都可以和普通变量一样,但是需要使用\cppinline{m_}前缀来修饰。

\begin{DWarn}
  为了实现较好的封装,Qt中大量使用D-Pointer技术,在这种情况下,一般通过\cppinline{d->localValue}的方式访问Private类的变量,这时候就不需要使用\cppinline{m_}来修饰成员变量。
  对于非D-Pointer的Private类,使用\cppinline{m_}前缀来修饰成员变量。
\end{DWarn}


\begin{cppcode}
  class TableInfo {
      ...
      private:
      QString m_tableName;               // 好
      static Pool<TableInfo>* m_pool;    // 好
    };

  class TableInfoPrivate {
      ...
      public:
      QString tableName;               // 好,Private类不需要任何修饰
      static Pool<TableInfo>* pool;    // 好,Private类不需要任何修饰
    };
\end{cppcode}

\subsection{结构体变量}

不管是静态的还是非静态的,结构体数据成员都可以和普通变量一样,不用像类那样接下划线:

\begin{cppcode}
  struct TableInfoData {
      QString tableName;               // 好,命名风格和Private保持一致
      static Pool<TableInfo>* pool;    // 好,命名风格和Private保持一致
    }
\end{cppcode}

结构体与类的使用讨论,参考 \DFullRef{structs-vs-classes}。

\section{常量命名} \label{constant-names}

\textbf{总述}

声明为 \cppinline{constexpr} 或 \cppinline{const} 的变量,或在程序运行期间其值始终保持不变的,命名时以 "k" 开头,大小写混合。例如:

\begin{cppcode}
  const int kDaysInAWeek = 7;
\end{cppcode}

\textbf{说明}

所有具有静态存储类型的变量 (例如静态变量或全局变量,参见 \href{http://en.cppreference.com/w/cpp/language/storage_duration#Storage_duration}{存储类型}) 都应当以此方式命名。对于其他存储类型的变量,如自动变量等,这条规则是可选的。如果不采用这条规则,就按照一般的变量命名规则。

\section{函数命名} \label{function-names}

\textbf{总述}

常规函数使用大小写混合,取值和设值函数则要求与变量名匹配: \cppinline{myExcitingFunction()},\cppinline{myExcitingMethod()},\cppinline{my_exciting_member_variable()},\cppinline{set_my_exciting_member_variable()}。

\textbf{说明}

一般来说,函数名首字母小写,每个单词首字母大写 (即 "驼峰变量名" 或 "帕斯卡变量名"),没有下划线。对于首字母缩写的单词,更倾向于将它们视作一个单词进行首字母大写 (例如,写作 \cppinline{startRpc()} 而非 \cppinline{startRPC()})。

\begin{cppcode}
  AddTableEntry()
  DeleteUrl()
  OpenFileOrDie()
\end{cppcode}

\begin{DWarn}
  对于DBus接口函数,属性,信号,确保首字母大写,这也适用于一些其他风格的IPC/RPC接口或代码生成器生成的接口,包括dbus/protobuf/thrift等。
\end{DWarn}

同样的命名规则同时适用于类作用域与命名空间作用域的常量,因为它们是作为 API 的一部分暴露对外的,因此应当让它们看起来像是一个函数,因为在这时,它们实际上是一个对象而非函数的这一事实对外不过是一个无关紧要的实现细节。

取值和设值函数的命名与变量一致。一般来说它们的名称与实际的成员变量对应,但并不强制要求。例如 \cppinline{int getCount()} 与 \cppinline{void setCount(int count)}。

\section{命名空间命名}

\textbf{总述}
\begin{DWarn}
  命名空间以大写字母命名。最高级命名空间的名字取决于项目名称。要注意避免嵌套命名空间的名字之间和常见的顶级命名空间的名字之间发生冲突.
\end{DWarn}

顶级命名空间的名称应当是项目名或者是该命名空间中的代码所属的团队的名字。命名空间中的代码,应当存放于和命名空间的名字匹配的文件夹或其子文件夹中.

注意 \DFullRef{general-naming-rules} 的规则同样适用于命名空间。命名空间中的代码极少需要涉及命名空间的名称,因此没有必要在命名空间中使用缩写.

要避免嵌套的命名空间与常见的顶级命名空间发生名称冲突。由于名称查找规则的存在,命名空间之间的冲突完全有可能导致编译失败。尤其是,不要创建嵌套的 \cppinline{std} 命名空间。建议使用更独特的项目标识符 (\cppinline{WebSearch::Index},\cppinline{WebSearch::IndexUtil}) 而非常见的极易发生冲突的名称 (比如 \cppinline{WebSearch::Util}).

对于 \cppinline{Internal} 命名空间,要当心加入到同一 \cppinline{internal} 命名空间的代码之间发生冲突 (由于内部维护人员通常来自同一团队,因此常有可能导致冲突)。在这种情况下,请使用文件名以使得内部名称独一无二 (例如对于 \cppinline{frobber.h},使用 \cppinline{WebSearch::Index::FrobberInternal})。

\section{枚举命名}

\textbf{总述}

枚举的命名应当和 \DFullRef{type-names} 一致: \cppinline{EnumName} 。

\textbf{说明}

单独的枚举值使用首字母大写的大小写混合命名方式。枚举名 \cppinline{UrlTableErrors} (以及 \cppinline{AlternateUrlTableErrors}) 是类型,所以要用大小写混合的方式.

\begin{cppcode}
  enum UrlTableErrors {
      OK = 0,
      ErrorOutOfMemory,
      ErrorMalformedInput,
    };
\end{cppcode}

2009 年 1 月之前,Google 一直建议采用\DFullRef{macro-names}的方式命名枚举值。由于枚举值和宏之间的命名冲突,直接导致了很多问题。由此,这里改为优先选择\DFullRef{type-names}的方式。新代码应该尽可能优先使用\DFullRef{type-names}的方式。但是老代码没必要切换到\DFullRef{type-names}的方式,除非宏风格确实会产生编译期问题。

\section{宏命名} \label{macro-names}

\textbf{总述}

你并不打算 \DFullRef{preprocessor-macros},对吧? 如果你一定要用,像这样命名:

\cppinline{MY_MACRO_THAT_SCARES_SMALL_CHILDREN}.

\textbf{说明}

参考 \DFullRef{preprocessor-macros}; 通常 *不应该* 使用宏。如果不得不用,其命名像枚举命名一样全部大写,使用下划线:

\begin{cppcode}
  #define ROUND(x) ...
  #define PI_ROUNDED 3.0
\end{cppcode}

\section{命名规则的特例}

\textbf{总述}

如果你命名的实体与已有 C/C++ 实体相似,可参考现有命名策略。

\begin{DWarn}
  如果是为了扩展STL的接口,或继承其他底层库的函数,则可以不受命名规则限制,以避免功能错误。
\end{DWarn}

\cppinline{bigopen()}: 函数名,参照 \cppinline{open()} 的形式

\cppinline{uint}: \cppinline{typedef}

\cppinline{bigpos}: \cppinline{struct} 或 \cppinline{class},参照 \cppinline{pos} 的形式

\cppinline{sparse_hash_map}: STL 型实体; 参照 STL 命名约定

\cppinline{LONGLONG_MAX}: 常量,如同 \cppinline{INT_MAX}

\section{注解}

\subsection{ 译者 (YuleFox) 笔记}

感觉 Google 的命名约定很高明,比如写了简单的类 QueryResult,接着又可以直接定义一个变量 \cppinline{query_result},区分度很好; 再次,类内变量以下划线结尾,那么就可以直接传入同名的形参,比如 \cppinline{TextQuery::TextQuery(std::string word) : word_(word) {}} ,其中 \cppinline{word_} 自然是类内私有成员.

\subsection{ deepin风格注解 }

对Qt风格同理:

比如写了简单的类 \cppinline{QueryResult},接着又可以直接定义一个变量 \cppinline{queryResult},区分度很好; 再次,类内变量以下划线结尾,那么就可以直接传入同名的形参,比如 \cppinline{TextQuery::TextQuery(QString word) : m_word(word) {}} ,其中 \cppinline{m_word} 自然是类内私有成员.

\input{qt/comments.tex}
% TODO(iceyer): use clang-format
% \input{qt/formatting.tex}
\chapter{规则特例}

前面说明的编程习惯基本都是强制性的。 但所有优秀的规则都允许例外, 这里就是探讨这些特例。

\section{现有不合规范的代码}

\textbf{总述}

对于现有不符合既定编程风格的代码可以网开一面。

\textbf{说明}

当你修改使用其他风格的代码时, 为了与代码原有风格保持一致可以不使用本指南约定。 如果不放心, 可以与代码原作者或现在的负责人员商讨。 记住, *一致性* 也包括原有的一致性。

\section{Windows 代码} \label{windows-code}

\textbf{总述}

Windows 程序员有自己的编程习惯, 主要源于 Windows 头文件和其它 Microsoft 代码。 我们希望任何人都可以顺利读懂你的代码, 所以针对所有平台的 C++ 编程只给出一个单独的指南。

\textbf{说明}

如果你习惯使用 Windows 编码风格, 这儿有必要重申一下某些你可能会忘记的指南:

% TODO(iceyer): naming and pragma once need to be discuss
\begin{itemize}
  \item  不要使用匈牙利命名法 (比如把整型变量命名成 \cppinline{iNum})。 使用 Google 命名约定, 包括对源文件使用 \cppinline{.cc} 扩展名。
  \item Windows 定义了很多原生类型的同义词 (YuleFox 注: 这一点, 我也很反感), 如 \cppinline{DWORD}, \cppinline{HANDLE} 等等。 在调用 Windows API 时这是完全可以接受甚至鼓励的。 即使如此, 还是尽量使用原有的 C++ 类型, 例如使用 \cppinline{const TCHAR *} 而不是 \cppinline{LPCTSTR}。
  \item 使用 Microsoft Visual C++ 进行编译时, 将警告级别设置为 3 或更高, 并将所有警告(warnings)当作错误(errors)处理。
  \item 不要使用 \cppinline{#pragma once}; 而应该使用 Google 的头文件保护规则。 头文件保护的路径应该相对于项目根目录 (Yang.Y 注: 如 \cppinline{#ifndef SRC_DIR_BAR_H_}, 参考 \DFullRef{qt-pragma-once-guard} 一节)。
  \item 除非万不得已, 不要使用任何非标准的扩展, 如 \cppinline{#pragma} 和 \cppinline{__declspec}。 使用 \cppinline{__declspec(dllimport)} 和 \cppinline{__declspec(dllexport)} 是允许的, 但必须通过宏来使用, 比如 \cppinline{DLLIMPORT} 和 \cppinline{DLLEXPORT}, 这样其他人在分享使用这些代码时可以很容易地禁用这些扩展。
\end{itemize}

然而, 在 Windows 上仍然有一些我们偶尔需要违反的规则:

\begin{itemize}
  \item 通常我们 \DFullRef{multiple-inheritance}, 但在使用 COM 和 ATL/WTL 类时可以使用多重继承。 为了实现 COM 或 ATL/WTL 类/接口, 你可能不得不使用多重实现继承。
  \item  虽然代码中不应该使用异常, 但是在 ATL 和部分 STL(包括 Visual C++ 的 STL) 中异常被广泛使用。 使用 ATL 时, 应定义 \cppinline{_ATL_NO_EXCEPTIONS} 以禁用异常。 你需要研究一下是否能够禁用 STL 的异常, 如果无法禁用, 可以启用编译器异常。 (注意这只是为了编译 STL, 自己的代码里仍然不应当包含异常处理)。
  \item  通常为了利用头文件预编译, 每个每个源文件的开头都会包含一个名为 \cppinline{StdAfx.h} 或 \cppinline{precompile.h} 的文件。 为了使代码方便与其他项目共享, 请避免显式包含此文件 (除了在 \cppinline{precompile.cc} 中), 使用 \cppinline{/FI} 编译器选项以自动包含该文件。
  \item  资源头文件通常命名为 \cppinline{resource.h} 且只包含宏, 这一文件不需要遵守本风格指南。
\end{itemize}
\input{qt/end.tex}

\part{Qt代码风格}

\input{qt/flyleaf.tex}
\chapter{头文件}

通常每一个 \cppinline{.cpp} 文件都有一个对应的 \cppinline{.h} 文件. 也有一些常见例外, 如单元测试代码和只包含 \cppinline{main()} 函数的\cppinline{.cpp} 文件。

正确使用头文件可令代码在可读性、文件大小和性能上大为改观。

下面的规则将引导你规避使用头文件时的各种陷阱。

\section{Self-contained 头文件} \label{self-contained-headers}

\DBox {
	头文件应该能够自给自足(self-contained,也就是可以作为第一个头文件被引入),以 \cppinline{.h} 结尾。至于用来插入文本的文件,说到底它们并不是头文件,所以应以 \cppinline{.inc} 结尾。不允许分离出 \cppinline{-inl.h} 头文件的做法。
}

所有头文件要能够自给自足。换言之,用户和重构工具不需要为特别场合而包含额外的头文件。详言之,一个头文件要有\DFullRef{pragma-once-guard},统统包含它所需要的其它头文件,也不要求定义任何特别 symbols。

不过有一个例外,即一个文件并不是 self-contained 的,而是作为文本插入到代码某处。或者,文件内容实际上是其它头文件的特定平台(platform-specific)扩展部分。这些文件就要用\cppinline{.inc} 文件扩展名。

如果 \cppinline{.h} 文件声明了一个模板或内联函数,同时也在该文件加以定义。凡是有用到这些的 \cppinline{.cpp} 文件,就得统统包含该头文件,否则程序可能会在构建中链接失败。不要把这些定义放到分离的 \cppinline{-inl.h} 文件里(译者注:过去该规范曾提倡把定义放到 -inl.h 里过)。

有个例外:如果某函数模板为所有相关模板参数显式实例化,或本身就是某类的一个私有成员,那么它就只能定义在实例化该模板的 \cppinline{.cpp} 文件里。

\section{#define 保护} \label{pragma-once-guard}

\DBox{
  所有头文件都应该使用 \cppinline{#define} 来防止头文件被多重包含,命名格式当是:\cppinline{<PROJECT>_<PATH>_<FILE>_H_}。
}

为保证唯一性,头文件的命名应该基于所在项目源代码树的全路径。例如,项目 foo 中的头文件 foo/src/bar/baz.h 可按如下方式保护:

\begin{cppcode}
  #ifndef FOO_BAR_BAZ_H_
  #define FOO_BAR_BAZ_H_
  ...
  #endif // FOO_BAR_BAZ_H_
\end{cppcode}

\section{前置声明} \label{forward-declarations}

\DBox {
	尽可能地避免使用前置声明。使用 \cppinline{#include} 包含需要的头文件即可。
}

\textbf{定义:}

所谓「前置声明」(forward declaration)是类、函数和模板的纯粹声明,没伴随着其定义.

\textbf{优点:}

\begin{itemize}
	\item 前置声明能够节省编译时间,多余的 \cppinline{#include} 会迫使编译器展开更多的文件,处理更多的输入。
	\item 前置声明能够节省不必要的重新编译的时间。 \cppinline{#include} 使代码因为头文件中无关的改动而被重新编译多次。
\end{itemize}

\textbf{缺点:}

\begin{itemize}
	\item 前置声明隐藏了依赖关系,头文件改动时,用户的代码会跳过必要的重新编译过程。
	\item 前置声明可能会被库的后续更改所破坏。前置声明函数或模板有时会妨碍头文件开发者变动其 API。例如扩大形参类型,加个自带默认参数的模板形参等等。
	\item 前置声明来自命名空间 \cppinline{std::} 的 symbol 时,其行为未定义。
	\item 很难判断什么时候该用前置声明,什么时候该用 \cppinline{#include} 。极端情况下,用前置声明代替 \cppinline{#include} 甚至都会暗暗地改变代码的含义:

\begin{cppcode}
// b.h:
struct B {};
struct D : B {};

// good_user.cpp:
#include "b.h"
void f(B*);
void f(void*);
void test(D* x) { f(x); }  // calls f(B*)
\end{cppcode}

	      如果 \cppinline{#include} 被 \cppinline{B} 和 \cppinline{D} 的前置声明替代, \cppinline{test()} 就会调用 \cppinline{f(void*)} 。	\item 前置声明了不少来自头文件的 symbol 时,就会比单单一行的 \cppinline{include} 冗长。
	\item 仅仅为了能前置声明而重构代码(比如用指针成员代替对象成员)会使代码变得更慢更复杂.
\end{itemize}

\textbf{结论:}

\begin{itemize}
	\item 尽量避免前置声明那些定义在其他项目中的实体。
	\item 函数:总是使用 \cppinline{#include} 。
	\item 类模板:优先使用 \cppinline{#include} 。
\end{itemize}

至于什么时候包含头文件,参见 \DFullRef{name-and-order-of-includes} 。


\section{内联函数} \label{inline-functions}

\DBox {
	只有当函数只有 10 行甚至更少时才将其定义为内联函数。
}

\textbf{定义:}

当函数被声明为内联函数之后, 编译器会将其内联展开, 而不是按通常的函数调用机制进行调用。

\textbf{优点:}

只要内联的函数体较小, 内联该函数可以令目标代码更加高效. 对于存取函数以及其它函数体比较短, 性能关键的函数, 鼓励使用内联.

\textbf{缺点:}

滥用内联将导致程序变得更慢. 内联可能使目标代码量或增或减, 这取决于内联函数的大小. 内联非常短小的存取函数通常会减少代码大小,但内联一个相当大的函数将戏剧性的增加代码大小. 现代处理器由于更好的利用了指令缓存, 小巧的代码往往执行更快。

\textbf{结论:}

一个较为合理的经验准则是, 不要内联超过 10 行的函数. 谨慎对待析构函数, 析构函数往往比其表面看起来要更长,因为有隐含的成员和基类析构函数被调用!

另一个实用的经验准则: 内联那些包含循环或 \cppinline{switch} 语句的函数常常是得不偿失 (除非在大多数情况下, 这些循环或 \cppinline{switch} 语句从不被执行).

有些函数即使声明为内联的也不一定会被编译器内联, 这点很重要; 比如虚函数和递归函数就不会被正常内联. 通常,递归函数不应该声明成内联函数.(YuleFox 注: 递归调用堆栈的展开并不像循环那么简单, 比如递归层数在编译时可能是未知的,大多数编译器都不支持内联递归函数). 虚函数内联的主要原因则是想把它的函数体放在类定义内, 为了图个方便, 抑或是当作文档描述其行为,比如精短的存取函数.

\section{ include 的路径及顺序} \label{name-and-order-of-includes}

\DBox{
	使用标准的头文件包含顺序可增强可读性, 避免隐藏依赖: 相关头文件, C 库, C++ 库, 其他库的 `.h`, 本项目内的 `.h`.
}

项目内头文件应按照项目源代码目录树结构排列, 避免使用 UNIX 特殊的快捷目录 \cppinline{.} (当前目录) 或 \cppinline{..} (上级目录)。

例如, \cppinline{google-awesome-project/src/base/logging.h} 应该按如下方式包含:

\begin{cppcode}
	#include "base/logging.h"
\end{cppcode}

又如, \cppinline{dir/foo.cpp} 或 \cppinline{dir/foo_test.cpp} 的主要作用是实现或测试 \cppinline{dir2/foo2.h}的功能, \cppinline{foo.cpp} 中包含头文件的次序如下:

\begin{enumerate}
	\item \cppinline{dir2/foo2.h} (优先位置, 详情如下)
	\item C 系统文件
	\item C++ 系统文件
	\item 其他库的 \cppinline{.h} 文件
	\item 本项目内 \cppinline{.h} 文件
\end{enumerate}

这种优先的顺序排序保证当 \cppinline{dir2/foo2.h} 遗漏某些必要的库时, \cppinline{dir/foo.cpp} 或 \cppinline{dir/foo_test.cpp} 的构建会立刻中止。因此这一条规则保证维护这些文件的人们首先看到构建中止的消息而不是维护其他包的人们。

\cppinline{dir/foo.cpp} 和 \cppinline{dir2/foo2.h} 通常位于同一目录下,如\cppinline{base/basictypes_unittest.cpp} 和 \cppinline[breaklines]{base/basictypes.h}, 但也可以放在不同目录下.

按字母顺序分别对每种类型的头文件进行二次排序是不错的主意。注意较老的代码可不符合这条规则,要在方便的时候改正它们。

您所依赖的符号 (symbols) 被哪些头文件所定义,您就应该包含(include)哪些头文件,`前置声明`(forward declarations) 情况除外。比如您要用到 \cppinline{bar.h} 中的某个符号, 哪怕您所包含的 \cppinline{foo.h} 已经包含了\cppinline{bar.h}, 也照样得包含 \cppinline{bar.h}, 除非 \cppinline{foo.h} 有明确说明它会自动向您提供 \cppinline{bar.h} 中的symbol. 不过,凡是 cc 文件所对应的「相关头文件」已经包含的,就不用再重复包含进其 cc 文件里面了,就像 \cppinline{foo.cpp}只包含 \cppinline{foo.h} 就够了,不用再管后者所包含的其它内容。

举例来说, \cppinline{google-awesome-project/src/foo/internal/fooserver.cpp} 的包含次序如下:

\begin{cppcode}
	#include "foo/public/fooserver.h" // 优先位置

	#include <sys/types.h>
	#include <unistd.h>

	#include <hash_map>
	#include <vector>

	#include "base/basictypes.h"
	#include "base/commandlineflags.h"
	#include "foo/public/bar.h"
\end{cppcode}

\textbf{例外:}

有时,平台特定(system-specific)代码需要条件编译(conditional includes),这些代码可以放到其它 includes 之后。当然,您的平台特定代码也要够简练且独立,比如:

\begin{cppcode}
	#include "foo/public/fooserver.h"

	#include "base/port.h"  // For LANG_CXX11.

	#ifdef LANG_CXX11
	#include <initializer_list>
	#endif  // LANG_CXX11
\end{cppcode}


\section{注解}

\subsection{译者 (YuleFox) 笔记}

\begin{itemize}
	\item  避免多重包含是学编程时最基本的要求;
	\item  前置声明是为了降低编译依赖,防止修改一个头文件引发多米诺效应;
	\item  内联函数的合理使用可提高代码执行效率;
	\item  \cppinline{-inl.h} 可提高代码可读性 (一般用不到吧:D);
	\item  标准化函数参数顺序可以提高可读性和易维护性 (对函数参数的堆栈空间有轻微影响, 我以前大多是相同类型放在一起);
	\item  包含文件的名称使用 \cppinline{.} 和 \cppinline{..} 虽然方便却易混乱, 使用比较完整的项目路径看上去很清晰, 很条理,包含文件的次序除了美观之外, 最重要的是可以减少隐藏依赖, 使每个头文件在 "最需要编译" (对应源文件处 :D) 的地方编译,有人提出库文件放在最后, 这样出错先是项目内的文件, 头文件都放在对应源文件的最前面, 这一点足以保证内部错误的及时发现了.
\end{itemize}

\subsection{译者(acgtyrant)笔记}

\begin{itemize}
	\item  原来还真有项目用 \cppinline{#include} 来插入文本,且其文件扩展名 \cppinline{.inc} 看上去也很科学。
	\item  Google 已经不再提倡 \cppinline{-inl.h} 用法。
	\item  注意,前置声明的类是不完全类型(incomplete type),我们只能定义指向该类型的指针或引用,或者声明(但不能定义)以不完全类型作为参数或者返回类型的函数。毕竟编译器不知道不完全类型的定义,我们不能创建其类的任何对象,也不能声明成类内部的数据成员。
	\item  类内部的函数一般会自动内联。所以某函数一旦不需要内联,其定义就不要再放在头文件里,而是放到对应的 \cppinline{.cpp} 文件里。这样可以保持头文件的类相当精炼,也很好地贯彻了声明与定义分离的原则。
	\item  在 \cppinline{#include} 中插入空行以分割相关头文件, C 库, C++ 库, 其他库的 \cppinline{.h} 和本项目内的 \cppinline{.h} 是个好习惯。
\end{itemize}


\chapter{作用域}

\section{命名空间} \label{qt-namespace}

\DBox{
鼓励在 \cppinline{.cpp} 文件内使用匿名命名空间或 \cppinline{static} 声明。使用具名的命名空间时,其名称可基于项目名或相对路径。
禁止使用 using 指示(using-directive)。禁止使用内联命名空间(inline namespace)。
}

\textbf{定义:}

命名空间将全局作用域细分为独立的,具名的作用域,可有效防止全局作用域的命名冲突。

\textbf{优点:}

虽然类已经提供了(可嵌套的)命名轴线 (YuleFox 注:将命名分割在不同类的作用域内), 命名空间在这基础上又封装了一层。

举例来说,两个不同项目的全局作用域都有一个类 \cppinline{Foo}, 这样在编译或运行时造成冲突。如果每个项目将代码置于不同命名空间中,
\cppinline{project1::Foo} 和 \cppinline{project2::Foo} 作为不同符号自然不会冲突。

内联命名空间会自动把内部的标识符放到外层作用域,比如:

% \begin{noindent}
\begin{cppcode}
namespace X {
inline namespace Y {
  void foo();
}  // namespace Y
}  // namespace X
\end{cppcode}
%\end{noindent}

\cppinline{X::Y::foo()} 与 \cppinline{X::foo()} 彼此可代替。内联命名空间主要用来保持跨版本的 ABI 兼容性。

\textbf{缺点:}

命名空间具有迷惑性,因为它们使得区分两个相同命名所指代的定义更加困难。

内联命名空间很容易令人迷惑,毕竟其内部的成员不再受其声明所在命名空间的限制。内联命名空间只在大型版本控制里有用。

有时候不得不多次引用某个定义在许多嵌套命名空间里的实体,使用完整的命名空间会导致代码的冗长。

在头文件中使用匿名空间导致违背 C++ 的唯一定义原则 (One Definition Rule (ODR))。

\textbf{结论:}

根据下文将要提到的策略合理使用命名空间。

\begin{itemize}
  \item 遵守 \DFullRef{qt-namespace-names} 中的规则。
  \item 像之前的几个例子中一样,在命名空间的最后注释出命名空间的名字。
  \item 用命名空间把文件包含,`gflags <https://gflags.github.io/gflags/>` 的声明/定义,以及类的前置声明以外的整个源文件封装起来,以区别于其它命名空间:

% \begin{noindent}
\begin{cppcode}
  // .h 文件
namespace mynamespace {

// 所有声明都置于命名空间中
// 注意不要使用缩进
class MyClass {
public:
...
void Foo();
};

} // namespace mynamespace
\end{cppcode}
%\end{noindent}

%\begin{noindent}
\begin{cppcode}
// .cpp 文件
namespace mynamespace {

// 函数定义都置于命名空间中
void MyClass::Foo() {
...
}

} // namespace mynamespace
\end{cppcode}
%\end{noindent}

更复杂的 \cppinline{.cpp} 文件包含更多,更复杂的细节,比如 gflags 或 using 声明。

%\begin{noindent}
\begin{cppcode}
#include "a.h"

DEFINE_FLAG(bool, someflag, false, "dummy flag");

namespace a {

...code for a...// 左对齐

} // namespace a
\end{cppcode}
% \end{noindent}

  \item 不要在命名空间 \cppinline{std} 内声明任何东西,包括标准库的类前置声明。在 \cppinline{std} 命名空间声明实体是未定义的行为,会导致如不可移植。声明标准库下的实体,需要包含对应的头文件。

  \item 不应该使用 \textit{using 指示} 引入整个命名空间的标识符号。

%\begin{noindent}
\begin{cppcode}
// 禁止 —— 污染命名空间
using namespace foo;
\end{cppcode}
% \end{noindent}

  \item  不要在头文件中使用 \textit{命名空间别名} 除非显式标记内部命名空间使用。因为任何在头文件中引入的命名空间都会成为公开 API 的一部分。

%\begin{noindent}
\begin{cppcode}
// 在 .cpp 中使用别名缩短常用的命名空间
namespace baz = ::foo::bar::baz;
\end{cppcode}

\begin{cppcode}
// 在 .h 中使用别名缩短常用的命名空间
namespace librarian {
namespace impl {  // 仅限内部使用
namespace sidetable = ::pipeline_diagnostics::sidetable;
}  // namespace impl

inline void my_inline_function() {
  // 限制在一个函数中的命名空间别名
  namespace baz = ::foo::bar::baz;
...
}
}  // namespace librarian
\end{cppcode}
% \end{noindent}

  \item  禁止用内联命名空间。
\end{itemize}

\section{匿名命名空间和静态变量} \label{unnamed-namespace-and-static-variables}

\DBox {
在 \cppinline{.cpp} 文件中定义一个不需要被外部引用的变量时,可以将它们放在匿名命名空间或声明为 \cppinline{static} 。但是不要在\cppinline{.h} 文件中这么做。
}

\textbf{定义:}

所有置于匿名命名空间的声明都具有内部链接性,函数和变量可以经由声明为 \cppinline{static} 拥有内部链接性,这意味着你在这个文件中声明的这些标识符都不能在另一个文件中被访问。即使两个文件声明了完全一样名字的标识符,它们所指向的实体实际上是完全不同的。

\textbf{结论:}

推荐、鼓励在 \cppinline{.cpp} 中对于不需要在其他地方引用的标识符使用内部链接性声明,但是不要在 \cppinline{.h} 中使用。

匿名命名空间的声明和具名的格式相同,在最后注释上 \cppinline{namespace} :

%\begin{noindent}
\begin{cppcode}
namespace {
...
}  // namespace
\end{cppcode}
% \end{noindent}

\section{非成员函数、静态成员函数和全局函数} \label{nonmember-static-member-and-global-functions}

\DBox{
使用静态成员函数或命名空间内的非成员函数,尽量不要用裸的全局函数。将一系列函数直接置于命名空间中,不要用类的静态方法模拟出命名空间的效果,类的静态方法应当和类的实例或静态数据紧密相关。
}

\textbf{优点:}

某些情况下,非成员函数和静态成员函数是非常有用的,将非成员函数放在命名空间内可避免污染全局作用域。

\textbf{缺点:}

将非成员函数和静态成员函数作为新类的成员或许更有意义,当它们需要访问外部资源或具有重要的依赖关系时更是如此。

\textbf{结论:}

有时,把函数的定义同类的实例脱钩是有益的,甚至是必要的。这样的函数可以被定义成静态成员,或是非成员函数。

非成员函数不应依赖于外部变量,应尽量置于某个命名空间内。相比单纯为了封装若干不共享任何静态数据的静态成员函数而创建类,不如使用\DFullRef{qt-namespace}。举例而言,对于头文件 \cppinline{myproject/foo_bar.h}  , 应当使用

\begin{cppcode}
  namespace myproject {
  namespace foo_bar {
  void Function1();
  void Function2();
  }  // namespace foo_bar
  }  // namespace myproject
\end{cppcode}

而非

\begin{cppcode}
  namespace myproject {
      class FooBar {
          public:
          static void Function1();
          static void Function2();
        };
    }  // namespace myproject
\end{cppcode}

定义在同一编译单元的函数,被其他编译单元直接调用可能会引入不必要的耦合和链接时依赖; 静态成员函数对此尤其敏感。可以考虑提取到新类中,或者将函数置于独立库的命名空间内。

如果你必须定义非成员函数,又只是在 \cppinline{.cpp} 文件中使用它,可使用匿名 \DFullRef{qt-namespace} 或 \cppinline{static} 链接关键字 (如 \cppinline{static int Foo() {...}}) 限定其作用域。

\section{局部变量} \label{local-variables}

\DBox{
将函数变量尽可能置于最小作用域内,并在变量声明时进行初始化。
}

C++ 允许在函数的任何位置声明变量。我们提倡在尽可能小的作用域中声明变量,离第一次使用越近越好。这使得代码浏览者更容易定位变量声明的位置,了解变量的类型和初始值。特别是,应使用初始化的方式替代声明再赋值,比如:

\begin{cppcode}
  int i;
  i = f(); // 坏——初始化和声明分离

  int j = g(); // 好——初始化时声明

  vector<int> v;
  v.push_back(1); // 用花括号初始化更好
  v.push_back(2);

  vector<int> v = {1, 2}; // 好——v 一开始就初始化
\end{cppcode}

属于 \cppinline{if}, \cppinline{while} 和 \cppinline{for} 语句的变量应当在这些语句中正常地声明,这样子这些变量的作用域就被限制在这些语句中了,举例而言:

\begin{cppcode}
  while (const char* p = strchr(str, '/')) str = p + 1;
\end{cppcode}

\begin{DWarn}
有一个例外,如果变量是一个对象,每次进入作用域都要调用其构造函数,每次退出作用域都要调用其析构函数。这会导致效率降低。
\end{DWarn}

\begin{cppcode}
  // 低效的实现
  for (int i = 0; i < 1000000; ++i) {
      Foo f;                  // 构造函数和析构函数分别调用 1000000 次!
      f.DoSomething(i);
    }
\end{cppcode}

在循环作用域外面声明这类变量要高效的多:

\begin{cppcode}
  Foo f;                      // 构造函数和析构函数只调用 1 次
  for (int i = 0; i < 1000000; ++i) {
      f.DoSomething(i);
    }
\end{cppcode}


\section{静态和全局变量} \label{static-and-global-variables}

\DBox{
禁止定义静态储存周期非 POD 变量,禁止使用含有副作用的函数初始化 POD 全局变量,因为多编译单元中的静态变量执行时的构造和析构顺序是未明确的,这将导致代码的不可移植。
}

禁止使用类的\href{http://zh.cppreference.com/w/cpp/language/storage_duration#.E5.AD.98.E5.82.A8.E6.9C.9F}{静态储存周期}变量:由于构造和析构函数调用顺序的不确定性,它们会导致难以发现的 bug。不过 \cppinline{constexpr}变量除外,毕竟它们又不涉及动态初始化或析构。

静态生存周期的对象,即包括了全局变量,静态变量,静态类成员变量和函数静态变量,都必须是原生数据类型 (POD : Plain OldData): 即 int, char 和 float, 以及 POD 类型的指针、数组和结构体。

静态变量的构造函数、析构函数和初始化的顺序在 C++ 中是只有部分明确的,甚至随着构建变化而变化,导致难以发现的 bug。所以除了禁用类类型的全局变量,我们也不允许用函数返回值来初始化 POD 变量,除非该函数(比如 \cppinline{getenv()} 或\cppinline{getpid()})不涉及任何全局变量。函数作用域里的静态变量除外,毕竟它的初始化顺序是有明确定义的,而且只会在指令执行到它的声明那里才会发生。

\begin{DNote}
  Xris 译注:

  同一个编译单元内是明确的,静态初始化优先于动态初始化,初始化顺序按照声明顺序进行,销毁则逆序。不同的编译单元之间初始化和销毁顺序属于未明确行为
  (unspecified behaviour)。

\end{DNote}

同理,全局和静态变量在程序中断时会被析构,无论所谓中断是从 \cppinline{main()} 返回还是对 \cppinline{exit()} 的调用。析构顺序正好与构造函数调用的顺序相反。但既然构造顺序未定义,那么析构顺序当然也就不定了。比如,在程序结束时某静态变量已经被析构了,但代码还在跑——比如其它线程——并试图访问它且失败;再比如,一个静态 string 变量也许会在一个引用了前者的其它变量析构之前被析构掉。

改善以上析构问题的办法之一是用 \cppinline{quick_exit()} 来代替 \cppinline{exit()}并中断程序。它们的不同之处是前者不会执行任何析构,也不会执行 \cppinline{atexit()} 所绑定的任何 handlers。如果您想在执行 \cppinline{quick_exit()} 来中断时执行某 handler(比如刷新 log),您可以把它绑定到\cppinline{_at_quick_exit()}. 如果您想在 \cppinline{exit()} 和 \cppinline{quick_exit()} 都用上该 handler,都绑定上去。

综上所述,我们只允许 POD 类型的静态变量,即完全禁用 \cppinline{vector} (使用 C 数组替代) 和 \cppinline{string} (使用\cppinline{const char []})。

如果您确实需要一个 \cppinline{class} 类型的静态或全局变量,可以考虑在 \cppinline{main()} 函数或 \cppinline{pthread_once()}
内初始化一个指针且永不回收。注意只能用 raw 指针,别用智能指针,毕竟后者的析构函数涉及到上文指出的不定顺序问题。

\begin{DNote}
  Yang.Y 译注:

  上文提及的静态变量泛指静态生存周期的对象,包括:全局变量,静态变量,静态类成员变量,以及函数静态变量。
\end{DNote}

\section{注解}

\subsection{ 译者 (YuleFox) 笔记}

\begin{itemize}
  \item \cppinline{cpp} 中的匿名命名空间可避免命名冲突,限定作用域,避免直接使用 \cppinline{using} 关键字污染命名空间。
  \item 嵌套类符合局部使用原则,只是不能在其他头文件中前置声明,尽量不要 \cppinline{public}。
  \item 尽量不用全局函数和全局变量,考虑作用域和命名空间限制,尽量单独形成编译单元。
  \item 多线程中的全局变量 (含静态成员变量) 不要使用 \cppinline{class} 类型 (含 STL 容器), 避免不明确行为导致的 bug。
  \item 作用域的使用,除了考虑名称污染,可读性之外,主要是为降低耦合,提高编译/执行效率。
\end{itemize}

\subsection{  译者(acgtyrant)笔记 }

\begin{itemize}
  \item 注意「using 指示(using-directive)」和「using 声明(using-declaration)」的区别。
  \item 匿名命名空间说白了就是文件作用域,就像 C static 声明的作用域一样,后者已经被 C++ 标准提倡弃用。
  \item 局部变量在声明的同时进行显式值初始化,比起隐式初始化再赋值的两步过程要高效,同时也贯彻了计算机体系结构重要的概念「局部性(locality)」。
  \item 注意别在循环犯大量构造和析构的低级错误。
\end{itemize}
\input{qt/classes.tex}
\chapter{函数}

\section{参数顺序}

\textbf{总述}

函数的参数顺序为: 输入参数在先, 后跟输出参数。

\textbf{说明}

C/C++ 中的函数参数或者是函数的输入, 或者是函数的输出, 或兼而有之。 输入参数通常是值参或 \cppinline{const} 引用, 输出参数或输入/输出参数则一般为非 \cppinline{const} 指针。 在排列参数顺序时, 将所有的输入参数置于输出参数之前。 特别要注意, 在加入新参数时不要因为它们是新参数就置于参数列表最后, 而是仍然要按照前述的规则, 即将新的输入参数也置于输出参数之前。

这并非一个硬性规定。 输入/输出参数 (通常是类或结构体) 让这个问题变得复杂。 并且, 有时候为了其他函数保持一致, 你可能不得不有所变通。

\section{编写简短函数}

\textbf{总述}

我们倾向于编写简短, 凝练的函数。

\textbf{说明}

我们承认长函数有时是合理的, 因此并不硬性限制函数的长度。 如果函数超过 40 行, 可以思索一下能不能在不影响程序结构的前提下对其进行分割。

即使一个长函数现在工作的非常好, 一旦有人对其修改, 有可能出现新的问题, 甚至导致难以发现的 bug。 使函数尽量简短, 以便于他人阅读和修改代码。

在处理代码时, 你可能会发现复杂的长函数。 不要害怕修改现有代码: 如果证实这些代码使用 / 调试起来很困难, 或者你只需要使用其中的一小段代码, 考虑将其分割为更加简短并易于管理的若干函数。

\section{引用参数}

\textbf{总述}

所有按引用传递的参数必须加上 \cppinline{const}。

\textbf{定义}

在 C 语言中, 如果函数需要修改变量的值, 参数必须为指针, 如 \cppinline{int foo(int *pval)}。 在 C++ 中, 函数还可以声明为引用参数: \cppinline{int foo(int &val)}。

\textbf{优点}

定义引用参数可以防止出现 \cppinline{(*pval)++} 这样丑陋的代码。 引用参数对于拷贝构造函数这样的应用也是必需的。 同时也更明确地不接受空指针。

\textbf{缺点}

容易引起误解, 因为引用在语法上是值变量却拥有指针的语义。

\textbf{结论}

函数参数列表中, 所有引用参数都必须是 \cppinline{const}:

\begin{cppcode}
  void Foo(const string &in, string *out);
\end{cppcode}

事实上这在 Google Code 是一个硬性约定: 输入参数是值参或 \cppinline{const} 引用, 输出参数为指针。 输入参数可以是 \cppinline{const} 指针, 但决不能是非 \cppinline{const} 的引用参数, 除非特殊要求, 比如 \cppinline{swap()}。

有时候, 在输入形参中用 \cppinline{const T*} 指针比 \cppinline{const T&} 更明智。 比如:

* 可能会传递空指针。

* 函数要把指针或对地址的引用赋值给输入形参。

总而言之, 大多时候输入形参往往是 \cppinline{const T&}。 若用 \cppinline{const T*} 则说明输入另有处理。 所以若要使用 \cppinline{const T*}, 则应给出相应的理由, 否则会使得读者感到迷惑。

\section{函数重载} \label{function-overloading}

\textbf{总述}

若要使用函数重载, 则必须能让读者一看调用点就胸有成竹, 而不用花心思猜测调用的重载函数到底是哪一种。 这一规则也适用于构造函数。

\textbf{定义}

你可以编写一个参数类型为 \cppinline{const string&} 的函数, 然后用另一个参数类型为 \cppinline{const char*} 的函数对其进行重载:


\begin{cppcode}
  class MyClass {
      public:
      void Analyze(const string &text);
      void Analyze(const char *text, size_t textlen);
    };
\end{cppcode}

\textbf{优点}

通过重载参数不同的同名函数, 可以令代码更加直观。 模板化代码需要重载, 这同时也能为使用者带来便利。

\textbf{缺点}

如果函数单靠不同的参数类型而重载 (acgtyrant 注:这意味着参数数量不变), 读者就得十分熟悉 C++ 五花八门的匹配规则, 以了解匹配过程具体到底如何。 另外, 如果派生类只重载了某个函数的部分变体, 继承语义就容易令人困惑。

\textbf{结论}
% TODO: change \textbf{列表初始化格式} ->  \DFullRef{braced-initializer-list} in formatting.tex not work now.
如果打算重载一个函数, 可以试试改在函数名里加上参数信息。 例如, 用 \cppinline{AppendString()} 和 \cppinline{AppendInt()} 等, 而不是一口气重载多个 \cppinline{Append()}。 如果重载函数的目的是为了支持不同数量的同一类型参数, 则优先考虑使用 \cppinline{std::vector} 以便使用者可以用\textbf{列表初始化格式}指定参数。

\section{缺省参数}

\textbf{总述}

只允许在非虚函数中使用缺省参数, 且必须保证缺省参数的值始终一致。 缺省参数与 \DFullRef{function-overloading} 遵循同样的规则。 一般情况下建议使用函数重载, 尤其是在缺省函数带来的可读性提升不能弥补下文中所提到的缺点的情况下。

\textbf{优点}

有些函数一般情况下使用默认参数, 但有时需要又使用非默认的参数。 缺省参数为这样的情形提供了便利, 使程序员不需要为了极少的例外情况编写大量的函数。 和函数重载相比, 缺省参数的语法更简洁明了, 减少了大量的样板代码, 也更好地区别了 "必要参数" 和 "可选参数"。

\textbf{缺点}

缺省参数实际上是函数重载语义的另一种实现方式, 因此所有 \DFullRef{function-overloading} 也都适用于缺省参数。

虚函数调用的缺省参数取决于目标对象的静态类型, 此时无法保证给定函数的所有重载声明的都是同样的缺省参数。

缺省参数是在每个调用点都要进行重新求值的, 这会造成生成的代码迅速膨胀。 作为读者, 一般来说也更希望缺省的参数在声明时就已经被固定了, 而不是在每次调用时都可能会有不同的取值。

缺省参数会干扰函数指针, 导致函数签名与调用点的签名不一致。 而函数重载不会导致这样的问题。

\textbf{结论}

对于虚函数, 不允许使用缺省参数, 因为在虚函数中缺省参数不一定能正常工作。 如果在每个调用点缺省参数的值都有可能不同, 在这种情况下缺省函数也不允许使用。 (例如, 不要写像 \cppinline{void f(int n = counter++);} 这样的代码。)

在其他情况下, 如果缺省参数对可读性的提升远远超过了以上提及的缺点的话, 可以使用缺省参数。 如果仍有疑惑, 就使用函数重载。

\section{函数返回类型后置语法}

\textbf{总述}

只有在常规写法 (返回类型前置) 不便于书写或不便于阅读时使用返回类型后置语法。

\textbf{定义}

C++ 现在允许两种不同的函数声明方式。 以往的写法是将返回类型置于函数名之前。 例如:

\begin{cppcode}
int foo(int x);
\end{cppcode}

C++11 引入了这一新的形式。 现在可以在函数名前使用 \cppinline{auto} 关键字, 在参数列表之后后置返回类型。 例如:

\begin{cppcode}
auto foo(int x) -> int;
\end{cppcode}

后置返回类型为函数作用域。 对于像 \cppinline{int} 这样简单的类型, 两种写法没有区别。 但对于复杂的情况, 例如类域中的类型声明或者以函数参数的形式书写的类型, 写法的不同会造成区别。

\textbf{优点}
% TODO: change \textbf{Lambda 表达式} -> \DFullRef{lambda-expressions} in formatting.tex not work now.
后置返回类型是显式地指定\textbf{Lambda 表达式}的返回值的唯一方式。 某些情况下, 编译器可以自动推导出 Lambda 表达式的返回类型, 但并不是在所有的情况下都能实现。 即使编译器能够自动推导, 显式地指定返回类型也能让读者更明了。

有时在已经出现了的函数参数列表之后指定返回类型, 能够让书写更简单, 也更易读, 尤其是在返回类型依赖于模板参数时。 例如:

\begin{cppcode}
  template <class T, class U> auto add(T t, U u) -> decltype(t + u);
\end{cppcode}

对比下面的例子:

\begin{cppcode}
  template <class T, class U> decltype(declval<T&>() + declval<U&>()) add(T t, U u);
\end{cppcode}

\textbf{缺点}

后置返回类型相对来说是非常新的语法, 而且在 C 和 Java 中都没有相似的写法, 因此可能对读者来说比较陌生。

在已有的代码中有大量的函数声明, 你不可能把它们都用新的语法重写一遍。 因此实际的做法只能是使用旧的语法或者新旧混用。 在这种情况下, 只使用一种版本是相对来说更规整的形式。

\textbf{结论}

在大部分情况下, 应当继续使用以往的函数声明写法, 即将返回类型置于函数名前。 只有在必需的时候 (如 Lambda 表达式) 或者使用后置语法能够简化书写并且提高易读性的时候才使用新的返回类型后置语法。 但是后一种情况一般来说是很少见的, 大部分时候都出现在相当复杂的模板代码中, 而多数情况下不鼓励写这样 \DFullRef{qt-template-metaprogramming}。

\input{qt/magic.tex}
% TODO(iceyer): use Qt smart pointer or STL?
\section{预处理宏} \label{qt-preprocessor-macros}

\begin{DNote}
  使用宏时要非常谨慎,尽量以内联函数,枚举和常量代替之。
\end{DNote}

宏意味着你和编译器看到的代码是不同的。这可能会导致异常行为,尤其因为宏具有全局作用域。

值得庆幸的是,C++ 中,宏不像在 C 中那么必不可少。以往用宏展开性能关键的代码,现在可以用内联函数替代。用宏表示常量可被 \cppinline{const} 变量代替。用宏 "缩写" 长变量名可被引用代替。用宏进行条件编译... 这个,千万别这么做,会令测试更加痛苦 ( \cppinline{#define} 防止头文件重包含当然是个特例).

宏可以做一些其他技术无法实现的事情,在一些代码库 (尤其是底层库中) 可以看到宏的某些特性 (如用 \cppinline{#} 字符串化,用 \cppinline{##} 连接等等). 但在使用前,仔细考虑一下能不能不使用宏达到同样的目的。

下面给出的用法模式可以避免使用宏带来的问题; 如果你要宏,尽可能遵守:

\begin{itemize}
  \item 不要在 \cppinline{.h} 文件中定义宏;
  \item 在马上要使用时才进行 \cppinline{#define}, 使用后要立即 \cppinline{#undef};
  \item 不要只是对已经存在的宏使用 \cppinline{#undef},选择一个不会冲突的名称;
  \item 不要试图使用展开后会导致 C++ 构造不稳定的宏,不然也至少要附上文档说明其行为;
  \item 不要用 \cppinline{##} 处理函数,类和变量的名字。
\end{itemize}

\section{模板编程} \label{qt-template-metaprogramming}

\begin{DNote}
不要使用复杂的模板编程。
\end{DNote}

\textbf{定义}

模板编程指的是利用 c++ 模板实例化机制是图灵完备性,可以被用来实现编译时刻的类型判断的一系列编程技巧。

\textbf{优点}

模板编程能够实现非常灵活的类型安全的接口和极好的性能,一些常见的工具比如 Google Test, std::tuple, std::function 和 Boost.Spirit. 这些工具如果没有模板是实现不了的。

\textbf{缺点}

\begin{itemize}
  \item 模板编程所使用的技巧对于使用 c++ 不是很熟练的人是比较晦涩,难懂的。在复杂的地方使用模板的代码让人更不容易读懂,并且 debug 和 维护起来都很麻烦。
  \item 模板编程经常会导致编译出错的信息非常不友好:在代码出错的时候,即使这个接口非常的简单,模板内部复杂的实现细节也会在出错信息显示。导致这个编译出错信息看起来非常难以理解。
  \item 大量的使用模板编程接口会让重构工具 (Visual Assist X, Refactor for C++ 等等) 更难发挥用途。首先模板的代码会在很多上下文里面扩展开来,所以很难确认重构对所有的这些展开的代码有用,其次有些重构工具只对已经做过模板类型替换的代码的 AST 有用。因此重构工具对这些模板实现的原始代码并不有效,很难找出哪些需要重构。
\end{itemize}

\textbf{结论}

\begin{itemize}
  \item 模板编程有时候能够实现更简洁更易用的接口,但是更多的时候却适得其反。因此模板编程最好只用在少量的基础组件,基础数据结构上,因为模板带来的额外的维护成本会被大量的使用给分担掉。
  \item 在使用模板编程或者其他复杂的模板技巧的时候,你一定要再三考虑一下。考虑一下你们团队成员的平均水平是否能够读懂并且能够维护你写的模板代码。或者一个非 c++ 程序员和一些只是在出错的时候偶尔看一下代码的人能够读懂这些错误信息或者能够跟踪函数的调用流程。如果你使用递归的模板实例化,或者类型列表,或者元函数,又或者表达式模板,或者依赖 SFINAE, 或者 sizeof 的 trick 手段来检查函数是否重载,那么这说明你模板用的太多了,这些模板太复杂了,我们不推荐使用。
  \item 如果你使用模板编程,你必须考虑尽可能的把复杂度最小化,并且尽量不要让模板对外暴露。你最好只在实现里面使用模板,然后给用户暴露的接口里面并不使用模板,这样能提高你的接口的可读性。并且你应该在这些使用模板的代码上写尽可能详细的注释。你的注释里面应该详细的包含这些代码是怎么用的,这些模板生成出来的代码大概是什么样子的。还需要额外注意在用户错误使用你的模板代码的时候需要输出更人性化的出错信息。因为这些出错信息也是你的接口的一部分,所以你的代码必须调整到这些错误信息在用户看起来应该是非常容易理解,并且用户很容易知道如何修改这些错误。
\end{itemize}
\chapter{命名约定}

最重要的一致性规则是命名管理。命名的风格能让我们在不需要去查找类型声明的条件下快速地了解某个名字代表的含义: 类型,变量,函数,常量,宏,等等,甚至 我们大脑中的模式匹配引擎非常依赖这些命名规则。\textbf{命名规则具有一定随意性,但相比按个人喜好命名,一致性更重要,所以无论你认为它们是否重要,规则总归是规则。}

\section{通用命名规则} \label{general-naming-rules}

\textbf{总述}

函数命名,变量命名,文件命名要有描述性; 少用缩写。

\textbf{说明}

尽可能使用描述性的命名,别心疼空间,毕竟相比之下让代码易于新读者理解更重要。不要用只有项目开发者能理解的缩写,也不要通过砍掉几个字母来缩写单词.

\begin{cppcode}
  int priceCountReader;     // 无缩写
  int numErrors;            // "num" 是一个常见的写法
  int numDnsConnections;    // 人人都知道 "DNS" 是什么
\end{cppcode}

\begin{cppcode}
  int n;                     // 毫无意义.
  int nerr;                  // 含糊不清的缩写.
  int nCompConns;            // 含糊不清的缩写.
  int wgcConnections;        // 只有贵团队知道是什么意思.
  int pcReader;              // "pc" 有太多可能的解释了.
  int cstmrID;               // 删减了若干字母.
\end{cppcode}

注意,一些特定的广为人知的缩写是允许的,例如用 \cppinline{i} 表示迭代变量和用 \cppinline{T} 表示模板参数。

\begin{DWarn}
  在D-Pointer风格中,\cppinline{d_ptr,q_ptr,dd_ptr,qq_ptr}都是保留的名称。
\end{DWarn}

模板参数的命名应当遵循对应的分类: 类型模板参数应当遵循 \DFullRef{type-names} 的规则,而非类型模板应当遵循  \DFullRef{variable-names} 的规则.

\section{文件命名}

\textbf{总述}

文件名要全部小写,可以包含下划线 (\cppinline{_}) 。如果存在无法\cppinline{_}的情况,可以考虑使用连字符\cppinline{-},否则不允许例外。

\begin{DWarn}
  Qt默认情况下不使用任何连接符合,这使得文件名非常难以看懂,我们不接受这种风格。deepin的Qt项目统一使用下划线(\cppinline{_})作为文件名连接符合。
\end{DWarn}

\textbf{说明}

可接受的文件命名示例:

\begin{cppcode}
  my_useful_class.cpp
  myusefulclass_test.cpp // \cppinline{_unittest} 和 \cppinline{_regtest} 已弃用.
\end{cppcode}

不接受的文件命名示例:

\begin{cppcode}
  my-useful-class.cpp  // 不接受,除非无法使用_
  myusefulclass.cpp    // 不接受,难以看懂
\end{cppcode}

C++ 文件要以 \cppinline{.cpp} 结尾,头文件以 \cppinline{.h} 结尾。专门插入文本的文件则以 \cppinline{.inc} 结尾,参见 \DFullRef{self-contained-headers}。

不要使用已经存在于 \cppinline{/usr/include} 下的文件名 (Yang.Y 注: 即编译器搜索系统头文件的路径),如 \cppinline{db.h}。

通常应尽量让文件名更加明确。\cppinline{http_server_logs.h} 就比 \cppinline{logs.h} 要好。定义类时文件名一般成对出现,如 \cppinline{foo_bar.h} 和 \cppinline{foo_bar.cpp},对应于类 \cppinline{FooBar}。

内联函数必须放在 \cppinline{.h} 文件中。如果内联函数比较短,就直接放在 \cppinline{.h} 中.

\section{类型命名} \label{type-names}

\textbf{总述}

类型名称的每个单词首字母均大写,不包含下划线: \cppinline{MyExcitingClass},\cppinline{MyExcitingEnum}。

\textbf{说明}

所有类型命名 —— 类,结构体,类型定义 (\cppinline{typedef}),枚举,类型模板参数 —— 均使用相同约定,即以大写字母开始,每个单词首字母均大写,不包含下划线。例如:

\begin{cppcode}
  // 类和结构体
  class UrlTable { ...
  class UrlTableTester { ...
  struct UrlTableProperties { ...

  // 类型定义
  typedef hash_map<UrlTableProperties *, string> PropertiesMap;

  // using 别名
  using PropertiesMap = hash_map<UrlTableProperties *, string>;

  // 枚举
  enum UrlTableErrors { ...
\end{cppcode}

\section{变量命名} \label{variable-names}

\textbf{总述}

\begin{DWarn}
  变量 (包括函数参数) 和数据成员名一律使用驼峰命名。

  在D-Pointer的Private类中,成员变量不加任何修饰。

  在一般的类中,使用\cppinline{m_}开头来标记成员变量。
\end{DWarn}

\textbf{说明}

\subsection{普通变量命名}

举例:

\begin{cppcode}
  QString tableName;   // 接受,驼峰命名

  QString table_name;  // 不接受 - 用下划线.
  QString tablename;   // 不接受 - 全小写.
\end{cppcode}

\subsection{类数据成员}

不管是静态的还是非静态的,类数据成员都可以和普通变量一样,但是需要使用\cppinline{m_}前缀来修饰。

\begin{DWarn}
  为了实现较好的封装,Qt中大量使用D-Pointer技术,在这种情况下,一般通过\cppinline{d->localValue}的方式访问Private类的变量,这时候就不需要使用\cppinline{m_}来修饰成员变量。
  对于非D-Pointer的Private类,使用\cppinline{m_}前缀来修饰成员变量。
\end{DWarn}


\begin{cppcode}
  class TableInfo {
      ...
      private:
      QString m_tableName;               // 好
      static Pool<TableInfo>* m_pool;    // 好
    };

  class TableInfoPrivate {
      ...
      public:
      QString tableName;               // 好,Private类不需要任何修饰
      static Pool<TableInfo>* pool;    // 好,Private类不需要任何修饰
    };
\end{cppcode}

\subsection{结构体变量}

不管是静态的还是非静态的,结构体数据成员都可以和普通变量一样,不用像类那样接下划线:

\begin{cppcode}
  struct TableInfoData {
      QString tableName;               // 好,命名风格和Private保持一致
      static Pool<TableInfo>* pool;    // 好,命名风格和Private保持一致
    }
\end{cppcode}

结构体与类的使用讨论,参考 \DFullRef{structs-vs-classes}。

\section{常量命名} \label{constant-names}

\textbf{总述}

声明为 \cppinline{constexpr} 或 \cppinline{const} 的变量,或在程序运行期间其值始终保持不变的,命名时以 "k" 开头,大小写混合。例如:

\begin{cppcode}
  const int kDaysInAWeek = 7;
\end{cppcode}

\textbf{说明}

所有具有静态存储类型的变量 (例如静态变量或全局变量,参见 \href{http://en.cppreference.com/w/cpp/language/storage_duration#Storage_duration}{存储类型}) 都应当以此方式命名。对于其他存储类型的变量,如自动变量等,这条规则是可选的。如果不采用这条规则,就按照一般的变量命名规则。

\section{函数命名} \label{function-names}

\textbf{总述}

常规函数使用大小写混合,取值和设值函数则要求与变量名匹配: \cppinline{myExcitingFunction()},\cppinline{myExcitingMethod()},\cppinline{my_exciting_member_variable()},\cppinline{set_my_exciting_member_variable()}。

\textbf{说明}

一般来说,函数名首字母小写,每个单词首字母大写 (即 "驼峰变量名" 或 "帕斯卡变量名"),没有下划线。对于首字母缩写的单词,更倾向于将它们视作一个单词进行首字母大写 (例如,写作 \cppinline{startRpc()} 而非 \cppinline{startRPC()})。

\begin{cppcode}
  AddTableEntry()
  DeleteUrl()
  OpenFileOrDie()
\end{cppcode}

\begin{DWarn}
  对于DBus接口函数,属性,信号,确保首字母大写,这也适用于一些其他风格的IPC/RPC接口或代码生成器生成的接口,包括dbus/protobuf/thrift等。
\end{DWarn}

同样的命名规则同时适用于类作用域与命名空间作用域的常量,因为它们是作为 API 的一部分暴露对外的,因此应当让它们看起来像是一个函数,因为在这时,它们实际上是一个对象而非函数的这一事实对外不过是一个无关紧要的实现细节。

取值和设值函数的命名与变量一致。一般来说它们的名称与实际的成员变量对应,但并不强制要求。例如 \cppinline{int getCount()} 与 \cppinline{void setCount(int count)}。

\section{命名空间命名}

\textbf{总述}
\begin{DWarn}
  命名空间以大写字母命名。最高级命名空间的名字取决于项目名称。要注意避免嵌套命名空间的名字之间和常见的顶级命名空间的名字之间发生冲突.
\end{DWarn}

顶级命名空间的名称应当是项目名或者是该命名空间中的代码所属的团队的名字。命名空间中的代码,应当存放于和命名空间的名字匹配的文件夹或其子文件夹中.

注意 \DFullRef{general-naming-rules} 的规则同样适用于命名空间。命名空间中的代码极少需要涉及命名空间的名称,因此没有必要在命名空间中使用缩写.

要避免嵌套的命名空间与常见的顶级命名空间发生名称冲突。由于名称查找规则的存在,命名空间之间的冲突完全有可能导致编译失败。尤其是,不要创建嵌套的 \cppinline{std} 命名空间。建议使用更独特的项目标识符 (\cppinline{WebSearch::Index},\cppinline{WebSearch::IndexUtil}) 而非常见的极易发生冲突的名称 (比如 \cppinline{WebSearch::Util}).

对于 \cppinline{Internal} 命名空间,要当心加入到同一 \cppinline{internal} 命名空间的代码之间发生冲突 (由于内部维护人员通常来自同一团队,因此常有可能导致冲突)。在这种情况下,请使用文件名以使得内部名称独一无二 (例如对于 \cppinline{frobber.h},使用 \cppinline{WebSearch::Index::FrobberInternal})。

\section{枚举命名}

\textbf{总述}

枚举的命名应当和 \DFullRef{type-names} 一致: \cppinline{EnumName} 。

\textbf{说明}

单独的枚举值使用首字母大写的大小写混合命名方式。枚举名 \cppinline{UrlTableErrors} (以及 \cppinline{AlternateUrlTableErrors}) 是类型,所以要用大小写混合的方式.

\begin{cppcode}
  enum UrlTableErrors {
      OK = 0,
      ErrorOutOfMemory,
      ErrorMalformedInput,
    };
\end{cppcode}

2009 年 1 月之前,Google 一直建议采用\DFullRef{macro-names}的方式命名枚举值。由于枚举值和宏之间的命名冲突,直接导致了很多问题。由此,这里改为优先选择\DFullRef{type-names}的方式。新代码应该尽可能优先使用\DFullRef{type-names}的方式。但是老代码没必要切换到\DFullRef{type-names}的方式,除非宏风格确实会产生编译期问题。

\section{宏命名} \label{macro-names}

\textbf{总述}

你并不打算 \DFullRef{preprocessor-macros},对吧? 如果你一定要用,像这样命名:

\cppinline{MY_MACRO_THAT_SCARES_SMALL_CHILDREN}.

\textbf{说明}

参考 \DFullRef{preprocessor-macros}; 通常 *不应该* 使用宏。如果不得不用,其命名像枚举命名一样全部大写,使用下划线:

\begin{cppcode}
  #define ROUND(x) ...
  #define PI_ROUNDED 3.0
\end{cppcode}

\section{命名规则的特例}

\textbf{总述}

如果你命名的实体与已有 C/C++ 实体相似,可参考现有命名策略。

\begin{DWarn}
  如果是为了扩展STL的接口,或继承其他底层库的函数,则可以不受命名规则限制,以避免功能错误。
\end{DWarn}

\cppinline{bigopen()}: 函数名,参照 \cppinline{open()} 的形式

\cppinline{uint}: \cppinline{typedef}

\cppinline{bigpos}: \cppinline{struct} 或 \cppinline{class},参照 \cppinline{pos} 的形式

\cppinline{sparse_hash_map}: STL 型实体; 参照 STL 命名约定

\cppinline{LONGLONG_MAX}: 常量,如同 \cppinline{INT_MAX}

\section{注解}

\subsection{ 译者 (YuleFox) 笔记}

感觉 Google 的命名约定很高明,比如写了简单的类 QueryResult,接着又可以直接定义一个变量 \cppinline{query_result},区分度很好; 再次,类内变量以下划线结尾,那么就可以直接传入同名的形参,比如 \cppinline{TextQuery::TextQuery(std::string word) : word_(word) {}} ,其中 \cppinline{word_} 自然是类内私有成员.

\subsection{ deepin风格注解 }

对Qt风格同理:

比如写了简单的类 \cppinline{QueryResult},接着又可以直接定义一个变量 \cppinline{queryResult},区分度很好; 再次,类内变量以下划线结尾,那么就可以直接传入同名的形参,比如 \cppinline{TextQuery::TextQuery(QString word) : m_word(word) {}} ,其中 \cppinline{m_word} 自然是类内私有成员.

\input{qt/comments.tex}
% TODO(iceyer): use clang-format
% \input{qt/formatting.tex}
\chapter{规则特例}

前面说明的编程习惯基本都是强制性的。 但所有优秀的规则都允许例外, 这里就是探讨这些特例。

\section{现有不合规范的代码}

\textbf{总述}

对于现有不符合既定编程风格的代码可以网开一面。

\textbf{说明}

当你修改使用其他风格的代码时, 为了与代码原有风格保持一致可以不使用本指南约定。 如果不放心, 可以与代码原作者或现在的负责人员商讨。 记住, *一致性* 也包括原有的一致性。

\section{Windows 代码} \label{windows-code}

\textbf{总述}

Windows 程序员有自己的编程习惯, 主要源于 Windows 头文件和其它 Microsoft 代码。 我们希望任何人都可以顺利读懂你的代码, 所以针对所有平台的 C++ 编程只给出一个单独的指南。

\textbf{说明}

如果你习惯使用 Windows 编码风格, 这儿有必要重申一下某些你可能会忘记的指南:

% TODO(iceyer): naming and pragma once need to be discuss
\begin{itemize}
  \item  不要使用匈牙利命名法 (比如把整型变量命名成 \cppinline{iNum})。 使用 Google 命名约定, 包括对源文件使用 \cppinline{.cc} 扩展名。
  \item Windows 定义了很多原生类型的同义词 (YuleFox 注: 这一点, 我也很反感), 如 \cppinline{DWORD}, \cppinline{HANDLE} 等等。 在调用 Windows API 时这是完全可以接受甚至鼓励的。 即使如此, 还是尽量使用原有的 C++ 类型, 例如使用 \cppinline{const TCHAR *} 而不是 \cppinline{LPCTSTR}。
  \item 使用 Microsoft Visual C++ 进行编译时, 将警告级别设置为 3 或更高, 并将所有警告(warnings)当作错误(errors)处理。
  \item 不要使用 \cppinline{#pragma once}; 而应该使用 Google 的头文件保护规则。 头文件保护的路径应该相对于项目根目录 (Yang.Y 注: 如 \cppinline{#ifndef SRC_DIR_BAR_H_}, 参考 \DFullRef{qt-pragma-once-guard} 一节)。
  \item 除非万不得已, 不要使用任何非标准的扩展, 如 \cppinline{#pragma} 和 \cppinline{__declspec}。 使用 \cppinline{__declspec(dllimport)} 和 \cppinline{__declspec(dllexport)} 是允许的, 但必须通过宏来使用, 比如 \cppinline{DLLIMPORT} 和 \cppinline{DLLEXPORT}, 这样其他人在分享使用这些代码时可以很容易地禁用这些扩展。
\end{itemize}

然而, 在 Windows 上仍然有一些我们偶尔需要违反的规则:

\begin{itemize}
  \item 通常我们 \DFullRef{multiple-inheritance}, 但在使用 COM 和 ATL/WTL 类时可以使用多重继承。 为了实现 COM 或 ATL/WTL 类/接口, 你可能不得不使用多重实现继承。
  \item  虽然代码中不应该使用异常, 但是在 ATL 和部分 STL(包括 Visual C++ 的 STL) 中异常被广泛使用。 使用 ATL 时, 应定义 \cppinline{_ATL_NO_EXCEPTIONS} 以禁用异常。 你需要研究一下是否能够禁用 STL 的异常, 如果无法禁用, 可以启用编译器异常。 (注意这只是为了编译 STL, 自己的代码里仍然不应当包含异常处理)。
  \item  通常为了利用头文件预编译, 每个每个源文件的开头都会包含一个名为 \cppinline{StdAfx.h} 或 \cppinline{precompile.h} 的文件。 为了使代码方便与其他项目共享, 请避免显式包含此文件 (除了在 \cppinline{precompile.cc} 中), 使用 \cppinline{/FI} 编译器选项以自动包含该文件。
  \item  资源头文件通常命名为 \cppinline{resource.h} 且只包含宏, 这一文件不需要遵守本风格指南。
\end{itemize}
\input{qt/end.tex}


\end{document}
