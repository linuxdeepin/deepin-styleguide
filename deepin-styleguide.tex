\documentclass[UTF8,a4paper,oneside]{ctexbook}

\usepackage{indentfirst}
\setlength{\parindent}{2em}

\usepackage[margin=1in]{geometry}

\usepackage{xcolor}
\definecolor{deepin_blue}{HTML}{0088FD}
\definecolor{deepin_black}{HTML}{000000}
\definecolor{deepin_orange}{RGB}{244, 151, 0}

\usepackage{minted}

\newminted{cpp}{frame=lines,framerule=1pt,breaklines,breakanywhere}
\newmintinline{cpp}{breaklines,breakanywhere}

\newminted{ini}{frame=lines,framerule=1pt,breaklines,breakanywhere}
\newmintinline{ini}{breaklines,breakanywhere}

\usepackage{nameref}
\usepackage[colorlinks=true]{hyperref}

\newcommand*{\DFullRef}[1]{\hyperref[{#1}]{\ref*{#1} \nameref*{#1}}}

\usepackage{titlesec}
\titleformat{\chapter}[hang]
{\huge\bfseries}
{\thechapter\hspace{20pt}\textcolor{deepin_black}{|}\hspace{20pt}}{0pt}
{\huge\bfseries}

\usepackage{enumitem}

\makeatletter
\renewcommand{\section}{
  \@startsection{section}{1}{0mm}
  {0.5\baselineskip}{0.5\baselineskip}{\bf\leftline}
 }
\makeatother

\makeatletter
\newcommand*{\DBox}[1]{
\@makeother\#
\vspace{0.5\baselineskip}
\noindent\fbox{\parbox{\linewidth}{#1}}
}
\makeatother

\makeatletter
\newcommand*{\GWarn}[1]{
\@makeother\#
\noindent\fbox{\parbox{\linewidth}{#1}}
}
\makeatother

\usepackage{framed}
\usepackage{quoting}

\newenvironment{DNote}
{
    \color{deepin_blue}
    \bfseries
    \quoting[leftmargin=0pt, vskip=0pt,noorphans]
}
{
    \endquoting
}

\newenvironment{DWarn}
{
    \color{deepin_orange}
    \bfseries
    \quoting[leftmargin=0pt, vskip=0pt,noorphans]
}
{
    \endquoting
}

\usepackage{verbatim}

% 设置段落间距
\setlength{\parskip}{0.5em}
\makeatletter
\@addtoreset{chapter}{part}
\makeatother
\title{deepin 开源项目风格指南}

\date{\today}

\begin{document}
\maketitle

\textbf{声明}

本项目是在\href{https://github.com/zh-google-styleguide/zh-google-styleguide}{Google 开源项目风格指南——中文版}基础上修改而来。

本项目使用 \LaTeX 构建。

\textbf{Google 开源项目风格指南——中文版声明}

项目原始地址是:\href{https://github.com/zh-google-styleguide/zh-google-styleguide}{Google 开源项目风格指南}

如果你关注的是 Google 官方英文版,请移步 \href{https://github.com/google/styleguide}{Google Style Guide}

每个较大的开源项目都有自己的风格指南:关于如何为该项目编写代码的一系列约定(有时候会比较武断)。当所有代码均保持一致的风格,在理解大型代码库时更为轻松。

“风格”的含义涵盖范围广,从“变量使用驼峰格式(camelCase)”到“决不使用全局变量”再到“决不使用异常”,等等诸如此类。

英文版项目维护的是在 Google 使用的编程风格指南。如果你正在修改的项目源自 Google,你可能会被引导至英文版项目页面,以了解项目所使用的风格。

中文版项目采用 reStructuredText 纯文本标记语法,并使用 Sphinx 生成 HTML / CHM / PDF 等文档格式。

英文版项目还包含 \href{https://github.com/google/styleguide/tree/gh-pages/cpplint}{cpplint} ——一个用来帮助适应风格准则的工具,以及 \href{https://raw.githubusercontent.com/google/styleguide/gh-pages/google-c-style.el}{google-c-style.el},Google 风格的 Emacs 配置文件。

\tableofcontents
\newpage

\part{项目风格}

\chapter{命名约定}

在 \deepin 发行版本中,大约有 100 个左右的项目是 \deepin 来进行维护的。为了保障项目的统一性,这里对 \deepin 的总体命名进行一个阐述。

\section{通用名词} \label{general-naming-define}

通用名词是指由 \deepin 所持有的或主导的相关名词以及缩写。

通用名词在代码,文件名,文档中有不同的变体。每个名词都会分别说明。

\subsection{\deepin}

\textbf{总述}

\deepin 是指由 deepin.org 所有发行的发行版本。在指代发行版本时,应该永远使用小写的\deepin。

\begin{DWarn}
在 deepin 23 以后的版本中,deepin 将网站的主体迁移到 deepin.org 中,这将影响绝大部分的项目,特别是 DBus 接口部分。deepin 承诺在 2028 年之前保障旧接口还是可以使用的。
\end{DWarn}

\textbf{使用}

在文档,图片中,需要使用全小写的\deepin,即使是首字母,也应该使用小写。

\begin{cppcode}
  deepin is an opensoucre os.  // 正确

  Deepin is an opensoucre os.  // 错误,即使是段落首字母,也不应该大写
\end{cppcode}

在代码中,需要使用全小写的\deepin,除非代码风格规定了必须使用全大写,或首字母大小的情况。

\begin{cppcode}
  #define DEEPIN_MACRO XXXX     // 正确,以代码规范为准
  const int kDeepinNumber = 1;  // 正确,以代码规范为准

  // 版权信息中也需要使用小写的 deepin
  // * Copyright (c) 2021. deepin All rights reserved.
\end{cppcode}

在文件名中,需要使用全小写的\deepin。

\begin{cppcode}
  /usr/lib/deepin-daemon/dde-system-daemon  // 正确
  /usr/share/Deepin/msc/res                 // 错误,应该为 /usr/share/deepin/msc/res
\end{cppcode}

\textbf{例外}

当 \deepin 和其他名词组成专有名词时,可以使用大小写混合,例如:

\begin{inicode}
  # desktop 文件中,deepin-music 相关的应用
  [Desktop Entry]
  Name=Deepin Music
\end{inicode}

这里 \iniinline{Deepin Music}是一个专有名词,在任何情况下都不可以拆开使用。

\subsection{DDE}

\textbf{总述}

DDE 是 \iniinline{Deepin Desktop Environment} 的缩写。

\DBox{
  \iniinline{Deepin Desktop Environment}也是专有名词,不要拆开使用,也不要写成\iniinline{deepin Desktop Environment},\iniinline{deepin desktop environment}等形式。
}

\textbf{使用}

在文档,图片中,需要使用全大写的`DDE`。

\begin{cppcode}
  The DDE is comprised of the Desktop Environment, deepin Window Manager, Control Center, Launcher and Dock.    // 正确
  Use dde in other os.                  // 错误,文档中只有大写
  Login to Dde.                         // 错误,文档中不允许混合大小写
\end{cppcode}

在代码中,需要使用全大写的`DDE`,除非代码风格规定了必须使用全大写,或首字母大写的情况。

\begin{cppcode}
  #define DDE_MACRO XXXX     // 正确,以代码规范为准
  const int kDdeNumber = 1;  // 正确,以代码规范为准
\end{cppcode}

在文件名中,需要使用全小写的`dde`。

\begin{cppcode}
  /usr/lib/deepin-daemon/dde-system-daemon  // 正确
\end{cppcode}

\section{项目命名} \label{deepin-project-naming}

\textbf{总述}

\deepin 项目应该使用全小写的命名方式,单词使用\cppinline{-}进行连接。但是如果是应用型的项目,也可以使用倒置域名进行命名。

\textbf{使用}

\begin{cppcode}
  org.deepin.lianliankan // 倒置域名格式,应用必须使用该方式命名
  plymouth-theme-deepin  // 正确
  deepin-font-manager    // 正确

  Robot-Autotest         // 错误,不使用大写
\end{cppcode}

\section{文件命名} \label{deepin-file-naming}

\textbf{总述}

对于安装到系统中的文件,其命名方式和\DFullRef{deepin-project-naming}相同。同时需要满足 GNU/Linux 的通用命名风格。

\textbf{使用}

\begin{cppcode}
  /usr/bin/dde-dock  // 正确
  /usr/share/polkit-1/actions/com.deepin.pkexec.dde-file-manager.policy  // 正确

  /usr/share/DeepinAIAssistant/translations  // 错误,不使用大写
  /usr/lib/deepin-daemon/logViewerService    // 错误,log-view-service
  /usr/lib/deepin-daemon/backlight_helper    // 错误,backlight-helper
\end{cppcode}

\section{DBus 命名}

\textbf{总述}

DBus 命名是一个较为模糊的地带,我们根据官方的设计文档\href{https://dbus.freedesktop.org/doc/dbus-api-design.html}{D-Bus API Design Guidelines}来指导 DBus 的命名规则:

DBus 由服务名,路径,接口,方法(包括属性,信号等)四个部分组成。

对于服务名,路径,接口,其应该分解成域名,项目,组件三个部分。例如:

\begin{cppcode}
  org.deepin.Manual1.Search
  /org/deepin/Manual1/Search
  org.deepin.Manual1.Search
\end{cppcode}

其中 \cppinline{org.deepin} 是域名,Manual 是项目名称,1 是 API 版本号,Search 是组件名称。其中:

\begin{itemize}
  \item 域名:使用倒置域名方法,目前 deepin 使用的域名为 \cppinline{org.deepin},  \cppinline{org.desktopspec}。
  \item 项目名称:使用大小写混合方式。根据\href{https://dbus.freedesktop.org/doc/dbus-api-design.html}{D-Bus API Design Guidelines},这里需要带上版本号。
  \item 组件名称:如果一个项目提供多个服务,那么这里就需要有组件名称,组件名称使用大小写混合方式。
\end{itemize}

DBus 的方法(包括属性,信号等)永远使用大小写混合方式。

\textbf{使用}

注意,这里 org.freedesktop.portal 是域名,这也是 DBus 的接口风格中最让人迷惑的地方。

\begin{cppcode}
  org.freedesktop.portal.Desktop      // 正确,但是缺少 API 版本号
  /org/freedesktop/portal/Desktop     // 正确,但是缺少 API 版本号
  org.freedesktop.portal.Desktop      // 正确,但是缺少 API 版本号
\end{cppcode}

\begin{cppcode}
  org.deepin.DDE1.Accounts       // 正确
  /org/deepin/DDE1/Accounts      // 正确
  org.deepin.DDE1.Accounts       // 正确
  com.deepin.daemon.Accounts     // 错误,这是目前的命名方式,其中 daemon 意义不明确

  org.desktopspec.ConfigManager  // 正确,deepin 将通用的标准在 desktopsepc 组织中进行实现
\end{cppcode}


\textbf{备注}

按照这种命名方式,deepin V20 中 DDE 相关的绝大部分 DBus 接口需要重新设计。

\chapter{创建}

在 deepin 发行版本中,大约有 100 个左右的项目是 deepin 来进行维护的。为了避免项目冗余,这里对项目以及文件的创建进行一个阐述。

\section{项目创建}

\textbf{总述}

在新创建一个项目之前,应首先考虑 deepin 发行版已有的项目是否支持功能扩展,不推荐一味的进行项目新建。对于某个新特性必须创建项目时,其命名方式和\DFullRef{deepin-project-naming}相同。

\DBox{
	\iniinline{deepin-default-settings}、\iniinline{uos-config}同属系统通用配置文件管理项目,在后续的版本中,将废弃 uos-config 项目,使用 deepin-default-settings 进行统一管理。
}

\DBox{
	\iniinline{dde-wayland-config}、\iniinline{kwayland-data}同属 wayland 协议配置文件管理项目,推荐使用 dde-wayland-config 进行统一管理。
}

\textbf{备注}

按照项目创建规则,目前 deepin/DDE 相关的部分项目需要重新设计。

\section{文件创建}

\textbf{总述}

在新创建一个文件时,其命名方式和\DFullRef{deepin-file-naming}相同,文件路径遵循\href{https://refspecs.linuxfoundation.org/FHS_3.0/fhs-3.0.html}{Filesystem Hierarchy Standard}文件系统层次标准,推荐使用 Debian 规则中的 install 文件进行管理,减少使用代码直接进行文件创建。

\textbf{使用}

\begin{cppcode}
  /usr/lib/libudis86.so       // 正确
  /etc/os-version             // 正确
  /usr/bin/apt                // 正确
  /etc/qemu-ifup              // 错误,/etc 目录下存放配置文件,不能存放二进制可执行文件
  /usr/share/uos_resources    // 错误,该文件没有归属于任意项目
\end{cppcode}

\part{Qt代码风格}

\chapter{扉页}

\section{译者前言}

Google 经常会发布一些开源项目, 意味着会接受来自其他代码贡献者的代码. 但是如果代码贡献者的编程风格与 Google 的不一致,
会给代码阅读者和其他代码提交者造成不小的困扰. Google 因此发布了这份自己的编程风格指南, 使所有提交代码的人都能获知 Google
的编程风格.

翻译初衷:

\begin{itemize}
	\item 规则的作用就是避免混乱. 但规则本身一定要权威, 有说服力, 并且是理性的. 我们所见过的大部分编程规范, 其内容或不够严谨,   或阐述过于简单, 或带有一定的武断性.
	\item Google 保持其一贯的严谨精神, 5 万汉字的指南涉及广泛, 论证严密. 我们翻译该系列指南的主因也正是其严谨性.
\end{itemize}

严谨意味着指南的价值不仅仅局限于它罗列出的规范, 更具参考意义的是它为了列出规范而做的谨慎权衡过程.

\chapter{头文件}

通常每一个 \cppinline{.cpp} 文件都有一个对应的 \cppinline{.h} 文件. 也有一些常见例外, 如单元测试代码和只包含 \cppinline{main()} 函数的\cppinline{.cpp} 文件。

正确使用头文件可令代码在可读性、文件大小和性能上大为改观。

下面的规则将引导你规避使用头文件时的各种陷阱。

\section{Self-contained 头文件} \label{self-contained-headers}

\DBox {
	头文件应该能够自给自足(self-contained,也就是可以作为第一个头文件被引入),以 \cppinline{.h} 结尾。至于用来插入文本的文件,说到底它们并不是头文件,所以应以 \cppinline{.inc} 结尾。不允许分离出 \cppinline{-inl.h} 头文件的做法。
}

所有头文件要能够自给自足。换言之,用户和重构工具不需要为特别场合而包含额外的头文件。详言之,一个头文件要有\DFullRef{pragma-once-guard},统统包含它所需要的其它头文件,也不要求定义任何特别 symbols。

不过有一个例外,即一个文件并不是 self-contained 的,而是作为文本插入到代码某处。或者,文件内容实际上是其它头文件的特定平台(platform-specific)扩展部分。这些文件就要用\cppinline{.inc} 文件扩展名。

如果 \cppinline{.h} 文件声明了一个模板或内联函数,同时也在该文件加以定义。凡是有用到这些的 \cppinline{.cpp} 文件,就得统统包含该头文件,否则程序可能会在构建中链接失败。不要把这些定义放到分离的 \cppinline{-inl.h} 文件里(译者注:过去该规范曾提倡把定义放到 -inl.h 里过)。

有个例外:如果某函数模板为所有相关模板参数显式实例化,或本身就是某类的一个私有成员,那么它就只能定义在实例化该模板的 \cppinline{.cpp} 文件里。

\section{#define 保护} \label{pragma-once-guard}

\DBox{
  所有头文件都应该使用 \cppinline{#define} 来防止头文件被多重包含,命名格式当是:\cppinline{<PROJECT>_<PATH>_<FILE>_H_}。
}

为保证唯一性,头文件的命名应该基于所在项目源代码树的全路径。例如,项目 foo 中的头文件 foo/src/bar/baz.h 可按如下方式保护:

\begin{cppcode}
  #ifndef FOO_BAR_BAZ_H_
  #define FOO_BAR_BAZ_H_
  ...
  #endif // FOO_BAR_BAZ_H_
\end{cppcode}

\section{前置声明} \label{forward-declarations}

\DBox {
	尽可能地避免使用前置声明。使用 \cppinline{#include} 包含需要的头文件即可。
}

\textbf{定义:}

所谓「前置声明」(forward declaration)是类、函数和模板的纯粹声明,没伴随着其定义.

\textbf{优点:}

\begin{itemize}
	\item 前置声明能够节省编译时间,多余的 \cppinline{#include} 会迫使编译器展开更多的文件,处理更多的输入。
	\item 前置声明能够节省不必要的重新编译的时间。 \cppinline{#include} 使代码因为头文件中无关的改动而被重新编译多次。
\end{itemize}

\textbf{缺点:}

\begin{itemize}
	\item 前置声明隐藏了依赖关系,头文件改动时,用户的代码会跳过必要的重新编译过程。
	\item 前置声明可能会被库的后续更改所破坏。前置声明函数或模板有时会妨碍头文件开发者变动其 API。例如扩大形参类型,加个自带默认参数的模板形参等等。
	\item 前置声明来自命名空间 \cppinline{std::} 的 symbol 时,其行为未定义。
	\item 很难判断什么时候该用前置声明,什么时候该用 \cppinline{#include} 。极端情况下,用前置声明代替 \cppinline{#include} 甚至都会暗暗地改变代码的含义:

\begin{cppcode}
// b.h:
struct B {};
struct D : B {};

// good_user.cpp:
#include "b.h"
void f(B*);
void f(void*);
void test(D* x) { f(x); }  // calls f(B*)
\end{cppcode}

	      如果 \cppinline{#include} 被 \cppinline{B} 和 \cppinline{D} 的前置声明替代, \cppinline{test()} 就会调用 \cppinline{f(void*)} 。	\item 前置声明了不少来自头文件的 symbol 时,就会比单单一行的 \cppinline{include} 冗长。
	\item 仅仅为了能前置声明而重构代码(比如用指针成员代替对象成员)会使代码变得更慢更复杂.
\end{itemize}

\textbf{结论:}

\begin{itemize}
	\item 尽量避免前置声明那些定义在其他项目中的实体。
	\item 函数:总是使用 \cppinline{#include} 。
	\item 类模板:优先使用 \cppinline{#include} 。
\end{itemize}

至于什么时候包含头文件,参见 \DFullRef{name-and-order-of-includes} 。


\section{内联函数} \label{inline-functions}

\DBox {
	只有当函数只有 10 行甚至更少时才将其定义为内联函数。
}

\textbf{定义:}

当函数被声明为内联函数之后, 编译器会将其内联展开, 而不是按通常的函数调用机制进行调用。

\textbf{优点:}

只要内联的函数体较小, 内联该函数可以令目标代码更加高效. 对于存取函数以及其它函数体比较短, 性能关键的函数, 鼓励使用内联.

\textbf{缺点:}

滥用内联将导致程序变得更慢. 内联可能使目标代码量或增或减, 这取决于内联函数的大小. 内联非常短小的存取函数通常会减少代码大小,但内联一个相当大的函数将戏剧性的增加代码大小. 现代处理器由于更好的利用了指令缓存, 小巧的代码往往执行更快。

\textbf{结论:}

一个较为合理的经验准则是, 不要内联超过 10 行的函数. 谨慎对待析构函数, 析构函数往往比其表面看起来要更长,因为有隐含的成员和基类析构函数被调用!

另一个实用的经验准则: 内联那些包含循环或 \cppinline{switch} 语句的函数常常是得不偿失 (除非在大多数情况下, 这些循环或 \cppinline{switch} 语句从不被执行).

有些函数即使声明为内联的也不一定会被编译器内联, 这点很重要; 比如虚函数和递归函数就不会被正常内联. 通常,递归函数不应该声明成内联函数.(YuleFox 注: 递归调用堆栈的展开并不像循环那么简单, 比如递归层数在编译时可能是未知的,大多数编译器都不支持内联递归函数). 虚函数内联的主要原因则是想把它的函数体放在类定义内, 为了图个方便, 抑或是当作文档描述其行为,比如精短的存取函数.

\section{ include 的路径及顺序} \label{name-and-order-of-includes}

\DBox{
	使用标准的头文件包含顺序可增强可读性, 避免隐藏依赖: 相关头文件, C 库, C++ 库, 其他库的 `.h`, 本项目内的 `.h`.
}

项目内头文件应按照项目源代码目录树结构排列, 避免使用 UNIX 特殊的快捷目录 \cppinline{.} (当前目录) 或 \cppinline{..} (上级目录)。

例如, \cppinline{google-awesome-project/src/base/logging.h} 应该按如下方式包含:

\begin{cppcode}
	#include "base/logging.h"
\end{cppcode}

又如, \cppinline{dir/foo.cpp} 或 \cppinline{dir/foo_test.cpp} 的主要作用是实现或测试 \cppinline{dir2/foo2.h}的功能, \cppinline{foo.cpp} 中包含头文件的次序如下:

\begin{enumerate}
	\item \cppinline{dir2/foo2.h} (优先位置, 详情如下)
	\item C 系统文件
	\item C++ 系统文件
	\item 其他库的 \cppinline{.h} 文件
	\item 本项目内 \cppinline{.h} 文件
\end{enumerate}

这种优先的顺序排序保证当 \cppinline{dir2/foo2.h} 遗漏某些必要的库时, \cppinline{dir/foo.cpp} 或 \cppinline{dir/foo_test.cpp} 的构建会立刻中止。因此这一条规则保证维护这些文件的人们首先看到构建中止的消息而不是维护其他包的人们。

\cppinline{dir/foo.cpp} 和 \cppinline{dir2/foo2.h} 通常位于同一目录下,如\cppinline{base/basictypes_unittest.cpp} 和 \cppinline[breaklines]{base/basictypes.h}, 但也可以放在不同目录下.

按字母顺序分别对每种类型的头文件进行二次排序是不错的主意。注意较老的代码可不符合这条规则,要在方便的时候改正它们。

您所依赖的符号 (symbols) 被哪些头文件所定义,您就应该包含(include)哪些头文件,`前置声明`(forward declarations) 情况除外。比如您要用到 \cppinline{bar.h} 中的某个符号, 哪怕您所包含的 \cppinline{foo.h} 已经包含了\cppinline{bar.h}, 也照样得包含 \cppinline{bar.h}, 除非 \cppinline{foo.h} 有明确说明它会自动向您提供 \cppinline{bar.h} 中的symbol. 不过,凡是 cc 文件所对应的「相关头文件」已经包含的,就不用再重复包含进其 cc 文件里面了,就像 \cppinline{foo.cpp}只包含 \cppinline{foo.h} 就够了,不用再管后者所包含的其它内容。

举例来说, \cppinline{google-awesome-project/src/foo/internal/fooserver.cpp} 的包含次序如下:

\begin{cppcode}
	#include "foo/public/fooserver.h" // 优先位置

	#include <sys/types.h>
	#include <unistd.h>

	#include <hash_map>
	#include <vector>

	#include "base/basictypes.h"
	#include "base/commandlineflags.h"
	#include "foo/public/bar.h"
\end{cppcode}

\textbf{例外:}

有时,平台特定(system-specific)代码需要条件编译(conditional includes),这些代码可以放到其它 includes 之后。当然,您的平台特定代码也要够简练且独立,比如:

\begin{cppcode}
	#include "foo/public/fooserver.h"

	#include "base/port.h"  // For LANG_CXX11.

	#ifdef LANG_CXX11
	#include <initializer_list>
	#endif  // LANG_CXX11
\end{cppcode}


\section{注解}

\subsection{译者 (YuleFox) 笔记}

\begin{itemize}
	\item  避免多重包含是学编程时最基本的要求;
	\item  前置声明是为了降低编译依赖,防止修改一个头文件引发多米诺效应;
	\item  内联函数的合理使用可提高代码执行效率;
	\item  \cppinline{-inl.h} 可提高代码可读性 (一般用不到吧:D);
	\item  标准化函数参数顺序可以提高可读性和易维护性 (对函数参数的堆栈空间有轻微影响, 我以前大多是相同类型放在一起);
	\item  包含文件的名称使用 \cppinline{.} 和 \cppinline{..} 虽然方便却易混乱, 使用比较完整的项目路径看上去很清晰, 很条理,包含文件的次序除了美观之外, 最重要的是可以减少隐藏依赖, 使每个头文件在 "最需要编译" (对应源文件处 :D) 的地方编译,有人提出库文件放在最后, 这样出错先是项目内的文件, 头文件都放在对应源文件的最前面, 这一点足以保证内部错误的及时发现了.
\end{itemize}

\subsection{译者(acgtyrant)笔记}

\begin{itemize}
	\item  原来还真有项目用 \cppinline{#include} 来插入文本,且其文件扩展名 \cppinline{.inc} 看上去也很科学。
	\item  Google 已经不再提倡 \cppinline{-inl.h} 用法。
	\item  注意,前置声明的类是不完全类型(incomplete type),我们只能定义指向该类型的指针或引用,或者声明(但不能定义)以不完全类型作为参数或者返回类型的函数。毕竟编译器不知道不完全类型的定义,我们不能创建其类的任何对象,也不能声明成类内部的数据成员。
	\item  类内部的函数一般会自动内联。所以某函数一旦不需要内联,其定义就不要再放在头文件里,而是放到对应的 \cppinline{.cpp} 文件里。这样可以保持头文件的类相当精炼,也很好地贯彻了声明与定义分离的原则。
	\item  在 \cppinline{#include} 中插入空行以分割相关头文件, C 库, C++ 库, 其他库的 \cppinline{.h} 和本项目内的 \cppinline{.h} 是个好习惯。
\end{itemize}


\chapter{作用域}

\section{命名空间} \label{namespace}

\DBox{
  鼓励在 ``.cc`` 文件内使用匿名命名空间或 ``static`` 声明. 使用具名的命名空间时, 其名称可基于项目名或相对路径.
  禁止使用 using 指示(using-directive)。禁止使用内联命名空间(inline namespace)。
}

\textbf{定义:}

命名空间将全局作用域细分为独立的, 具名的作用域, 可有效防止全局作用域的命名冲突。

\textbf{优点:}

虽然类已经提供了(可嵌套的)命名轴线 (YuleFox 注: 将命名分割在不同类的作用域内), 命名空间在这基础上又封装了一层。

举例来说, 两个不同项目的全局作用域都有一个类 ``Foo``, 这样在编译或运行时造成冲突. 如果每个项目将代码置于不同命名空间中,
``project1::Foo`` 和 ``project2::Foo`` 作为不同符号自然不会冲突.

内联命名空间会自动把内部的标识符放到外层作用域,比如:

\begin{cppcode}
  namespace X {
      inline namespace Y {
          void foo();
        }  // namespace Y
    }  // namespace X
\end{cppcode}

``X::Y::foo()`` 与 ``X::foo()`` 彼此可代替。内联命名空间主要用来保持跨版本的 ABI 兼容性。

\textbf{缺点:}

命名空间具有迷惑性, 因为它们使得区分两个相同命名所指代的定义更加困难。

内联命名空间很容易令人迷惑,毕竟其内部的成员不再受其声明所在命名空间的限制。内联命名空间只在大型版本控制里有用。

有时候不得不多次引用某个定义在许多嵌套命名空间里的实体,使用完整的命名空间会导致代码的冗长。

在头文件中使用匿名空间导致违背 C++ 的唯一定义原则 (One Definition Rule (ODR))。

\textbf{结论:}

根据下文将要提到的策略合理使用命名空间。

\begin{itemize}
  \item 遵守 `命名空间命名 <naming.html#namespace-names>` 中的规则。
  \item 像之前的几个例子中一样,在命名空间的最后注释出命名空间的名字。
  \item 用命名空间把文件包含, `gflags <https://gflags.github.io/gflags/>` 的声明/定义, 以及类的前置声明以外的整个源文件封装起来, 以区别于其它命名空间:

        % \begin{noindent}
\begin{cppcode}
  // .h 文件
  namespace mynamespace {

      // 所有声明都置于命名空间中
      // 注意不要使用缩进
  class MyClass {
  public:
  ...
  void Foo();
  };
} // namespace mynamespace
\end{cppcode}
%\end{noindent}

        %\begin{noindent}
\begin{cppcode}
// .cc 文件
namespace mynamespace {

  // 函数定义都置于命名空间中
  void MyClass::Foo() {
  ...
  }

} // namespace mynamespace
\end{cppcode}
%\end{noindent}

        更复杂的 ``.cc`` 文件包含更多, 更复杂的细节, 比如 gflags 或 using 声明。

        %\begin{noindent}
\begin{cppcode}
  #include "a.h"

  DEFINE_FLAG(bool, someflag, false, "dummy flag");

  namespace a {

      ...code for a...// 左对齐

    } // namespace a
\end{cppcode}
% \end{noindent}

  \item 不要在命名空间 ``std`` 内声明任何东西, 包括标准库的类前置声明. 在 ``std`` 命名空间声明实体是未定义的行为,会导致如不可移植. 声明标准库下的实体, 需要包含对应的头文件。

  \item 不应该使用 *using 指示* 引入整个命名空间的标识符号。

        %\begin{noindent}
\begin{cppcode}
  // 禁止 —— 污染命名空间
  using namespace foo;
\end{cppcode}
% \end{noindent}

  \item  不要在头文件中使用 *命名空间别名* 除非显式标记内部命名空间使用。因为任何在头文件中引入的命名空间都会成为公开API的一部分。

        %\begin{noindent}
\begin{cppcode}
  // 在 .cc 中使用别名缩短常用的命名空间
  namespace baz = ::foo::bar::baz;
\end{cppcode}

\begin{cppcode}
// 在 .h 中使用别名缩短常用的命名空间
namespace librarian {
  namespace impl {  // 仅限内部使用
      namespace sidetable = ::pipeline_diagnostics::sidetable;
    }  // namespace impl

  inline void my_inline_function() {
    // 限制在一个函数中的命名空间别名
    namespace baz = ::foo::bar::baz;
    ...
  }
}  // namespace librarian
\end{cppcode}
% \end{noindent}

  \item  禁止用内联命名空间
\end{itemize}

\section{匿名命名空间和静态变量} \label{unnamed-namespace-and-static-variables}

\DBox {
  在 ``.cc`` 文件中定义一个不需要被外部引用的变量时,可以将它们放在匿名命名空间或声明为 ``static`` 。但是不要在
  ``.h`` 文件中这么做。
}

\textbf{定义:}

所有置于匿名命名空间的声明都具有内部链接性,函数和变量可以经由声明为 ``static`` 拥有内部链接性,这意味着你在这个文件中声明的这些标识符都不能在另一个文件中被访问。即使两个文件声明了完全一样名字的标识符,它们所指向的实体实际上是完全不同的。

\textbf{结论:}

推荐、鼓励在 ``.cc`` 中对于不需要在其他地方引用的标识符使用内部链接性声明,但是不要在 ``.h`` 中使用。

匿名命名空间的声明和具名的格式相同,在最后注释上 ``namespace`` :

%\begin{noindent}
\begin{cppcode}
      namespace {
      ...
      }  // namespace
\end{cppcode}
% \end{noindent}
\part{GoLang代码风格}

\chapter{扉页}

\section{前言}
风格一致的代码更合理,学习成本更少,且更容易维护。随着新的约定出现或者出现错误后更容易迁移、更新、修复 \texttt{bug} 。

相反,在一个代码库中包含多个完全不同或冲突的代码风格会导致维护成本增加、不确定性上升和部分认知偏差。所有这些都会直接导致速度降低、代码审查痛苦,并且会增加 \texttt{bug} 数量。

因此需要为代码库制定体套标准,目的是规范 \texttt{Go} 项目的开发,保持代码的一致性,使代码库易于管理和维护。

本规范主要基于 \texttt{Uber GoLang Style Guide} 进行编写,同时结合了工作中的实践,给出了提高性能和安全性的编码技巧。

\section{参考文献}
\begin{itemize}[leftmargin=4em]
  \item \href{https://github.com/uber-go/guide/blob/master/style.md}{Uber GoLang Style Guide}
  \item \href{https://github.com/golang-standards/project-layout}{Standard Go Project Layout}
  \item \href{https://golang.org/doc/effective\_go#mixed-caps}{函数命名规则}
  \item \href{https://github.com/golang/go/wiki/CodeReviewComments}{Go代码审查意见}
  \item \href{https://golang.org/doc/effective\_go}{Effective Go}
  \item \href{https://golang.org/ref/spec}{Go语言规范}
  \item \href{https://yougg.github.io/2017/06/12/go\%E8\%AF\%AD\%E8\%A8\%80\%E5\%AE\%89\%E5\%85\%A8\%E7\%BC\%96\%E7\%A8\%8B\%E8\%A7\%84\%E8\%8C\%83/}{Go语言安全编程规范}
  \item \href{https://gruntwork.io/guides/style\%20guides/golang-style-guide}{Gruntwork Go Style Guide}
\end{itemize}

\chapter{缩进}
\texttt{Go} 提供了 \texttt{gofmt} 命令用于格式化代码,因此所有代码文件都必须经过 \texttt{gofmt} 格式化。

建议在开发工具中配置文件保存时自动执行 \texttt{gofmt} ,以自动修复代码中的格式问题。

同时自动执行 \texttt{golint} 和 \texttt{go vet} ,以及时发现代码中的错误并修复。

\chapter{分组}
Go 语言支持将相似的声明放在一个组内,适用于常量、变量和类型声明,不要将不相关的声明放在一组。

\begin{itemize}[leftmargin=4em]
\item 错误用法

  \begin{minted}{go}
    const _STATE_SUCESS = 1
    const _STATE_FAILED = 2
  \end{minted}
\item 正确用法

  \begin{minted}{go}
    const (
    	_STATE_SUCESS = iota + 1
    	_STATE_FAILED
    )
  \end{minted}
\end{itemize}

分组使用的位置没有限制,例如:你可以在函数内部使用它们。
\begin{itemize}[leftmargin=4em]
\item 错误用法

  \begin{minted}{go}
    func calcCost() {
    	var total int
    	idx := 0
    }
  \end{minted}
\item 正确用法

  \begin{minted}{go}
    func calcCost() {
    	var (
    		total int
    		idx int
    	)
    }
  \end{minted}
\end{itemize}

\section{Import}
分组对 \texttt{import} 也同样适用,但标准库和第三方库需要隔开,例如:
\begin{itemize}[leftmargin=4em]
\item 错误用法

  \begin{minted}{go}
    import (
    	"os"
    	"os/exec"
    	"example.com/client-go"
    	"example.com/http"
    )
  \end{minted}
\item 正确用法

  \begin{minted}{go}
    import (
    	"os"
    	"os/exec"

    	"example.com/client-go"
    	"example.com/http"
    )
  \end{minted}
\end{itemize}

推荐使用 \texttt{goimports} 导入包,其会进行自动分组。

\section{函数}
函数在编写时,也应按照以下原则进行分组:
\begin{itemize}[leftmargin=4em]
\item 函数应在 \texttt{struct、const、var} 等定义的后面;
\item 导出的函数应先出现在文件中;
\item 相同接受者的函数应在一起;
\item 普通工具函数应在文件末尾;
\item 函数应按调用顺序排序。
\end{itemize}

如:
\begin{itemize}[leftmargin=4em]
\item 错误用法

  \begin{minted}{go}
    func (s *something) Cost() {
    	return calcCost(s.weights)
    }

    type something struct{ ... }

    func calcCost(n []int) int {...}

    func (s *something) Stop() {...}

    func newSomething() *something {
    	return &something{}
    }
  \end{minted}
\item 正确用法

  \begin{minted}{go}
    type something struct{ ... }

    func newSomething() *something {
    	return &something{}
    }

    func (s *something) Cost() {
    	return calcCost(s.weights)
    }

    func (s *something) Stop() {...}

    func calcCost(n []int) int {...}
  \end{minted}
\end{itemize}

\chapter{命名}
变量名和函数使用驼峰命名法,首字母小写,但如果需要导出,首字母则应该大写。名称应能准确表达其含义,且容易记忆。

代码中应避免使用全局变量,若要使用全局变量,并且无需对外导出时,则在变量名前添加 \texttt{'\_'} 。

\begin{itemize}[leftmargin=4em]
\item 错误用法

  \begin{minted}{go}
    var count int

    func calc_cost() {
    	var Total int
    }
  \end{minted}
\item 正确用法

  \begin{minted}{go}
    var _count int

    func calcCost() {
    	var total int
    }
  \end{minted}
\end{itemize}

\section{避免使用内置名称}
Go 语言规范概述了几个内置的,不应在 Go 项目中使用的名称标识 \texttt{predeclared identifiers} 。

根据上下文的不同,将这些标识符作为名称重复使用, 将在当前作用域(或任何嵌套作用域)中隐藏原始标识符,或者混淆代码。
在最好的情况下,编译器会报错;在最坏的情况下,这样的代码可能会引入潜在的、难以恢复的错误。
\begin{itemize}[leftmargin=4em]
\item 错误用法

  \begin{minted}[breaklines]{go}
    var error string
    // `error` 作用域隐式覆盖
    // or

    func handleErrorMessage(error string) {
    	// `error` 作用域隐式覆盖
    }

    type Foo struct {
    	// 虽然这些字段在技术上不构成阴影,但`error`或`string`字符串的重映射现在是不明确的。
    	error  error
    	string string
    }

    func (f Foo) Error() error {
    	// `error` 和 `f.error` 在视觉上是相似的
    	return f.error
    }

    func (f Foo) String() string {
    	// `string` and `f.string` 在视觉上是相似的
    	return f.string
    }
  \end{minted}
\item 正确用法

  \begin{minted}{go}
    var errorMessage string
    // `error` 指向内置的非覆盖
    // or

    func handleErrorMessage(msg string) {
    	// `error` 指向内置的非覆盖
    }
    type Foo struct {
    	// `error` and `string` 现在是明确的。
    	err error
    	str string
    }

    func (f Foo) Error() error {
    	return f.err
    }

    func (f Foo) String() string {
    	return f.str
    }
  \end{minted}
\end{itemize}

注意,编译器在使用预先分配的标识符时不会生成错误,但是诸如 \texttt{go vet} 之类的工具会正确地指出这些和其它情况下的隐式问题。

\section{常量}
常量因使用蛇形命名法,全局常量需要全部大写。无需对外导出的全局常量,须在常量名前添加 \texttt{'\_'} 。

\begin{itemize}[leftmargin=4em]
\item 错误用法

  \begin{minted}{go}
    const maxBufSize = 1024
    const Version = "1.0"
  \end{minted}
\item 正确用法

  \begin{minted}{go}
    const _MAX_BUF_SIZE = 1024
    const VERSION = "1.0"
  \end{minted}
\end{itemize}

\section{包名}
包名需满足以下规则:
\begin{itemize}[leftmargin=4em]
\item 全部小写,没有大写或下划线及连接符;
\item 包名应避免重复;
\item 简短且简洁,容易记忆;
\item 不用复数;
\item 避免使用 \texttt{common/util/shared/lib} 这类模糊不清的包名。
\end{itemize}

如:
\begin{itemize}[leftmargin=4em]
\item 错误用法

  \begin{minted}{go}
    package URL
    package urls
  \end{minted}
\item 正确用法

  \begin{minted}{go}
    package url
  \end{minted}
\end{itemize}

\chapter{注释}
使用 \texttt{'//'} 的语法添加注释,允许多行。注释推荐使用单独的行,在对应语句之上,避免使用行内注释。

所有对外导出的常量、变量、结构体、方法、接口等都应添加注释。

\begin{itemize}[leftmargin=4em]
\item 错误用法

  \begin{minted}[breaklines]{go}
    const (
    	STATE_SUCCESS = iota + 1 // Indicates that the task has been completed and successfully
    	STATE_FAILED
    )
  \end{minted}
\item 正确用法

  \begin{minted}{go}
    const (
    	// Indicates that the task has been completed and successfully.
    	STATE_SUCCESS = iota + 1
    	// Indicates that the task has been completed, but failure.
    	STATE_FAILED
    )
  \end{minted}
\end{itemize}

\section{TODO 注释}
具体参见 \textbf{Qt} 中的 \textbf{TODO 注释}规则。

\chapter{Import}
包导入时应避免使用别名,但以下情况必须使用别名:
\begin{itemize}[leftmargin=4em]
\item 包名称与导入路径的最后一个元素不匹配;
\item 包名称与已有的包重复。
\end{itemize}

如:
\begin{minted}[xleftmargin=3.5em]{go}
  import (
  	"net/http"

  	myhttp "example.com/http"
  	client "example.com/client-go"
  )
\end{minted}

别名的命名规范与包名一致。

\chapter{变量}
\section{变量声明}
使用标准 \texttt{var} 关键字。请勿指定类型,除非它与表达式的类型不同。
\begin{itemize}[leftmargin=4em]
\item 错误用法

  \begin{minted}{go}
    var _s string = F()

    func F() string { return "A" }
  \end{minted}
\item 正确用法

  \begin{minted}{go}
    // 由于 F 已经明确了返回一个字符串类型,因此我们没有必要显式指定 _s 的类型
    var _s = F()

    func F() string { return "A" }
  \end{minted}
\end{itemize}

如果表达式的类型与所需的类型不完全匹配,请指定类型。如:
\begin{minted}[xleftmargin=3.5em]{go}
  type myError struct{}

  func (myError) Error() string { return "error" }

  func F() myError { return myError{} }

  var _e error = F()
\end{minted}

如果将变量明确设置为某个值,则应使用短变量声明形式 (\texttt{:=})。
\begin{itemize}[leftmargin=4em]
\item 错误用法

  \begin{minted}{go}
    var s = "foo"
  \end{minted}
\item 正确用法

  \begin{minted}{go}
    s := "foo"
  \end{minted}
\end{itemize}

但是,在某些情况下, \texttt{var} 使用关键字时默认值会更清晰。例如,声明空切片。
\begin{itemize}[leftmargin=4em]
\item 错误用法

  \begin{minted}{go}
    func f(list []int) {
    	filtered := []int{}
    	for _, v := range list {
    		if v > 10 {
    			filtered = append(filtered, v)
    		}
    	}
    }
  \end{minted}
\item 正确用法

  \begin{minted}{go}
    func f(list []int) {
    	var filtered []int
    	for _, v := range list {
    		if v > 10 {
    			filtered = append(filtered, v)
    		}
    	}
    }
  \end{minted}
\end{itemize}

\section{缩小变量作用域}
如果有可能,尽量缩小变量作用范围。除非它与减少嵌套的规则冲突。
\begin{itemize}[leftmargin=4em]
\item 错误用法

  \begin{minted}{go}
    err := ioutil.WriteFile(name, data, 0644)
    if err != nil {
    	return err
    }
  \end{minted}
\item 正确用法

  \begin{minted}{go}
    if err := ioutil.WriteFile(name, data, 0644); err != nil {
    	return err
    }
  \end{minted}
\end{itemize}

如果需要在 if 之外使用函数调用的结果,则不应尝试缩小范围。
\begin{itemize}[leftmargin=4em]
\item 错误用法

  \begin{minted}{go}
    if data, err := ioutil.ReadFile(name); err == nil {
    	err = cfg.Decode(data)
    	if err != nil {
    		return err
    	}

    	fmt.Println(cfg)
    	return nil
    } else {
    	return err
    }
  \end{minted}
\item 正确用法

  \begin{minted}{go}
    data, err := ioutil.ReadFile(name)
    if err != nil {
    	return err
    }

    if err := cfg.Decode(data); err != nil {
    	return err
    }

    fmt.Println(cfg)
    return nil
  \end{minted}
\end{itemize}

\section{使用原始字符串字面值}
\texttt{Go} 支持使用原始字符串字面值,也就是 '`' 来表示原生字符串,在需要转义的场景下,我们应该尽量使用这种方案来替换。

可以跨越多行并包含引号。使用这些字符串可以避免更难阅读的手工转义的字符串。

\begin{itemize}[leftmargin=4em]
\item 错误用法

  \begin{minted}{go}
    wantError := "unknown name:\"test\""
  \end{minted}
\item 正确用法

  \begin{minted}{go}
    wantError := `unknown error:"test"`
  \end{minted}
\end{itemize}

\section{避免可变全局变量}
使用选择依赖注入方式避免改变全局变量;既适用于函数指针又适用于其它值类型。
\begin{itemize}[leftmargin=4em]
\item 错误用法

  \begin{minted}{go}
    // sign.go
    var _timeNow = time.Now
    func sign(msg string) string {
    	now := _timeNow()
    	return signWithTime(msg, now)
    }


    // sign_test.go
    func TestSign(t *testing.T) {
    	oldTimeNow := _timeNow
    	_timeNow = func() time.Time {
    		return someFixedTime
    	}
    	defer func() { _timeNow = oldTimeNow }()
    	assert.Equal(t, want, sign(give))
    }
  \end{minted}
\item 正确用法

  \begin{minted}{go}
    // sign.go
    type signer struct {
    	now func() time.Time
    }
    func newSigner() *signer {
    	return &signer{
    		now: time.Now,
    	}
    }
    func (s *signer) Sign(msg string) string {
    	now := s.now()
    	return signWithTime(msg, now)
    }


    // sign_test.go
    func TestSigner(t *testing.T) {
    	s := newSigner()
    	s.now = func() time.Time {
    		return someFixedTime
    	}
    	assert.Equal(t, want, s.Sign(give))
    }
  \end{minted}
\end{itemize}

\chapter{Enum}
\textbf{枚举从 1 开始。}

在 \texttt{Go} 中引入枚举的标准方法是声明一个自定义类型和一个使用了 \texttt{iota} 的 \texttt{const} 组。
由于变量的默认值为 0,因此通常应以非零值开头枚举。
\begin{itemize}[leftmargin=4em]
\item 错误用法

  \begin{minted}{go}
    type Operation int

    const (
    	Add Operation = iota
    	Subtract
    	Multiply
    )

    // Add=0, Subtract=1, Multiply=2
  \end{minted}
\item 正确用法

  \begin{minted}{go}
    type Operation int

    const (
    	Add Operation = iota + 1
    	Subtract
    	Multiply
    )

    // Add=1, Subtract=2, Multiply=3
  \end{minted}
\end{itemize}

在某些情况下,使用零值是有意义的(枚举从零开始),例如,当零值是理想的默认行为时。
\begin{minted}[xleftmargin=3.5em]{go}
type LogOutput int

const (
	LogToStdout LogOutput = iota
	LogToFile
	LogToRemote
)

// LogToStdout=0, LogToFile=1, LogToRemote=2
\end{minted}

\chapter{Map}
对于空 \texttt{map} 请使用 \texttt{make(..)} 初始化, 并且 \texttt{map} 是通过编程方式填充的。
这使得 \texttt{map} 初始化在表现上不同于声明,并且它还可以方便地在 \texttt{make} 后添加大小提示。
\begin{itemize}[leftmargin=4em]
\item 错误用法

  \begin{minted}{go}
    // 声明和初始化看起来非常相似的
    var (
    	// m1 读写安全;
    	// m2 在写入时会 panic
    	m1 = map[T1]T2{}
    	m2 map[T1]T2
    )
  \end{minted}
\item 正确用法

  \begin{minted}{go}
    // 声明和初始化看起来差别非常大
    var (
    	// m1 读写安全;
    	// m2 在写入时会 panic
    	m1 = make(map[T1]T2)
    	m2 map[T1]T2
    )
  \end{minted}
\end{itemize}

尽可能在初始化时提供 \texttt{map} 容量大小。如果 \texttt{map} 包含固定的元素列表,
则使用 \texttt{map literals}(\texttt{map} 初始化列表)初始化映射。
\begin{itemize}[leftmargin=4em]
\item 错误用法

  \begin{minted}{go}
    m := make(map[T1]T2, 3)
    m[k1] = v1
    m[k2] = v2
    m[k3] = v3
  \end{minted}
\item 正确用法

  \begin{minted}{go}
    m := map[T1]T2{
      k1: v1,
      k2: v2,
      k3: v3,
    }
  \end{minted}
\end{itemize}

基本准则是:在初始化时使用 \texttt{map} 初始化列表来添加一组固定的元素。
否则使用 \texttt{make}(如果可以,尽量指定 \texttt{map} 容量)。

\section{指定容器容量}
尽可能指定容器容量,以便为容器预先分配内存。这将在添加元素时最小化后续分配(通过复制和调整容器大小)。

指定 \texttt{Map} 容量提示,尽可能在使用 \texttt{make()} 初始化的时候提供容量信息。
\begin{minted}[xleftmargin=3.5em]{go}
  make(map[T1]T2, hint)
\end{minted}

向 \texttt{make()} 提供容量提示会在初始化时尝试调整 \texttt{map} 的大小,这将减少在将元素添加到 \texttt{map} 时为 \texttt{map} 重新分配内存。
与 \texttt{slices} 不同, \texttt{map capacity} 提示并不保证完全的抢占式分配,而是用于估计所需的 \texttt{hashmap bucket} 的数量。 因此,在将元素添加到 \texttt{map} 时,甚至在指定 \texttt{map} 容量时,仍可能发生分配。
\begin{itemize}[leftmargin=4em]
\item 错误用法

  \begin{minted}{go}
    // m 是在没有大小提示的情况下创建的; 在运行时可能会有更多分配
    m := make(map[string]os.FileInfo)

    files, _ := ioutil.ReadDir("./files")
    for _, f := range files {
    	m[f.Name()] = f
    }
  \end{minted}
\item 正确用法

  \begin{minted}{go}
    // m 是有大小提示创建的;在运行时可能会有更少的分配
    files, _ := ioutil.ReadDir("./files")

    m := make(map[string]os.FileInfo, len(files))
    for _, f := range files {
    	m[f.Name()] = f
    }
  \end{minted}
\end{itemize}

在尽可能的情况下,在使用 \texttt{make()} 初始化切片时提供容量信息,特别是在追加切片时。
\begin{minted}[xleftmargin=3.5em]{go}
  make([]T, length, capacity)
\end{minted}

与 \texttt{maps} 不同, \texttt{slice capacity} 不是一个提示:
编译器将为提供给 \texttt{make()} 的 \texttt{slice} 的容量分配足够的内存,
这意味着后续的 \texttt{append()} 操作将导致零分配(直到 \texttt{slice} 的长度与容量匹配,
在此之后,任何 \texttt{append} 都可能调整大小以容纳其它元素)。

\chapter{Channel}
\textbf{channel 的 size 要么是 1,要么是无缓冲的。}

\texttt{channel} 通常 \texttt{size} 应为 \texttt{1} 或是无缓冲的。默认情况下,\texttt{channel} 是无缓冲的,其 \texttt{size} 为零。
任何其它尺寸都必须经过严格的审查,确定是否是通道边界,竞态条件,以及逻辑上下文导致需要 \texttt{size} 大于 \texttt{1} ,尽量从源头进行排查。
\begin{itemize}[leftmargin=4em]
\item 错误用法

  \begin{minted}{go}
    // 应该足以满足任何情况!
    c := make(chan int, 64)
  \end{minted}
\item 正确用法

  \begin{minted}{go}
    // 大小:1
    c := make(chan int, 1) // 或者
    // 无缓冲 channel,大小为 0
    c := make(chan int)
  \end{minted}
\end{itemize}

\section{禁止重复释放channel}
重复释放一般存在于异常流程判断中,如果恶意攻击者构造出异常条件使程序重复释放 \texttt{channel} ,则会触发运行时恐慌,从而造成 \texttt{DoS} 攻击。

\begin{itemize}[leftmargin=4em]
\item 错误用法

  下面代码中多次关掉channel会触发运行时错误。
  \begin{minted}{go}
    func foo(c chan int) {
    	defer close(c)
    	err := processBusiness()
    	if err != nil {
    		c <- 0
    		close(c) // 【错误】重复释放channel
    		return
    	}
    	c <- 1
    }
  \end{minted}
\item 正确用法

  使用defer延迟关闭channel,并且确保channel只释放一次。
  \begin{minted}{go}
    func foo(c chan int) {
    	defer close(c) // 【修改】使用defer延迟关闭channel
    	err := processBusiness()
    	if err != nil {
    		c <- 0
    		return
    	}
    	c <- 1
    }
  \end{minted}
\end{itemize}

\chapter{Slice}
\section{nil 是一个有效的 slice}
\texttt{nil} 是一个有效的长度为 \texttt{0} 的 \texttt{slice} ,这意味着,不应明确返回长度为零的切片。
应该返回 \texttt{nil} 来代替。
\begin{itemize}[leftmargin=4em]
\item 错误用法

  \begin{minted}{go}
    if x == "" {
    	return []int{}
    }
  \end{minted}
\item 正确用法

  \begin{minted}{go}
    if x == "" {
    	return nil
    }
  \end{minted}
\end{itemize}

要检查切片是否为空,请始终使用 \texttt{len(s) == 0} 。而非 \texttt{nil} 。
\begin{itemize}[leftmargin=4em]
\item 错误用法

  \begin{minted}{go}
    func isEmpty(s []string) bool {
    	return s == nil
    }
  \end{minted}
\item 正确用法

  \begin{minted}{go}
    func isEmpty(s []string) bool {
    	return len(s) == 0
    }
  \end{minted}
\end{itemize}

零值切片可立即使用,无需调用 \texttt{make} 创建。
\begin{itemize}[leftmargin=4em]
\item 错误用法

  \begin{minted}{go}
    nums := []int{}
    // or, nums := make([]int)

    if add1 {
    	nums = append(nums, 1)
    }

    if add2 {
    	nums = append(nums, 2)
    }
  \end{minted}
\item 正确用法

  \begin{minted}{go}
    var nums []int

    if add1 {
    	nums = append(nums, 1)
    }

    if add2 {
    	nums = append(nums, 2)
    }
  \end{minted}
\end{itemize}

记住,虽然 \texttt{nil} 切片是有效的切片,但它不等于长度为 \texttt{0} 的切片(一个为 \texttt{nil} ,另一个不是),
并且在不同的情况下(例如序列化),这两个切片的处理方式可能不同。

\section{追加时优先指定切片容量}
在尽可能的情况下,在初始化要追加的切片时为make()提供一个容量值。
\begin{itemize}[leftmargin=4em]
\item 错误用法

  \begin{minted}{go}
    for n := 0; n < b.N; n++ {
    	data := make([]int, 0)
    	for k := 0; k < size; k++{
    		data = append(data, k)
    	}
    }
  \end{minted}
\item 正确用法

  \begin{minted}{go}
    for n := 0; n < b.N; n++ {
    	data := make([]int, 0, size)
    	for k := 0; k < size; k++{
    		data = append(data, k)
    	}
    }
  \end{minted}
\end{itemize}

\chapter{Interface}
\section{interface 合理性验证}
在编译时验证接口的符合性,包括:
\begin{itemize}[leftmargin=4em]
\item 将实现特定接口的导出类型作为接口API 的一部分进行检查;
\item 实现同一接口的(导出和非导出)类型属于实现类型的集合;
\item 任何违反接口合理性检查的场景;都会终止编译,并通知给用户。
\end{itemize}

补充:上面3条是编译器对接口的检查机制, 使错误使用接口在编译期报错。 所以可以利用这个机制让部分问题在编译期暴露。

如果 \texttt{\*Handler} 与 \texttt{http.Handler} 的接口不匹配,
那么语句 \texttt{var \_ http.Handler = (\*Handler)(nil)} 将无法编译通过。

\begin{itemize}[leftmargin=4em]
\item 错误用法

  \begin{minted}{go}
    // 如果Handler没有实现http.Handler,会在运行时报错
    type Handler struct {
    // ...
    }
    func (h *Handler) ServeHTTP(
    	w http.ResponseWriter,
    	r *http.Request,
    ) {
    //...
    }
  \end{minted}
\item 正确用法

  \begin{minted}{go}
    type Handler struct {
    	// ...
    }
    // 用于触发编译期的接口的合理性检查机制
    // 如果Handler没有实现http.Handler,会在编译期报错
    var _ http.Handler = (*Handler)(nil)
    func (h *Handler) ServeHTTP(
    	w http.ResponseWriter,
    	r *http.Request,
    ) {
    	// ...
    }
  \end{minted}
\end{itemize}

赋值的右边应该是断言类型的零值。 对于指针类型(如 \texttt{\*Handler})、切片和映射,这是 \texttt{nil}; 对于结构类型,这是空结构。

\begin{minted}[xleftmargin=3.5em]{go}
  type LogHandler struct {
  	h   http.Handler
  	log *zap.Logger
  }
  var _ http.Handler = LogHandler{}
  func (h LogHandler) ServeHTTP(
  	w http.ResponseWriter,
  	r *http.Request,
  ) {
  	// ...
  }
\end{minted}

\section{指向 interface 的指针}
通常用不到指向接口类型的指针,应该将接口作为值进行传递,在这样的传递过程中,实质上传递的底层数据仍然可以是指针。

接口实质上在底层用两个字段表示:
\begin{itemize}[leftmargin=4em]
\item 一个指向某些特定类型信息的指针。可以将其视为 "type" ;
\item 数据指针。如果存储的数据是指针,则直接存储。如果存储的数据是一个值,则存储指向该值的指针。
\end{itemize}

如果希望接口方法修改基础数据,则必须使用指针传递(将对象指针赋值给接口变量)。
\begin{minted}[xleftmargin=3.5em]{go}
  type F interface {
  	f()
  }

  type S1 struct{}

  func (s S1) f() {}

  type S2 struct{}

  func (s *S2) f() {}

  // f1.f()无法修改底层数据
  // f2.f() 可以修改底层数据,给接口变量f2赋值时使用的是对象指针
  var f1 F = S1{}
  var f2 F = &S2{}
\end{minted}

\chapter{结构体}
\section{结构体嵌入}
嵌入类型(例如 \texttt{mutex} )应位于结构体内的字段列表的顶部,并且必须有一个空行将嵌入式字段与常规字段分隔开。
\begin{itemize}[leftmargin=4em]
\item 错误用法

  \begin{minted}{go}
    type Client struct {
    	version int
    	http.Client
    }
  \end{minted}
\item 正确用法

  \begin{minted}{go}
    type Client struct {
    	http.Client

    	version int
    }
  \end{minted}
\end{itemize}

内嵌应该提供切实的好处,比如以语义上合适的方式添加或增强功能。 它应该在对用户没有任何不利影响的情况下使用。

嵌入不应该:
\begin{itemize}[leftmargin=4em]
\item 纯粹是为了美观或方便;
\item 使外部类型更难构造或使用;
\item 影响外部类型的零值。如果外部类型有一个有用的零值,则在嵌入内部类型之后应该仍然有一个有用的零值;
\item 作为嵌入内部类型的副作用,从外部类型公开不相关的函数或字段;
\item 公开未导出的类型;
\item 影响外部类型的复制形式;
\item 更改外部类型的API或类型语义;
\item 嵌入内部类型的非规范形式;
\item 公开外部类型的实现详细信息;
\item 允许用户观察或控制类型内部;
\item 通过包装的方式改变内部函数的一般行为,这种包装方式会给用户带来一些意料之外情况。
\end{itemize}

简单地说,做到有意识和有目的嵌入。一种很好的测试体验是,“是否所有这些导出的内部方法、字段都将直接添加到外部类型”如果答案是 \texttt{some} 或 \texttt{no} ,不要嵌入内部类型,而是使用字段。
\begin{itemize}[leftmargin=4em]
\item 错误用法

  \begin{minted}{go}
    type A struct {
    	// Bad: A.Lock() and A.Unlock() 现在可用
    	// 不提供任何功能性好处,并允许用户控制有关A的内部细节。
    	sync.Mutex
    }

    type Book struct {
    	// Bad: 指针更改零值的有用性
    	io.ReadWriter
    	// other fields
    }
    // later
    var b Book
    b.Read(...)  // panic: nil pointer
    b.String()   // panic: nil pointer
    b.Write(...) // panic: nil pointer
    type Client struct {
    	sync.Mutex
    	sync.WaitGroup
    	bytes.Buffer
    	url.URL
    }
  \end{minted}
\item 正确用法

  \begin{minted}{go}
    type countingWriteCloser struct {
    	// Good: Write() 在外层提供用于特定目的,
    	// 并且委托工作到内部类型的Write()中。
    	io.WriteCloser
    	count int
    }
    func (w *countingWriteCloser) Write(bs []byte) (int, error) {
    	w.count += len(bs)
    	return w.WriteCloser.Write(bs)
    }
    type Book struct {
    	// Good: 有用的零值
    	bytes.Buffer
    	// other fields
    }
    // later
    var b Book
    b.Read(...)  // ok
    b.String()   // ok
    b.Write(...) // ok
    type Client struct {
    	mtx sync.Mutex
    	wg  sync.WaitGroup
    	buf bytes.Buffer
    	url url.URL
    }
  \end{minted}
\end{itemize}

\section{使用字段名初始化结构体}
初始化结构体时,应该指定字段名称。
省略结构中的零值字段,初始化具有字段名的结构时,除非提供有意义的上下文,否则忽略值为零的字段。
\begin{itemize}[leftmargin=4em]
\item 错误用法

  \begin{minted}{go}
    k := User{"John", "Doe", Admin: false}
  \end{minted}
\item 正确用法

  \begin{minted}{go}
    k := User{
    	FirstName: "John",
    	LastName: "Doe",
    }
  \end{minted}
\end{itemize}

这有助于通过省略该上下文中的默认值来减少阅读的障碍。只指定有意义的值。

在字段名提供有意义上下文的地方包含零值。例如,表驱动测试中的测试用例可以受益于字段的名称,即使它们是零值的。
\begin{minted}[xleftmargin=3.5em]{go}
  tests := []struct{
  	give string
  	want int
  }{
  	give: "0",
  	want: 0,
  	// ...
  }
\end{minted}

对零值结构使用 \texttt{var} ,如果在声明中省略了结构的所有字段,请使用 \texttt{var} 声明结构。
\begin{itemize}[leftmargin=4em]
\item 错误用法

  \begin{minted}{go}
    user := User{}
  \end{minted}
\item 正确用法

  \begin{minted}{go}
    var user User
  \end{minted}
\end{itemize}

这将零值结构与那些具有类似于为{初始化 \texttt{Maps}} 创建的、区别于非零值字段的结构区分开来。
初始化 \texttt{struct} 引用,在初始化结构引用时,请使用 \texttt{\&T\{\}} 代替 \texttt{new(T)} ,以使其与结构体初始化一致。
\begin{itemize}[leftmargin=4em]
\item 错误用法

  \begin{minted}{go}
    sval := T{Name: "foo"}

    // inconsistent
    sptr := new(T)
    sptr.Name = "bar"
  \end{minted}
\item 正确用法

  \begin{minted}{go}
    sval := T{Name: "foo"}

    sptr := &T{Name: "bar"}
  \end{minted}
\end{itemize}

\chapter{控制语句}
代码应通过尽可能先处理错误情况和特殊情况,并尽早返回或继续循环来减少嵌套。减少嵌套多个级别的代码的代码量。
\begin{itemize}[leftmargin=4em]
\item 错误用法

  \begin{minted}{go}
    for _, v := range data {
    	if v.F1 == 1 {
    		v = process(v)
    		if err := v.Call(); err == nil {
    			v.Send()
    		} else {
    			return err
    		}
    	} else {
    		log.Printf("Invalid v: %v", v)
    	}
    }
  \end{minted}
\item 正确用法

  \begin{minted}{go}
    for _, v := range data {
    	if v.F1 != 1 {
    		log.Printf("Invalid v: %v", v)
    		continue
    	}

    	v = process(v)
    	if err := v.Call(); err != nil {
    		return err
    	}
    	v.Send()
    }
  \end{minted}
\end{itemize}

去掉不必要的 \texttt{else} ,如果在 \texttt{if} 的两个分支中都设置了变量,则可以将其替换为单个 \texttt{if} 。
\begin{itemize}[leftmargin=4em]
\item 错误用法

  \begin{minted}{go}
    var a int
    if b {
    	a = 100
    } else {
    	a = 10
    }
  \end{minted}
\item 正确用法

  \begin{minted}{go}
    a := 10
    if b {
    	a = 100
    }
  \end{minted}
\end{itemize}

\chapter{Defer}
使用 \texttt{defer} 释放资源,诸如文件和锁。
\begin{itemize}[leftmargin=4em]
\item 错误用法

  \begin{minted}{go}
    p.Lock()
    if p.count < 10 {
    	p.Unlock()
    	return p.count
    }

    p.count++
    newCount := p.count
    p.Unlock()

    return newCount

    // 当有多个 return 分支时,很容易遗忘 unlock
  \end{minted}
\item 正确用法

  \begin{minted}{go}
    p.Lock()
    defer p.Unlock()

    if p.count < 10 {
    	return p.count
    }

    p.count++
    return p.count

    // 更可读
  \end{minted}
\end{itemize}

\texttt{defer} 的开销非常小,除非证明函数执行时间处于纳秒级的程度,才应避免这样做。
使用 \texttt{defer} 提升可读性是值得的,只有微不足道的成本。
尤其适用于那些不仅仅是简单内存访问的较大的方法,在这些方法中其它计算的资源消耗远超过 \texttt{defer} 。

\chapter{Error}
\texttt{Go} 中有多种声明错误(Error) 的选项:
\begin{itemize}[leftmargin=4em]
\item \texttt{errors.New} 对于简单静态字符串的错误;
\item \texttt{fmt.Errorf} 用于格式化的错误字符串;
\item 实现 \texttt{Error()} 方法的自定义类型;
\item 用 \texttt{"pkg/errors".Wrap} 的 \texttt{Wrapped errors} 。
\end{itemize}

返回错误时,考虑以下因素以确定最佳选择:
\begin{itemize}[leftmargin=4em]
\item 这是一个不需要额外信息的简单错误吗?如果是这样, \texttt{errors.New} 足够了;
\item 客户需要检测并处理此错误吗?如果是这样,则应使用自定义类型并实现该 \texttt{Error()} 方法;
\item 当前是否正在传播下游函数返回的错误?如果是这样,使用错误包装;
\item 否则 \texttt{fmt.Errorf} 就可以了。
\end{itemize}

如果客户端需要检测错误,并且已使用创建了一个简单的错误 \texttt{errors.New} ,请使用一个错误变量。
\begin{itemize}[leftmargin=4em]
\item 错误用法

  \begin{minted}{go}
    // package foo

    func Open() error {
    	return errors.New("could not open")
    }


    // package bar

    func use() {
    	if err := foo.Open(); err != nil {
    		if err.Error() == "could not open" {
    			// handle
    		} else {
    			panic("unknown error")
    		}
    	}
    }
  \end{minted}
\item 正确用法

  \begin{minted}{go}
    // package foo

    var ErrCouldNotOpen = errors.New("could not open")

    func Open() error {
    	return ErrCouldNotOpen
    }

    // package bar

    if err := foo.Open(); err != nil {
    	if errors.Is(err, foo.ErrCouldNotOpen) {
    		// handle
    	} else {
    		panic("unknown error")
    	}
    }
  \end{minted}
\end{itemize}

如果有可能需要客户端检测的错误,并且想向其中添加更多信息(例如,它不是静态字符串),则应使用自定义类型。
\begin{itemize}[leftmargin=4em]
\item 错误用法

  \begin{minted}{go}
    func open(file string) error {
    	return fmt.Errorf("file %q not found", file)
    }

    func use() {
    	if err := open("testfile.txt"); err != nil {
    		if strings.Contains(err.Error(), "not found") {
    			// handle
    		} else {
    			panic("unknown error")
    		}
    	}
    }
  \end{minted}
\item 正确用法

  \begin{minted}{go}
    type errNotFound struct {
    	file string
    }

    func (e errNotFound) Error() string {
    	return fmt.Sprintf("file %q not found", e.file)
    }

    func open(file string) error {
    	return errNotFound{file: file}
    }

    func use() {
    	if err := open("testfile.txt"); err != nil {
    		if _, ok := err.(errNotFound); ok {
    			// handle
    		} else {
    			panic("unknown error")
    		}
    	}
    }
  \end{minted}
\end{itemize}

直接导出自定义错误类型时要小心,因为它们已成为程序包公共 API 的一部分。最好公开匹配器功能以检查错误。
\begin{minted}[xleftmargin=3.5em]{go}
  // package foo

  type errNotFound struct {
  	file string
  }

  func (e errNotFound) Error() string {
  	return fmt.Sprintf("file %q not found", e.file)
  }

  func IsNotFoundError(err error) bool {
  	_, ok := err.(errNotFound)
  	return ok
  }

  func Open(file string) error {
  	return errNotFound{file: file}
  }


  // package bar

  if err := foo.Open("foo"); err != nil {
  	if foo.IsNotFoundError(err) {
  		// handle
  	} else {
  		panic("unknown error")
  	}
  }
\end{minted}

\section{Error Wrapping}
一个(函数/方法)调用失败时,有三种主要的传播错误方式:
\begin{itemize}[leftmargin=4em]
\item 如果没有要添加的其它上下文,并且要维护原始错误类型,则返回原始错误;
\item 添加上下文,使用 \texttt{"pkg/errors".Wrap} 以便错误消息提供更多上下文, \texttt{"pkg/errors".Cause} 可用于提取原始错误;
\item 如果调用者不需要检测或处理的特定错误情况,使用 \texttt{fmt.Errorf} 。
\end{itemize}

建议在可能的地方添加上下文,用来获得诸如“调用服务 foo:连接被拒绝”之类的更有用的错误,而不是诸如“连接被拒绝”之类的模糊错误。

在将上下文添加到返回的错误时,请避免使用“failed to”之类的短语以保持上下文简洁,
这些短语会陈述明显的内容,并随着错误在堆栈中的渗透而逐渐堆积:
\begin{itemize}[leftmargin=4em]
\item 错误用法

  \begin{minted}{go}
    s, err := store.New()
    if err != nil {
    	return fmt.Errorf("failed to create new store: %v", err)
    }
  \end{minted}
\item 正确用法

  \begin{minted}{go}
    s, err := store.New()
    if err != nil {
    	return fmt.Errorf("new store: %v", err)
    }
  \end{minted}
\end{itemize}

但是,一旦将错误发送到另一个系统,就应该明确消息是错误消息(例如使用err标记,或在日志中以”Failed”为前缀)。

\section{处理类型断言失败}
\texttt{type assertion} 的单个返回值形式针对不正确的类型将产生 \texttt{panic} 。因此,始终使用“comma ok”的惯用法。
\begin{itemize}[leftmargin=4em]
\item 错误用法

  \begin{minted}{go}
    t := i.(string)
  \end{minted}
\item 正确用法

  \begin{minted}{go}
    t, ok := i.(string)
    if !ok {
    	// 优雅地处理错误
    }
  \end{minted}
\end{itemize}

\chapter{Panic}
在生产环境中运行的代码必须避免出现 \texttt{panic} 。 \texttt{panic} 是 \texttt{cascading failures} 级联失败的主要根源 。
如果发生错误,该函数必须返回错误,并允许调用方决定如何处理它。
\begin{itemize}[leftmargin=4em]
\item 错误用法

  \begin{minted}{go}
    func run(args []string) {
    	if len(args) == 0 {
    		panic("an argument is required")
    	}
    	// ...
    }

    func main() {
    	run(os.Args[1:])
    }
  \end{minted}
\item 正确用法

  \begin{minted}{go}
    func run(args []string) error {
    	if len(args) == 0 {
    		return errors.New("an argument is required")
    	}
    	// ...
    	return nil
    }

    func main() {
    	if err := run(os.Args[1:]); err != nil {
    		fmt.Fprintln(os.Stderr, err)
    		os.Exit(1)
    	}
    }
  \end{minted}
\end{itemize}

\texttt{panic/recover} 不是错误处理策略。仅当发生不可恢复的事情(例如: \texttt{nil} 引用)时,程序才必须 \texttt{panic} 。
程序初始化是一个例外:程序启动时应使程序中止的不良情况可能会引起 \texttt{panic} 。
\begin{minted}[xleftmargin=3.5em,breaklines]{go}
var _statusTemplate = template.Must(template.New("name").Parse("_statusHTML"))
\end{minted}

即使在测试代码中,也优先使用 \texttt{t.Fatal} 或者 \texttt{t.FailNow} 而不是 \texttt{panic} 来确保失败被标记。
\begin{itemize}[leftmargin=4em]
\item 错误用法

  \begin{minted}{go}
    // func TestFoo(t *testing.T)

    f, err := ioutil.TempFile("", "test")
    if err != nil {
    	panic("failed to set up test")
    }
  \end{minted}
\item 正确用法

  \begin{minted}{go}
    // func TestFoo(t *testing.T)

    f, err := ioutil.TempFile("", "test")
    if err != nil {
    	t.Fatal("failed to set up test")
    }
  \end{minted}
\end{itemize}

\chapter{Printf-style}
如果你在函数外声明 \texttt{Printf-style} 函数的格式字符串,请将其设置为 \texttt{const} 常量。
这有助于 \texttt{go vet} 对格式字符串执行静态分析。
\begin{itemize}[leftmargin=4em]
\item 错误用法

  \begin{minted}{go}
    msg := "unexpected values %v, %v\n"
    fmt.Printf(msg, 1, 2)
  \end{minted}
\item 正确用法

  \begin{minted}{go}
    const msg = "unexpected values %v, %v\n"
    fmt.Printf(msg, 1, 2)
  \end{minted}
\end{itemize}

声明 \texttt{Printf-style} 函数时,确保 \texttt{go vet} 可以检测到它并检查格式字符串。
尽可能使用预定义的 \texttt{Printf-style} 函数名称, \texttt{go vet} 将默认进行检查。

如果不能使用预定义的名称,则以 \texttt{f} 结束选择的名称: \texttt{Wrapf} ,而不是 \texttt{Wrap} 。
\texttt{go vet} 可以要求检查特定的 \texttt{Printf} 样式名称,但名称必须以 \texttt{f} 结尾。

\chapter{函数}
\section{避免参数语义不明确}
函数调用中的意义不明确的参数可能会损害可读性。当参数名称的含义不明显时,请为参数添加 C 样式注释 (\texttt{/* ... */})
\begin{itemize}[leftmargin=4em]
\item 错误用法

  \begin{minted}{go}
    // func printInfo(name string, isLocal, done bool)

    printInfo("foo", true, true)
  \end{minted}
\item 正确用法

  \begin{minted}{go}
    // func printInfo(name string, isLocal, done bool)

    printInfo("foo", true /* isLocal */, true /* done */)
  \end{minted}
\end{itemize}

对于上面的示例代码,还有一种更好的处理方式是将上面的 \texttt{bool} 类型换成自定义类型。
将来,该参数可以支持不仅仅局限于两个状态(\texttt{true/false})。
\begin{minted}[xleftmargin=3.5em]{go}
type Region int

const (
	UnknownRegion Region = iota
	Local
)

type Status int

const (
	StatusReady Status= iota + 1
	StatusDone
	// Maybe we will have a StatusInProgress in the future.
)

func printInfo(name string, region Region, status Status)
\end{minted}

优先使用此种方式。

\section{最小职责原则}
函数实现时应遵循最小职责原则,尽量使函数的功能单一且简洁,避免多种功能揉合。
\begin{itemize}[leftmargin=4em]
\item 错误用法

  \begin{minted}{go}
    // Don't do this
    func main() {
    	fmt.Println(mulOfSums(1, 1))
    }

    func mulOfSums(a, b int) int {
    	return (a + b) * (a + b)
    }
  \end{minted}
\item 正确用法

  \begin{minted}{go}
    // Do this instead
    func main() {
    	fmt.Println(mul(add(1, 1), add(1, 1)))
    }

    func add(a, b int) int {
    	return a + b
    }

    func mul(a, b int) int {
    	return a * b
    }
  \end{minted}
\end{itemize}

\section{避免使用 init()}
尽可能避免使用 \texttt{init()} 。当 \texttt{init()} 是不可避免或可取的,代码应先尝试:
\begin{enumerate}[leftmargin=4em]
\item 无论程序环境或调用如何,都要完全确定;
\item 避免依赖于其它 \texttt{init()} 函数的顺序或副作用。虽然 \texttt{init()} 顺序是明确的,但代码可以更改, 因此 \texttt{init()} 函数之间的关系可能会使代码变得脆弱和容易出错;
\item 避免访问或操作全局或环境状态,如机器信息、环境变量、工作目录、程序参数/输入等;
\item 避免I/O,包括文件系统、网络和系统调用。
\end{enumerate}

不能满足这些要求的代码可能属于要作为 \texttt{main()} 调用的一部分(或程序生命周期中的其它地方), 或者作为 \texttt{main()} 本身的一部分写入。
特别是,打算由其它程序使用的库应该特别注意完全确定性, 而不是执行“init magic”。

\begin{itemize}[leftmargin=4em]
\item 错误用法

  \begin{minted}{go}
    type Foo struct {
    	// ...
    }
    var _defaultFoo Foo
    func init() {
    	_defaultFoo = Foo{
    		// ...
    	}
    }

    type Config struct {
    	// ...
    }
    var _config Config
    func init() {
    	// Bad: 基于当前目录
    	cwd, _ := os.Getwd()
    	// Bad: I/O
    	raw, _ := ioutil.ReadFile(
    		path.Join(cwd, "config", "config.yaml"),
    	)
    	yaml.Unmarshal(raw, &_config)
    }
  \end{minted}
\item 正确用法

  \begin{minted}{go}
    var _defaultFoo = Foo{
    	// ...
    }
    // or, 为了更好的可测试性:
    var _defaultFoo = defaultFoo()
    func defaultFoo() Foo {
    	return Foo{
    		// ...
    	}
    }
    type Config struct {
    	// ...
    }
    func loadConfig() Config {
    	cwd, err := os.Getwd()
    	// handle err
    	raw, err := ioutil.ReadFile(
    		path.Join(cwd, "config", "config.yaml"),
    	)
    	// handle err
    	var config Config
    	yaml.Unmarshal(raw, &config)
    	return config
    }
  \end{minted}
\end{itemize}

考虑到上述情况,在某些情况下, \texttt{init()} 可能更可取或是必要的,可能包括:
\begin{itemize}[leftmargin=4em]
\item 不能表示为单个赋值的复杂表达式;
\item 可插入的钩子,如 \texttt{database/sql} 、编码类型注册表等;
\item 对 \texttt{Google Cloud Functions} 和其它形式的确定性预计算的优化。
\end{itemize}

\section{主函数退出方式(Exit)}
Go 程序使用 \texttt{os.Exit} 或者 \texttt{log.Fatal\*} 立即退出 (使用 \texttt{panic} 不是退出程序的好方法)。

仅在 \texttt{main()} 中调用其中一个 \texttt{os.Exit} 或者 \texttt{log.Fatal\*} 。
所有其它函数应将错误通过返回值返回,并进行处理。
\begin{itemize}[leftmargin=4em]
\item 错误用法

  \begin{minted}{go}
    func main() {
    	body := readFile(path)
    	fmt.Println(body)
    }
    func readFile(path string) string {
    	f, err := os.Open(path)
    	if err != nil {
    		log.Fatal(err)
    	}
    	b, err := ioutil.ReadAll(f)
    	if err != nil {
    		log.Fatal(err)
    	}
    	return string(b)
    }
  \end{minted}
\item 正确用法

  \begin{minted}{go}
    func main() {
    	body, err := readFile(path)
    	if err != nil {
    		log.Fatal(err)
    	}
    	fmt.Println(body)
    }
    func readFile(path string) (string, error) {
    	f, err := os.Open(path)
    	if err != nil {
    		return "", err
    	}
    	b, err := ioutil.ReadAll(f)
    	if err != nil {
    		return "", err
    	}
    	return string(b), nil
    }
  \end{minted}
\end{itemize}

原则上,具有多种功能的程序退出存在一些问题:
\begin{itemize}[leftmargin=4em]
\item 不明显的控制流:任何函数都可以退出程序,因此很难对控制流进行推理;
\item 难以测试:退出程序的函数也将退出调用它的测试。这使得函数很难测试,并引入了跳过 \texttt{go test} 尚未运行的其它测试的风险;
\item 跳过清理:当函数退出程序时,会跳过已经进入 \texttt{defer} 队列里的函数调用。这增加了跳过重要清理任务的风险。
\end{itemize}

一次性退出原则,在 \texttt{main()} 函数中最多一次调用 \texttt{os.Exit} 或者 \texttt{log.Fatal} 。
如果有多个错误场景停止程序执行,请将该逻辑放在单独的函数下并从中返回错误。
这会缩短 \texttt{main()} 函数,并将所有关键业务逻辑放入一个单独的、可测试的函数中。

\begin{itemize}[leftmargin=4em]
\item 错误用法

  \begin{minted}{go}
    package main
    func main() {
    	args := os.Args[1:]
    	if len(args) != 1 {
    		log.Fatal("missing file")
    	}
    	name := args[0]
    	f, err := os.Open(name)
    	if err != nil {
    		log.Fatal(err)
    	}
    	defer f.Close()
    	// 如果我们调用log.Fatal 在这条线之后
    	// f.Close 将会被执行.
    	b, err := ioutil.ReadAll(f)
    	if err != nil {
    		log.Fatal(err)
    	}
    	// ...
    }
  \end{minted}
\item 正确用法

  \begin{minted}{go}
    package main
    func main() {
    	if err := run(); err != nil {
    		log.Fatal(err)
    	}
    }

    func run() error {
    	args := os.Args[1:]
    	if len(args) != 1 {
    		return errors.New("missing file")
    	}
    	name := args[0]
    	f, err := os.Open(name)
    	if err != nil {
    		return err
    	}
    	defer f.Close()
    	b, err := ioutil.ReadAll(f)
    	if err != nil {
    		return err
    	}
    	// ...
    }
  \end{minted}
\end{itemize}

\chapter{项目布局}
除了编码风格需要保持一致外,项目的布局也应该保持一致。项目目录应按以下要求进行组织:

\begin{itemize}[leftmargin=4em]
\item 通用目录

  \begin{itemize}
  \item \texttt{/cmd}

    项目的主干。

    每个应用程序的目录名应该与你想要的可执行文件的名称相匹配(例如, \texttt{/cmd/myapp} )。

    不要在这个目录中放置太多代码。如果你认为代码可以导入并在其它项目中使用,那么它应该位于 \texttt{/pkg} 目录中。
    如果代码不是可重用的,或者你不希望其它人重用它,请将该代码放到 \texttt{/internal} 目录中。你会惊讶于别人会怎么做,所以要明确你的意图!

    通常有一个小的 \texttt{main} 函数,从 \texttt{/internal} 和 \texttt{/pkg} 目录导入和调用代码,除此之外没有别的东西。

  \item \texttt{/internal}

    私有应用程序和库代码。这是你不希望其它人在其应用程序或库中导入代码。请注意,这个布局模式是由 Go 编译器本身执行的。
    注意,你并不局限于顶级 internal 目录。在项目树的任何级别上都可以有多个内部目录。

    你可以选择向 \texttt{internal} 包中添加一些额外的结构,以分隔共享和非共享的内部代码。
    这不是必需的(特别是对于较小的项目),但是最好有有可视化的线索来显示预期的包的用途。

    你的实际应用程序代码可以放在 \texttt{/internal/app} 目录下(例如 \texttt{/internal/app/myapp}),
    这些应用程序共享的代码可以放在 \texttt{/internal/pkg} 目录下(例如 \texttt{/internal/pkg/myprivlib})。

  \item \texttt{/pkg}

    外部应用程序可以使用的库代码(例如 \texttt{/pkg/mypubliclib})。其它项目会导入这些库,希望它们能正常工作,所以不建议在此目录存放别的东西。
    注意, \texttt{internal} 目录是确保私有包不可导入的更好方法,因为它是由 Go 强制执行的。

    \texttt{/pkg} 目录仍然是一种很好的方式,可以显式地表示该目录中的代码对于其它人来说是安全使用的好方法。
    由 Travis Jeffery  撰写的 \href{https://travisjeffery.com/b/2019/11/i-ll-take-pkg-over-internal/}{I'll take pkg over internal} 博客文章提供了 \texttt{pkg} 和 \texttt{internal} 目录的一个很好的概述,以及什么时候使用它们是有意义的。

    如果你的应用程序项目真的很小,并且额外的嵌套并不能增加多少价值,
    那就不要使用它。当它变得足够大时,你的根目录会变得非常繁琐时(尤其是当你有很多非 Go 应用组件时),请考虑一下。

  \item \texttt{/vendor}

    应用程序依赖项(手动管理或使用你喜欢的依赖项管理工具,如新的内置 \texttt{Go Modules} 功能)。
    \texttt{go mod vendor} 命令将为你创建 \texttt{/vendor} 目录。
    请注意,如果未使用默认情况下处于启用状态的 \texttt{Go 1.14} ,则可能需要在 \texttt{go build} 命令中添加 \texttt{-mod=vendor} 标志。

    如果你正在构建一个库,那么不要提交你的应用程序依赖项。

    七牛云有维护专门的代理模块功能 \href{https://github.com/goproxy/goproxy.cn/blob/master/README.zh-CN.md}{模块代理} 。

  \item \texttt{/configs}

    配置文件模板或默认配置。

    将你的 \texttt{confd} 或 \texttt{consul-template} 模板文件放在这里。

  \item \texttt{/init}

    System init(systemd,upstart,sysv)和 process manager/supervisor(runit,supervisor)配置。

  \item \texttt{/scripts}

    执行各种构建、安装、分析等操作的脚本。

    这些脚本保持了根级别的 Makefile 变得小而简单(例如, \href{https://github.com/hashicorp/terraform/blob/master/Makefile}{terraform Makefile} )。

  \item \texttt{/build}

    打包和持续集成。

    将你的云( AMI )、容器( Docker )、操作系统( deb、rpm、pkg )包配置和脚本放在 \texttt{/build/package} 目录下。

    将你的 CI (travis、circle、drone)配置和脚本放在 \texttt{/build/ci} 目录中。请注意,有些 CI 工具(例如 Travis CI)对配置文件的位置非常挑剔。
    尝试将配置文件放在 \texttt{/build/ci} 目录中,将它们链接到 CI 工具期望它们的位置(如果可能的话)。

  \item \texttt{/deployments}

    IaaS、PaaS、系统和容器编排部署配置和模板(docker-compose、kubernetes/helm、mesos、terraform、bosh)。
    注意,在一些存储库中(特别是使用 kubernetes 部署的应用程序),这个目录被称为 \texttt{/deploy} 。

  \item \texttt{/test}

    额外的外部测试应用程序和测试数据。你可以随时根据需求构造 \texttt{/test} 目录。对于较大的项目,有一个数据子目录是有意义的。
    例如,你可以使用 \texttt{/test/data} 或 \texttt{/test/testdata} (如果你需要忽略目录中的内容)。
    请注意,Go 还会忽略以“.”或“\_”开头的目录或文件,因此在如何命名测试数据目录方面有更大的灵活性。
  \end{itemize}
\item 服务应用程序目录

  \begin{itemize}
  \item \texttt{/api}

    OpenAPI/Swagger 规范,JSON 模式文件,协议定义文件。
  \end{itemize}
\item Web 应用程序目录

  \begin{itemize}
  \item \texttt{/web}

    特定于 Web 应用程序的组件:静态 Web 资产、服务器端模板和 SPAs。
  \end{itemize}
\item 其它目录

  \begin{itemize}
  \item \texttt{/docs}

    设计和用户文档(除了 \texttt{godoc} 生成的文档之外)。

  \item \texttt{/tools}

    这个项目的支持工具。注意,这些工具可以从 \texttt{/pkg} 和 \texttt{/internal} 目录导入代码。

  \item \texttt{/examples}

    你的应用程序和/或公共库的示例。

  \item \texttt{/third\_party}

    外部辅助工具,分叉代码和其它第三方工具(例如 Swagger UI)。

  \item \texttt{/githooks}

    Git hooks。

  \item \texttt{/assets}

    与存储库一起使用的其它资产(图像、徽标等)。

  \item \texttt{/website}

    如果你不使用 Github 页面,则在这里放置项目的网站数据。
  \end{itemize}
\end{itemize}

项目中推荐使用 \texttt{Go Modules} ,除非你有特定的理由不使用它。

\chapter{经验技巧}
\section{优先使用 strconv}
将数字、布尔值转换为字符串或从字符串转换时, \texttt{strconv} 速度比 \texttt{fmt} 快。
\begin{itemize}[leftmargin=4em]
\item 错误用法

  \begin{minted}{go}
    for i := 0; i < b.N; i++ {
    	s := fmt.Sprint(rand.Int())
    }
  \end{minted}
\item 正确用法

  \begin{minted}{go}
    for i := 0; i < b.N; i++ {
    	s := strconv.Itoa(rand.Int())
    }
  \end{minted}
\end{itemize}

\section{避免字符串到字节的转换}
不要反复从固定字符串创建字节 \texttt{slice} 。相反,请执行一次转换并捕获结果。
\begin{itemize}[leftmargin=4em]
\item 错误用法

  \begin{minted}{go}
    for i := 0; i < b.N; i++ {
    	w.Write([]byte("Hello world"))
    }
  \end{minted}
\item 正确用法

  \begin{minted}{go}
    data := []byte("Hello world")
    for i := 0; i < b.N; i++ {
    	w.Write(data)
    }
  \end{minted}
\end{itemize}

\section{零值 Mutex 是有效的}
零值 \texttt{sync.Mutex} 和 \texttt{sync.RWMutex} 是有效的。所以指向 \texttt{mutex} 的指针基本是不必要的。
\begin{itemize}[leftmargin=4em]
\item 错误用法

  \begin{minted}{go}
    mu := new(sync.Mutex)
    mu.Lock()
  \end{minted}
\item 正确用法

  \begin{minted}{go}
    var mu sync.Mutex
    mu.Lock()
  \end{minted}
\end{itemize}

如果你使用结构体指针, \texttt{mutex} 应该作为结构体的非指针字段。即使该结构体不被导出,也不要直接把 \texttt{mutex} 嵌入到结构体中。
\begin{itemize}[leftmargin=4em]
\item 错误用法

  \begin{minted}{go}
    // Mutex 字段, Lock 和 Unlock 方法是 SMap 导出的 API 中不刻意说明的一部分
    type SMap struct {
    	sync.Mutex

    	data map[string]string
    }

    func NewSMap() *SMap {
    	return &SMap{
    		data: make(map[string]string),
    	}
    }

    func (m *SMap) Get(k string) string {
    	m.Lock()
    	defer m.Unlock()

    	return m.data[k]
    }
  \end{minted}
\item 正确用法

  \begin{minted}{go}
    // mutex 及其方法是 SMap 的实现细节,对其调用者不可见
    type SMap struct {
    	mu sync.Mutex

    	data map[string]string
    }

    func NewSMap() *SMap {
    	return &SMap{
    		data: make(map[string]string),
    	}
    }

    func (m *SMap) Get(k string) string {
    	m.mu.Lock()
    	defer m.mu.Unlock()

    	return m.data[k]
    }
  \end{minted}
\end{itemize}

\section{在边界处拷贝 Slices 和 Maps}
\texttt{Slices} 和 \texttt{Maps} 包含了指向底层数据的指针,因此在需要复制它们时要特别注意。

接收 \texttt{Slices} 和 \texttt{Maps} 需注意,当 \texttt{map} 或 \texttt{slice} 作为函数参数传入时,如果存储了对它们的引用,则用户可以对其进行修改。
\begin{itemize}[leftmargin=4em]
\item 错误用法

  \begin{minted}{go}
    func (d *Driver) SetTrips(trips []Trip) {
    	d.trips = trips
    }

    trips := ...
    d1.SetTrips(trips)

    // 你是要修改 d1.trips 吗?
    trips[0] = ...
  \end{minted}
\item 正确用法

  \begin{minted}{go}
    func (d *Driver) SetTrips(trips []Trip) {
    	d.trips = make([]Trip, len(trips))
    	copy(d.trips, trips)
    }

    trips := ...
    d1.SetTrips(trips)

    // 这里我们修改 trips[0],但不会影响到 d1.trips
    trips[0] = ...
  \end{minted}
\end{itemize}

返回 \texttt{Slices} 或 \texttt{Maps} 同样需注意,用户对暴露内部状态的 \texttt{map} 或 \texttt{slice} 可修改。
\begin{itemize}[leftmargin=4em]
\item 错误用法

  \begin{minted}{go}
    type Stats struct {
    	mu sync.Mutex

    	counters map[string]int
    }



    // Snapshot 返回当前状态。
    func (s *Stats) Snapshot() map[string]int {
    	s.mu.Lock()
    	defer s.mu.Unlock()

    	return s.counters
    }

    // snapshot 不再受互斥锁保护
    // 因此对 snapshot 的任何访问都将受到数据竞争的影响
    // 影响 stats.counters
    snapshot := stats.Snapshot()
  \end{minted}
\item 正确用法

  \begin{minted}{go}
    type Stats struct {
    	mu sync.Mutex

    	counters map[string]int
    }

    func (s *Stats) Snapshot() map[string]int {
    	s.mu.Lock()
    	defer s.mu.Unlock()

    	result := make(map[string]int, len(s.counters))
    	for k, v := range s.counters {
    		result[k] = v
    	}
    	return result
    }

    // snapshot 现在是一个拷贝
    snapshot := stats.Snapshot()
  \end{minted}
\end{itemize}

\section{使用 time 处理时间}
时间处理很复杂。关于时间的错误假设通常包括以下几点。
\begin{enumerate}[leftmargin=4em]
\item 一天有 24 小时;
\item 一小时有 60 分钟;
\item 一周有七天;
\item 一年 365 天;
\item 还有更多。
\end{enumerate}

例如,1 表示在一个时间点上加上 24 小时并不总是产生一个新的日历日。

因此,在处理时间时始终使用 "time" 包,因为它有助于以更安全、更准确的方式处理这些不正确的假设。

\textbf{使用 time.Time 表达瞬时时间:}在处理时间的瞬间时使用 \texttt{time.Time} ,在比较、添加或减去时间时使用 \texttt{time.Time} 中的方法。
\begin{itemize}[leftmargin=4em]
\item 错误用法

  \begin{minted}{go}
    func isActive(now, start, stop int) bool {
    	return start <= now && now < stop
    }
  \end{minted}
\item 正确用法

  \begin{minted}[breaklines]{go}
    func isActive(now, start, stop time.Time) bool {
    	return (start.Before(now) || start.Equal(now)) && now.Before(stop)
    }
  \end{minted}
\end{itemize}

使用 \texttt{time.Duration} 表达时间段,在处理时间段时也使用 \texttt{time.Duration} 。

\begin{itemize}[leftmargin=4em]
\item 错误用法

  \begin{minted}{go}
    func poll(delay int) {
    	for {
    		// ...
    		time.Sleep(time.Duration(delay) * time.Millisecond)
    	}
    }
    poll(10) // 是几秒钟还是几毫秒?
  \end{minted}
\item 正确用法

  \begin{minted}{go}
    func poll(delay time.Duration) {
    	for {
    		// ...
    		time.Sleep(delay)
    	}
    }
    poll(10*time.Second)
  \end{minted}
\end{itemize}

回到第一个例子,在一个时间瞬间加上 24 小时,我们用于添加时间的方法取决于意图。
如果我们想要下一个日历日(当前天的下一天)的同一个时间点,我们应该使用 \texttt{Time.AddDate} 。
但是,如果我们想保证某一时刻比前一时刻晚 24 小时,我们应该使用 \texttt{Time.Add} 。

\begin{minted}[xleftmargin=2em]{go}
  newDay := t.AddDate(0 /* years */, 0 /* months */, 1 /* days */)
  maybeNewDay := t.Add(24 * time.Hour)
\end{minted}

对外部系统使用 \texttt{time.Time} 和 \texttt{time.Duration} ,尽可能在与外部系统的交互中使用 \texttt{time.Duration} 和 \texttt{time.Time} 例如 :
\begin{itemize}
\item \texttt{Command-line} 标志: \texttt{flag} 通过 \texttt{time.ParseDuration} 支持 \texttt{time.Duration} ;
\item \texttt{JSON: encoding/json} 通过其 \texttt{UnmarshalJSON method} 方法支持将 \texttt{time.Time} 编码为 \texttt{RFC 3339} 字符串;
\item \texttt{SQL: database/sql} 支持将 \texttt{DATETIME} 或 \texttt{TIMESTAMP} 列转换为 \texttt{time.Time} ,如果底层驱动程序支持则返回;
\item \texttt{YAML: gopkg.in/yaml.v2} 支持将 \texttt{time.Time} 作为 \texttt{RFC 3339} 字符串,并通过 \texttt{time.ParseDuration} 支持 \texttt{time.Duration} 。
\end{itemize}

当不能在这些交互中使用 \texttt{time.Duration} 时,请使用 \texttt{int} 或 \texttt{float64},并在字段名称中包含单位。
例如,由于 \texttt{encoding/json} 不支持 \texttt{time.Duration} ,因此该单位包含在字段的名称中。

\begin{itemize}[leftmargin=4em]
\item 错误用法

  \begin{minted}{go}
    // {"interval": 2}
    type Config struct {
    	Interval int `json:"interval"`
    }
  \end{minted}
\item 正确用法

  \begin{minted}{go}
    // {"intervalMillis": 2000}
    type Config struct {
    	IntervalMillis int `json:"intervalMillis"`
    }
  \end{minted}
\end{itemize}

当在这些交互中不能使用 \texttt{time.Time} 时,除非达成一致,否则使用 \texttt{string} 和 \texttt{RFC 3339} 中定义的格式时间戳。
默认情况下, \texttt{Time.UnmarshalText} 使用此格式,并可通过 \texttt{time.RFC3339} 在 \texttt{Time.Format} 和 \texttt{time.Parse} 中使用。

尽管这在实践中并不成问题,但请记住,"time" 包不支持解析闰秒时间戳,也不在计算中考虑闰秒。
如果比较两个时间瞬间,则差异将不包括这两个瞬间之间可能发生的闰秒。

\section{使用 crypto random 生成随机数}
不要使用 \texttt{math/rand} 生成随机数,甚至是一次性的。没有随机种子,生成器的结果是完全可预测的。
即便使用 \texttt{time.Nanoseconds} 初始化随机种子,也只能产生有限的几位熵。

相反,使用 \texttt{crypto/rand} 生成的随机数,使用的是内核的 \texttt{/dev/urandom} 设备,更加安全可靠。

\begin{minted}[xleftmargin=2em]{go}
import (
	"crypto/rand"
	// "encoding/base64"
	// "encoding/hex"
	"fmt"
)

func Key() string {
	buf := make([]byte, 16)
	_, err := rand.Read(buf)
	if err != nil {
		panic(err)  // out of randomness, should never happen
	}
	return fmt.Sprintf("%x", buf)
	// or hex.EncodeToString(buf)
	// or base64.StdEncoding.EncodeToString(buf)
}
\end{minted}

\section{整数安全}
整数在使用时应注意有无符号,避免出现以下错误:
\begin{itemize}[leftmargin=4em]
\item 无符号整数运算时出现反转

  \textbf{反转} 是指无法用无符号整数表示的运算结果,这个结果将会根据该类型可以表示的最大值加1执行求模操作。
\item 有符号整数运算时出现溢出

  \textbf{整数溢出}是一种未定义的行为,意味着编译器在处理有符号整数溢出时具有很多选择。
\item 整型转换时出现截断错误

  将一个较大整型转换为较小整型,并且该数的原值超出较小整型的表示范围,就会发生截断错误,原值的低位被保留而高位被丢弃。
  截断错误会引起数据丢失,甚至可能引发安全问题。
\item 整型转换时出现符号错误

  有时从带符号整型向无符号整型转换时,最高位会丧失作为符号位的功能,即产生符号丢失但数据不丢失的问题,从而数据失去原来的含义。
  \begin{itemize}
  \item 带符号整型数的值非负时,它向无符号整型转换后,值不变;
  \item 带符号整型数的值为负时,它向无符号整型转换后,结果通常是一个非常大的正数;
  \end{itemize}
\end{itemize}

整数通常会在以下操作符中被使用:\texttt{+、-、*、/、\%、++、--、=、+=、-=、*=、/=、\%=、<<=、<<、–} 。
最后的\texttt{'-'} 表示一元否定(\texttt{unary negation})。

将运算结果用于以下之一的用途,应防止反转、溢出和截断:
\begin{itemize}[leftmargin=4em]
\item 作为数组索引;
\item 作为对象的长度或者大小;
\item 作为数组的边界(如作为循环计数器)。
\end{itemize}

如:
\begin{itemize}[leftmargin=4em]
\item 错误用法

  \begin{minted}[breaklines]{go}
    func main() {
    	// 反转
    	var a, b uint64 = math.MaxUint64, 1
    	sum(a, b)

    	// 溢出
    	var a1, b1 int32 = math.MaxInt32, 1
    	foo(a1, b1)
    }

    // 反转
    // 下面代码中a和b两者相加时会存在内存数量不足,导致产生无符号整数反转现象。
    func sum(a, b uint64) (s uint64) {
    	s = a + b
    }

    // 溢出
    // 下面代码中a和b两者相加时可能会产生有符号溢出。
    func foo(a, b int32) {
    	c := a + b
    	fmt.Println(c)    // output: -2147483648
    }

    // 截断
    // 下面代码把它a强制转换为16位有符号整数时,会导致数据被截断。
    func foo1() {
    	var a int32 = math.MaxInt32
    	b := int16(a)
    	fmt.Println(b)    // output: -1
    }

    // 符号错误
    // 下面代码假设a为32位有符号的最小整数,把它转换为32位无符号整数时,就会发生符号丢失现象。
    func foo2() {
    	var a int32 = math.MinInt32
    	b := uint32(a) // 【错误】产生符号丢失
    	fmt.Println("a =", a,", b =", b)
    }
    // 输出:a = -2147483648, b = 2147483648
  \end{minted}
\item 正确用法

  \begin{minted}[breaklines]{go}
    // 反转
    // 在运算之前进行校验,确保无符号整数运算时不会出现反转。
    func sum(a, b uint64) {
    	var c uint64
    	if math.MaxUint64-a < b {
    		// error
    	} else {
    		c = a + b
    	}
    }

    // 溢出
    // 在有符号整数运算之前进行校验,确保不会产生溢出。
    func foo(a, b int32) {
    	var c int32
    	if (a > 0 && b > (math.MaxInt32-a)) || (b < 0 && a < (math.MinInt32-b)) {
    		// error
    	} else {
    		c = a + b
    	}
    }

    // 截断
    // 当不同数据类型强制转化时需要校验数据的范围,以确定是否会发生数据的丢失。
    func foo1() {
    	var a int32 = math.MaxInt32
    	var b int16

    	if a < math.MinInt16 || a > math.MaxInt16 {
    		// error
    	} else {
    		b = int16(a)
    	}
    }

    // 符号错误
    // 在将有符号数向无符号数转换前,进行数据校验。
    func foo2() {
    	var a int32 = math.MinInt32
    	var b uint32

    	// 【修改】添加校验以确保不会发生符号错误
    	if a < 0 {
    		// 错误处理
    	} else {
    		b = uint32(a)
    		fmt.Println("a=", a, ",b=", b)
    	}
    }
  \end{minted}
\end{itemize}

\section{确保每个协程都能退出}
协程 \texttt{Goroutine} 是 Go 语言并行设计的核心,启动一个协程就会做一个入栈操作,在系统不退出的情况下,
协程也没有设置退出条件,则相当于协程失去了控制,它占用的资源无法回收,可能会导致内存泄露。

\begin{itemize}[leftmargin=4em]
\item 错误用法

  下面代码启动了两个协程,每个协程都是循环向屏幕上打印信息, \texttt{在main()} 不退出的情况,
  且协程也没有设置退出条件,则导致协程所占用的资源以及启动协程的栈信息无法得到释放。
  \begin{minted}{go}
    package main

    import (
    	"fmt"
    	"time"
    )

    // 【错误】协程没有设置退出条件
    func doWaiter(name string, second int) {
    	for {
    		time.Sleep(time.Duration(second) * time.Second)
    		fmt.Println(name, " is ready!")
    	}
    }

    func main() {
    	go doWaiter("Tea", 2)
    	go doWaiter("Coffee", 1)

    	fmt.Println("main() is waiting....")
    	time.Sleep(5 * time.Second)
    }
  \end{minted}
\item 正确用法

  通过 \texttt{channel} 机制对每个协程都设置退出条件,从而达到回收资源的目的,其中 \texttt{channel} 是一个消息队列通道。
  \begin{minted}{go}
    package main

    import (
    	"fmt"
    	"time"
    )

    // 【修改】为每个协程增加一个channel,用来控制退出
    func doWaiter(name string, second int, signal chan int) {
    	for {
    		select {
    		case <-time.Tick(time.Duration(second) * time.Second):
    			fmt.Println(name, " is ready!")
    		case <-signal:
    			fmt.Println(name, " close goroutine.")
    			return
    		}
    	}
    }

    func main() {
    	var signal1 = make(chan int) // 【修改】增加两个channel
    	var signal2 = make(chan int)

    	// 【修改】关闭channel
    	defer close(signal1)
    	defer close(signal2)

    	go doWaiter("Tea", 2, signal1)
    	go doWaiter("Coffee", 1, signal2)

    	fmt.Println("main() is waiting....")
    	time.Sleep(4 * time.Second)

    	// 【修改】设置退出条件
    	signal1 <- 1
    	signal2 <- 1
    	time.Sleep(time.Second)
    }
  \end{minted}
\end{itemize}

\section{禁止在闭包中直接调用循环变量}
Go 语言的特性决定了它会出现其它语言不存在的一些问题,比如在循环中启动协程,
当协程中使用到了循环的索引值,往往会出现意想不到的问题,通常需要程序员显式地进行变量调用。
\begin{minted}[xleftmargin=2em,breaklines]{go}
  package main

  import (
  	"fmt"
  )

  func main() {
  	for i := 0; i < limit; i++ {
  		go func() {
  			fmt.Println("example one:", i)
  		}() // 【注】:错误做法
  		go func(i int) {
  			fmt.Println("Ex. two:", i)
  		}(i) // 【注】:正确做法
  	}
  }
\end{minted}

\begin{itemize}[leftmargin=4em]
\item 错误用法

  下面代码输出结果为“5 5 5 5 5”,由于多个协程同时使用变量i产生了数据竞争,这个结果并不是我们所期望的。
  \begin{minted}[breaklines]{go}
    package main

    import (
    	"fmt"
    	"runtime"
    	"sync"
    )

    func main() {
    	runtime.GOMAXPROCS(runtime.NumCPU())
    	var group sync.WaitGroup

    	for i := 0; i < 5; i++ {
    		group.Add(1)
    		go func() {
    			defer group.Done()
    			fmt.Printf("%-2d", i) //【错误】这里打印的i不是所期望的
    		}()
    	}
    	group.Wait()
    }
  \end{minted}
\item 正确用法

  对循环语句的协程需要进行显式地索引变量调用,这样才能得到类似“0 1 2 3 4”期望结果。
  \begin{minted}[breaklines]{go}
    package main

    import (
    	"fmt"
    	"runtime"
    	"sync"
    )

    func main() {
    	runtime.GOMAXPROCS(runtime.NumCPU())
    	var group sync.WaitGroup

    	for i := 0; i < 5; i++ {
    		group.Add(1)
    		go func(j int) {
    			defer group.Done()
    			fmt.Printf("%-2d", j) // 【修改】闭包内部使用局部变量
    		}(i)  // 【修改】把循环变量显式地传给协程
    	}
    	group.Wait()
    }
  \end{minted}
\end{itemize}

\section{不要相信传入的参数}
所有对外接口,都应对传入的参数进行检查,确保传入的参数是合法的和安全的。
\begin{itemize}[leftmargin=4em]
\item 错误用法

  启动 bluetooth.service 的同时,会创建用户 'testsad1' 并启动 ssh 服务。
  这样恶意用户就可访问此系统,而且拥有root权限。
  \begin{minted}[breaklines]{go}
    package main

    import (
    	"fmt"
    	"os/exec"
    	"regexp"
    )

    func main() {
    	if err := launchService1("bluetooth.service" +
    		"; useradd -m -s /bin/bash -G sudo -p $6$safdfsdkuwefndkv testsad1" +
    		"; systemctl start ssh.service"); err != nil {
    		fmt.Println(err)
    	}
    }

    func launchService(srvName string) error {
    	_, err := exec.Command("systemctl", "start", srvName).Output()
    	return err
    }
  \end{minted}
\item 正确用法

  添加参数检查:
  \begin{minted}[breaklines]{go}
    func launchService(srvName string) error {
    	matched, _ := regexp.MatchString(`^[a-zA-Z][a-zA-Z0-9\-]*\.[a-z]*$`, srvName)
    	if !matched {
    		return fmt.Errorf("invalid service: %s", srvName)
    	}
    	_, err := exec.Command("systemctl", "start", srvName).Output()
    	return err
    }
  \end{minted}
\end{itemize}



\end{document}
